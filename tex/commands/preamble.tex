%\documentclass{amsart}
	\usepackage{amsthm}
	\usepackage{amsmath}
	% \usepackage{amssymb}
	\usepackage{esint}
	\usepackage{mathtools}
	\usepackage{tikz} % Custom diagrams
	\usepackage{tikz-cd} % Commutative diagrams
		\usetikzlibrary{arrows}
		\usetikzlibrary{knots, intersections, decorations.pathreplacing, decorations.pathmorphing, hobby}
		\tikzset{
			commutative diagrams/.cd,
			arrow style=tikz,
			diagrams={>=Computer Modern Rightarrow},
			path details/.style={ 
				every node/.style={
									midway, 
									sloped, 
									font=\footnotesize
								}
				}
		}
		% \usetikzlibrary{knots,intersections,decorations.pathreplacing,hobby}
		% \tikzset{
		%   path details/.style={
		%     >=stealth, 
		%     every node/.style={
		%       midway, 
		%       sloped, 
		%       font=\tiny
		%     },
		%   }
		% }
	\usepackage{tabu}

	% Figure support
	\usepackage{float}

	% Supplemental support - table of contents, appendix, bibliography, etc.
	% \usepackage{tocloft}
	% \usepackage[square, sort, numbers]{natbib}
	% 	\bibliographystyle{plainnat}
	\usepackage[title]{appendix}
	\usepackage{hyperref}
		\hypersetup{
			colorlinks,
			citecolor=black,
			filecolor=black,
			linkcolor=black,
			urlcolor=black
		}
	\usepackage{multicol}
	\usepackage{caption}
	\makeatletter

	% Math environment definitions
	\theoremstyle{plain} \newtheorem*{thm*}{\protect\theoremname}
	\theoremstyle{plain} \newtheorem{thm}{\protect\theoremname}
	\theoremstyle{definition} \newtheorem{defn}[thm]{\protect\definitionname}
	\theoremstyle{definition} \newtheorem*{defn*}{\protect\definitionname}
	\theoremstyle{definition} \newtheorem*{exercise*}{\protect\exercisename}
	\theoremstyle{definition} \newtheorem{exercise}{\protect\exercisename}%[section]
	\theoremstyle{definition} \newtheorem{claim}{\protect\claimname}
	\theoremstyle{definition} \newtheorem*{claim*}{\protect\claimname}
	\theoremstyle{plain} \newtheorem*{prop*}{\protect\propname}
	\theoremstyle{plain} \newtheorem{prop}{\protect\propname}
	\theoremstyle{plain} \newtheorem*{lem*}{\protect\lemmaname}
	\theoremstyle{plain} \newtheorem{lem}{\protect\lemmaname}
	\theoremstyle{plain} \newtheorem{cor}{\protect\corollaryname}
	\theoremstyle{plain} \newtheorem*{cor*}{\protect\corollaryname}
	\theoremstyle{definition} \newtheorem{remark}[thm]{\protect\remarkname}
	\theoremstyle{definition} \newtheorem*{remark*}{\protect\remarkname}
%   \theoremstyle{definition} \newtheorem{example}[thm]{\protect\examplename}
	\theoremstyle{definition} \newtheorem*{example*}{\protect\examplename}
%   \theoremstyle{definition} \newtheorem{problem}{\protect\problemname}
	\theoremstyle{definition} \newtheorem{concept}[thm]{\protect\conceptname}
	\theoremstyle{definition} \newtheorem*{concept*}{\protect\conceptname}

	\providecommand{\theoremname}{Theorem}
	\providecommand{\definitionname}{Definition}
	\providecommand{\exercisename}{Exercise}
	\providecommand{\claimname}{Claim}
	\providecommand{\propname}{Proposition}
	\providecommand{\lemmaname}{Lemma}
	\providecommand{\corollaryname}{Corollary}
	\providecommand{\remarkname}{Remark}
%  \providecommand{\examplename}{Example}
%  \providecommand{\problemname}{Problem}
	\providecommand{\conceptname}{Concept}

	\makeatother
