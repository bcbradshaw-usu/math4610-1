\documentclass[10pt,fleqn]{article}
\usepackage{hyperref}
\usepackage{graphicx}

\setlength{\topmargin}{-.75in}
\addtolength{\textheight}{2.00in}
\setlength{\oddsidemargin}{.00in}
\addtolength{\textwidth}{.75in}

\nofiles

\pagestyle{empty}

\setlength{\parindent}{0in}

% new math commands


\setlength{\oddsidemargin}{-0.25in}
\setlength{\evensidemargin}{-0.25in}
\setlength{\textwidth}{6.75in}
\setlength{\headheight}{0.0in}
\setlength{\topmargin}{-0.25in}
\setlength{\textheight}{9.00in}

\makeindex

\usepackage{mathrsfs}

%\usepackage[pdftex]{graphicx}
\usepackage{epstopdf}

\newcounter{beans}

\newcommand{\ds}{\displaystyle}
\newcommand{\limit}[2]{\displaystyle\lim_{#1\to#2}}

\newcommand{\binomial}[2]{\ \left( \begin{array}{c}
                                  #1 \\
                                  #2
                                 \end{array}
                            \right) \
                         }
\newcommand{\ExampleRule}[2]
  {
  \noindent
  \rule{\linewidth}{1pt}
  \begin{example}
    #1
    \label{#2}
  \end{example}
  \rule{\linewidth}{1pt}
  \vskip0.125in
  }

\newcommand{\defbox}[1]
  {
   \ \\
   \noindent
   \setlength\fboxrule{1pt}
   \fbox{
        \begin{minipage}{6.5in}
          #1
        \end{minipage}
        }
   \ \\
  }
\newcommand{\verysmallworkbox}[1]
  {
   \ \\
   \noindent
   \setlength\fboxrule{1pt}
   \fbox{
        \begin{minipage}{6.5in}
           #1
           \ \\
           \vskip0.5in \ \\
           \ \\
        \end{minipage}
        }
   \ \\
  }
\newcommand{\smallworkbox}[1]
  {
   \ \\
   \noindent
   \setlength\fboxrule{1pt}
   \fbox{
        \begin{minipage}{6.5in}
           #1
           \ \\
           \vskip2.5in \ \\
           \ \\
        \end{minipage}
        }
   \ \\
  }
\newcommand{\halfworkbox}[1]
  {
   \ \\
   \noindent
   \setlength\fboxrule{1pt}
   \fbox{
        \begin{minipage}{6.5in}
           #1 \hfill
           \ \\
           \vskip3.25in \ \\
           \ \\
        \end{minipage}
        }
   \ \\
  }
\newcommand{\largeworkbox}[1]
  {
   \ \\
   \noindent
   \setlength\fboxrule{1pt}
   \fbox{
        \begin{minipage}{6.5in}
           #1
           \ \\
           \vskip7.5in \ \\
           \ \\
        \end{minipage}
        }
   \ \\
  }
\newcommand{\flexworkbox}[2]
  {
   \ \\
   \noindent
   \setlength\fboxrule{1pt}
   \fbox{
        \begin{minipage}{6.5in}
           #1
           \ \\

           \vskip#2 \ \\
           \ \\
        \end{minipage}
        }
   \ \\
  }


% symbols for sets of numbers

\newcommand{\natnumb}{$\cal N$}
\newcommand{\whonumb}{$\cal W$}
\newcommand{\intnumb}{$\cal Z$}
\newcommand{\ratnumb}{$\cal Q$}
\newcommand{\irrnumb}{$\cal I$}
\newcommand{\realnumb}{$\cal R$}
\newcommand{\cmplxnumb}{$\cal C$}

% misc. commands

\newcommand{\mma}{{\it Mathematica}}
\newcommand{\sech}{\mbox{ sech}}
 
\newtheorem{theorem}{Theorem}
\newtheorem{example}{Example}
\newtheorem{definition}{Definition}
\newtheorem{problem}{Problem}

\setcounter{secnumdepth}{2}
\setcounter{tocdepth}{4}


\begin{document}
%%%%%%%%%%%%%%%%%%%%%%%%%%%%%%%%%%%%%%%%%%%%%%%%%%%%%%%%%%%%%%%%%%%%%%%%%%%%%%%%
%%%%%%%%%%%%%%%%%%%%%%%%%%%%%%%%%%%%%%%%%%%%%%%%%%%%%%%%%%%%%%%%%%%%%%%%%%%%%%%%
\vskip0.1in\hrule\vskip0.1in \noindent
{\bf{\Large Math 4610 Fundamentals of Numerical Analysis Tasksheet 3}}
\vskip0.1in\hrule\vskip0.1in \noindent
The problems for the Tasksheet 03 are included below. The deadline for turning
in your work on these problems will be posted on the repository. In addition,
you will turn in your work through the math4610 repository. A directory will be
constructed that will be used as a place to store your work.
\vskip0.1in\hrule\vskip0.1in \noindent
{\bf{\large Tasks}}
\vskip0.1in\hrule\vskip0.1in \noindent
\begin{trivlist}
  \item[\bf Task 1:] Computationally verify that the central difference
        approximation for the second derivative approximation defined in Task
        Sheet 2, Task 5 is actually second order accurate. Using values of
        \(h\neq 0\), compute the approximation and the difference between the
        exact value and the approximation. Print out the data in columns like
        those presented in the notes for the first order approximation of the
        the first derivative.
\vskip0.1in\hrule\vskip0.1in \noindent
  \item[\bf Task 2:] Produce a log-log plot of the error from the data in Task
        1. Discuss how the results show that the approximation is second order
        accurate over a range of value where \(h>0\). Also, determine when the
        approximation begins to fail due to finite precision in the number
        representation and/or finite precision of arithmetic. You will need to
        write a code that produces the plot. If needed, you can use the Python
        code in the notes. 
\vskip0.1in\hrule\vskip0.1in \noindent
  \item[\bf Task 3:] Write routines that will produce the machine epsilon as
        discussed in class. Write two separate routines, one for the single
        precision setting on your computer and one for double precision. Compile
        and test each of of these routines separately. This means writing a
        main() program in separate files. 
\vskip0.1in\hrule\vskip0.1in \noindent
  \item[\bf Task 4:] In this task you will be asked to begin a software manual
        for the various routines that you will be creating over the semester.
        The software manual will span the entire semester and should include a
        lot of your work. The following steps should help organize your work.
        \begin{trivlist}
           \item Within your terminal, create folders to include the software
                 pages in the math4610 repository. Create a folder using
                 something like
                 \begin{verbatim}

                     koebbe% mkdir software_manual

                 \end{verbatim}
                 Then create a file that will contain a table of contents for
                 pages that are created for individual routines. Maybe something
                 like
                 \begin{verbatim}

                     koebbe% cd software_manual
                     koebbe% vim software_manual_toc.md

                 \end{verbatim}
                 When you edit the file, create a unique link for each new
                 software page added. There is a
                 \href{../../../softwareManual/softwareManualTemplate.md}
                 {Markdown template} that provides an example of what is
                 \href{../audio/tps_cover.wav}{expected}. You can make a copy of
                 the templacte file in the folder created above. Then edit the
                 file to create a page for your code.
 
                 For example, create a page for the single precison
                 version of the code from Task 3.
                 \begin{verbatim}

                     koebbe% cp softwareManual/softwareManualTemplate.md smaceps.md

                 \end{verbatim}
                 or
                 \begin{verbatim}

                     koebbe% cp softwareManual/softwareManualTemplate.md smaceps.html
                     koebbe% vim smaceps.html

                 \end{verbatim}
                 depending on whether or not you want to use MathJax for math
                 notation.
          \item Build the links to the individual pages in the table of contents
                file by adding an entry to the table for each routine.
          \item Build the link from your README.md repository file at the top
                level. That is, add a line like
                \begin{verbatim}

  [Software Manual Table of Contents:](./software_manual/software_manual_toc.md)

                \end{verbatim}
                
        \end{trivlist}
        There are a lot of ways to do this. The steps above  should provide the
        structure you need. You might consider building a template for the pages
        from the page you can download and make copies from the template.
\vskip0.1in\hrule\vskip0.1in \noindent
  \item[\bf Task 5:] This task will walk you through the development of a shared
        library for your routines from the two routines written and tested in
        this tasksheet. Suppose that the two routines are named, smaceps() and
        dmaceps().
        \begin{trivlist}
          \item Create two files that contain only the code for the routines for
                the machine epsilon computation. No main() method is needed.
          \item Compile each of the files using an object name. For example
                \begin{verbatim}

                  koebbe% gcc -o smaceps smaceps.c
                  koebbe% gcc -o dmaceps dmaceps.c

                \end{verbatim}
                or
                \begin{verbatim}

                  koebbe% gcc -o *.c

                \end{verbatim}
                This will produce two files:
                \begin{verbatim}

                  koebbe% ls *.o
                  dmaceps.o  smaceps.o
                  koebbe%

                \end{verbatim}
          \item Created a shared library,
                \begin{verbatim}

                  koebbe% ar cvf mylib dmaceps.o smaceps.o

                \end{verbatim}
                This is the archive command in Unix and the \lq\lq\ cvf\rq\rq\
                informs the archive command to create, be verbose, and use the
                file name specified for the library. The output will be a file
                of the form
                \begin{verbatim}

                  koebbe% ls mylib.a
                  mylib.a
                
                \end{verbatim}
          \item Make the library a random access library.
                \begin{verbatim}

                  koebbe% ranlib mylib.a
                
                \end{verbatim}
                This last step allows a link/load command to access the routines
                in any order.
        \end{trivlist}
\vskip0.1in\hrule\vskip0.1in \noindent
  \item[\bf Task 6:] Search the internet for discussions of shared libraries.
        Write a brief summary of what you find including the pros and cons of
        shared libraries. Your write up should be a brief paragraph (3 or 4
        sentences) that describe your findings. Include links to the sites you
        cite.
\end{trivlist}
\vskip0.1in\hrule\vskip0.1in \noindent
  \href{../../tasksheet_02/html/tasksheet_02.html}{Previous}
  \href{../../toc/md/tasksheet_toc.md}{Table of Contents} |
  \href{../../tasksheet_04/html/tasksheet_04.html}{Next}
\vskip0.1in\hrule\vskip0.1in \noindent
%%%%%%%%%%%%%%%%%%%%%%%%%%%%%%%%%%%%%%%%%%%%%%%%%%%%%%%%%%%%%%%%%%%%%%%%%%%%%%%%
%%%%%%%%%%%%%%%%%%%%%%%%%%%%%%%%%%%%%%%%%%%%%%%%%%%%%%%%%%%%%%%%%%%%%%%%%%%%%%%%
\end{document}
