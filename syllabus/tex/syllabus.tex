\documentclass[10pt,fleqn]{article}

% new math commands


\setlength{\oddsidemargin}{-0.25in}
\setlength{\evensidemargin}{-0.25in}
\setlength{\textwidth}{6.75in}
\setlength{\headheight}{0.0in}
\setlength{\topmargin}{-0.25in}
\setlength{\textheight}{9.00in}

\makeindex

\usepackage{mathrsfs}

%\usepackage[pdftex]{graphicx}
\usepackage{epstopdf}

\newcounter{beans}

\newcommand{\ds}{\displaystyle}
\newcommand{\limit}[2]{\displaystyle\lim_{#1\to#2}}

\newcommand{\binomial}[2]{\ \left( \begin{array}{c}
                                  #1 \\
                                  #2
                                 \end{array}
                            \right) \
                         }
\newcommand{\ExampleRule}[2]
  {
  \noindent
  \rule{\linewidth}{1pt}
  \begin{example}
    #1
    \label{#2}
  \end{example}
  \rule{\linewidth}{1pt}
  \vskip0.125in
  }

\newcommand{\defbox}[1]
  {
   \ \\
   \noindent
   \setlength\fboxrule{1pt}
   \fbox{
        \begin{minipage}{6.5in}
          #1
        \end{minipage}
        }
   \ \\
  }
\newcommand{\verysmallworkbox}[1]
  {
   \ \\
   \noindent
   \setlength\fboxrule{1pt}
   \fbox{
        \begin{minipage}{6.5in}
           #1
           \ \\
           \vskip0.5in \ \\
           \ \\
        \end{minipage}
        }
   \ \\
  }
\newcommand{\smallworkbox}[1]
  {
   \ \\
   \noindent
   \setlength\fboxrule{1pt}
   \fbox{
        \begin{minipage}{6.5in}
           #1
           \ \\
           \vskip2.5in \ \\
           \ \\
        \end{minipage}
        }
   \ \\
  }
\newcommand{\halfworkbox}[1]
  {
   \ \\
   \noindent
   \setlength\fboxrule{1pt}
   \fbox{
        \begin{minipage}{6.5in}
           #1 \hfill
           \ \\
           \vskip3.25in \ \\
           \ \\
        \end{minipage}
        }
   \ \\
  }
\newcommand{\largeworkbox}[1]
  {
   \ \\
   \noindent
   \setlength\fboxrule{1pt}
   \fbox{
        \begin{minipage}{6.5in}
           #1
           \ \\
           \vskip7.5in \ \\
           \ \\
        \end{minipage}
        }
   \ \\
  }
\newcommand{\flexworkbox}[2]
  {
   \ \\
   \noindent
   \setlength\fboxrule{1pt}
   \fbox{
        \begin{minipage}{6.5in}
           #1
           \ \\

           \vskip#2 \ \\
           \ \\
        \end{minipage}
        }
   \ \\
  }


% symbols for sets of numbers

\newcommand{\natnumb}{$\cal N$}
\newcommand{\whonumb}{$\cal W$}
\newcommand{\intnumb}{$\cal Z$}
\newcommand{\ratnumb}{$\cal Q$}
\newcommand{\irrnumb}{$\cal I$}
\newcommand{\realnumb}{$\cal R$}
\newcommand{\cmplxnumb}{$\cal C$}

% misc. commands

\newcommand{\mma}{{\it Mathematica}}
\newcommand{\sech}{\mbox{ sech}}
 
\newtheorem{theorem}{Theorem}
\newtheorem{example}{Example}
\newtheorem{definition}{Definition}
\newtheorem{problem}{Problem}

\setcounter{secnumdepth}{2}
\setcounter{tocdepth}{4}


\usepackage[pdftex]{graphicx}
\usepackage{epstopdf}
\usepackage{hyperref}

\begin{document}

% Student Syllabus File
%
\pagestyle{empty}
\vskip0.1in\hrule\vskip0.1in \noindent
{\bf
   Math 4610 Fundamentals of Computational Mathematics
} \hfill Fall 2020
\vskip0.1in\hrule\vskip0.1in
\noindent
{\bf Instructor:} \hfill   Joe Koebbe \\
\smallskip\noindent
{\bf Office:}     \hfill   ANSC 209 \\
\smallskip\noindent
{\bf Office Hours:} \hfill MWF 8:00am-9:20am \\
\smallskip\noindent
{\bf \ } \hfill MWF 10:30am-11:20am \\
\smallskip\noindent
{\bf \ } \hfill MWF 1:00pm-2:00pm \\
\smallskip\noindent
{\bf Phone:}      \hfill   1-435-797-2825 \\
\smallskip\noindent
{\bf email:}      \hfill   joe.koebbe@usu.edu \\
\smallskip\noindent
{\bf webpage:}    \hfill   http://www.math.usu.edu/\~{}koebbe
\smallskip\noindent
{\bf Github repositories:}    \hfill   https://jvkoebbe.github.io/math4610

\smallskip\noindent

If you cannot make office hours, you can also set up an appointment. This can be
done before or after class meetings or by email. 
Students are always welcome to show
up without an appointment during office hours.
%%%%%%%%%%%%%%%%%%%%%%%%%%%%%%%%%%%%%%%%%%%%%%%%%%%%%%%%%%%%%%%%%%%%%%%%%%%%%%%%
%%%%%%%%%%%%%%%%%%%%%%%%%%%%%%%%%%%%%%%%%%%%%%%%%%%%%%%%%%%%%%%%%%%%%%%%%%%%%%%%
\vskip0.1in\hrule\vskip0.1in
\noindent
{\bf USU Math 4610 Fundamentals of Computational Mathematics Course
     Description:}
\vskip0.1in\hrule\vskip0.1in
\noindent
This course presents fundamental topics and algorithms that are common to many
areas of computational mathematics. Topics include: truncation and round-off
error, basic numerical linear algebra (including Gaussian elimination and
calculation of eigenvalues), root-finding methods, interpolation methods, and
numerical differentiation/integration. Prerequisite: Math 1220 with a C- or
better in Math 2250 or 2270 with a C- or better in a high-level programming
language (C/C++, Python, Fortran) 
%%%%%%%%%%%%%%%%%%%%%%%%%%%%%%%%%%%%%%%%%%%%%%%%%%%%%%%%%%%%%%%%%%%%%%%%%%%%%%%%
%%%%%%%%%%%%%%%%%%%%%%%%%%%%%%%%%%%%%%%%%%%%%%%%%%%%%%%%%%%%%%%%%%%%%%%%%%%%%%%%
\vskip0.1in\hrule\vskip0.1in
\noindent
{\bf Computer Programming Requirements:}
\vskip0.1in\hrule\vskip0.1in
\noindent
If you have not completed a computer programming class, for example USU courses
CS 1400 Introduction to Computer Science - CS 1 and/or CS 1410 Introduction to
Computer Science - CS 2, you will {\bf NOT} be able to complete this course.
This course is not intended to be a platform for learning a computer programming
language. This course is intended to provide students with an opportunity to
apply an already mastered computer programming skill set to algorithm
development and implementation. Students {\bf MUST} be able to write computer
programs on the first day of class - {\bf NO EXCEPTIONS}.
%%%%%%%%%%%%%%%%%%%%%%%%%%%%%%%%%%%%%%%%%%%%%%%%%%%%%%%%%%%%%%%%%%%%%%%%%%%%%%%%
%%%%%%%%%%%%%%%%%%%%%%%%%%%%%%%%%%%%%%%%%%%%%%%%%%%%%%%%%%%%%%%%%%%%%%%%%%%%%%%%
\vskip0.1in\hrule\vskip0.1in
\noindent
{\bf Textbook Information:}
\vskip0.1in\hrule\vskip0.1in
\noindent
There is no required textbook for Math 4610. Instead the course materials are
made freely available to students enrolled in Math 4610. The materials are
posted at the following online location:
\begin{verbatim}

    https://jvkoebbe.github.io/math4610/
 
\end{verbatim}
\noindent
The materials are available to any person who is interested in learning about
computational mathematics. There are a number of textbooks and other materials
that might be useful in working through the materials Github available at the
link given above. Note that the materials at the link above are governed by a
Creative Commons License. This license allows free use of the materials for
noncommercial purposes. This includes any derivative works from the materials at
the given site.
%%%%%%%%%%%%%%%%%%%%%%%%%%%%%%%%%%%%%%%%%%%%%%%%%%%%%%%%%%%%%%%%%%%%%%%%%%%%%%%%
%%%%%%%%%%%%%%%%%%%%%%%%%%%%%%%%%%%%%%%%%%%%%%%%%%%%%%%%%%%%%%%%%%%%%%%%%%%%%%%%
\vskip0.1in\hrule\vskip0.1in
\noindent
\underline{Additional References:}
\vskip0.1in\hrule\vskip0.1in
\noindent
There will be a number of topics that might arise during the semester. At the
instructor's discretion any materials needed for these topics will be provided
either in class, through handouts, or through posts to the instructor's main
web page.
\begin{verbatim}

  http://www.math.usu.edu/~koebbe

\end{verbatim}
Note that this syllabus is posted at both the instructor's site and also on the
Github site with link given above. Also, note that the syllabus will change from
year to year based on how the course evolves and the instructor teaching the
course in a current semester.
%%%%%%%%%%%%%%%%%%%%%%%%%%%%%%%%%%%%%%%%%%%%%%%%%%%%%%%%%%%%%%%%%%%%%%%%%%%%%%%%
%%%%%%%%%%%%%%%%%%%%%%%%%%%%%%%%%%%%%%%%%%%%%%%%%%%%%%%%%%%%%%%%%%%%%%%%%%%%%%%%
\vskip0.1in\hrule\vskip0.1in \noindent
{\bf Additional Computer Skills:}
\vskip0.1in\hrule\vskip0.1in
\noindent
During the semester, students will be required to use a few more computational
tools to complete work. These include, but are not limited to the following
list:

\medskip

\noindent
\begin{list}{$\bullet$}{\usecounter{beans} \parsep=0pt \listparindent=0pt
\topsep=0pt \rightmargin=.35in \leftmargin=.35in  \labelsep=5 pt \itemsep=2pt}
  \item {\bf Command Line Terminals} Students will be required to use
        Linux/Unix terminals to complete many tasks. Basic usage will be
        covered in class and a Linux/Unix emulator (Cygwin) has been installed
        for public use in the Engineering Lab on campus. Students that have a
        Linux box or are able to install Cygwin on a personal laptop will be a
        bit better off.
  \item {\bf git:} Students will need to master the use of \lq\lq\ git\rq\rq\
        as a part of the class. Note that git is automatically installed with
        Cygwin. So, git is available on the Engineering Computer lab. 
  \item {\bf Github Account:} Students must be able to create a Github account
        to interface work in the class with a publicly available repository.
        This will be covered in class.
  \item {\bf Graphical Output:} Students must be able to produce graphs and
        and other visual aids for homework tasks. This will be covered in class.
\end{list}

\medskip
%%%%%%%%%%%%%%%%%%%%%%%%%%%%%%%%%%%%%%%%%%%%%%%%%%%%%%%%%%%%%%%%%%%%%%%%%%%%%%%%
%%%%%%%%%%%%%%%%%%%%%%%%%%%%%%%%%%%%%%%%%%%%%%%%%%%%%%%%%%%%%%%%%%%%%%%%%%%%%%%%
\vskip0.1in\hrule\vskip0.1in \noindent
{\bf Instructor Course Comments:}
\vskip0.1in\hrule\vskip0.1in
\noindent
This course is intended for advanced undergraduate students and graduate
students who have already taken Math 2270 Linear Algebra {\bf and} Math 2280
Ordinary Differential Equations {\bf or} Math 2250 Linear Algebra and Ordinary
Differential Equations. The content will contain methods for the approximate
solution of simple mathematical problems through algorithms implemented into a
high level computer programming language. The basic mathematics skill set needed
for this course provides the backbone for computational mathematics and more
advanced courses in numerical analysis and computational mathematics. However,
you should not be enrolled in the course if you have not learned to write
computer programs in a high level programming language (e.g, C, C++, Python,
Fortran, or Java). If you have not taken a course in computer science, you will
need to drop the course and complete a course like CS 1410 at USU.

\medskip

\noindent
\begin{list}{$\bullet$}{\usecounter{beans} \parsep=0pt \listparindent=0pt
\topsep=0pt \rightmargin=.35in \leftmargin=.35in  \labelsep=5 pt \itemsep=2pt}
  \item The course meets three days per week. We will be covering a lot of
        material in each and every class meeting. 
  \item Although role will not be taken, students are required to attend this
        course in the sense that lectures will not be repeated. Students are
        expected to attend all classes and participate in discussions during
        class time. Time will be taken in each lecture to answer questions about
        material previously presented.
  \item It is highly recommended that students take notes during class to keep
        up with the material presented and to keep up with the lectures.
  \item Be on time! Most students in this course will take professional jobs
        that require an employee to be on time. This course is no different.
\end{list}

\medskip

\noindent
If you have any questions about the course material, course policies, or any
other matters, please contact the instructor for the course via email, before
or after class, or during office hours.
%%%%%%%%%%%%%%%%%%%%%%%%%%%%%%%%%%%%%%%%%%%%%%%%%%%%%%%%%%%%%%%%%%%%%%%%%%%%%%%%
%%%%%%%%%%%%%%%%%%%%%%%%%%%%%%%%%%%%%%%%%%%%%%%%%%%%%%%%%%%%%%%%%%%%%%%%%%%%%%%%
\vskip0.1in\hrule\vskip0.1in
\noindent
{\bf Why Not Matlab or Mathematica:}
\vskip0.1in\hrule\vskip0.1in
\noindent
Students often ask questions about Matlab and/or Wolfram Alpha as options for
writing programs. There are two important reasons for not allowing Matlab as a
programming language. These are the following:
\begin{enumerate}
\item First, in many of the intrinsic routines, most of the details are buried
      in the code written by people employed at the Math Works, the people who
      produce Matlab. This means that students will not see the nuts and bolts
      of how numerical algorithms work.
\item {\bf Cost!!!} While at USU, the cost for licensing Matlab or Maple is
      more than reasonable. However, corporate licensing is very expensive. It
      is highly unlikely that any company will want to license another product
      along with their own code. Students need to know how to write their own
      code, including the hidden details in intrinsic routines. This course
      assumes students want to learn these details.
\end{enumerate}
Note that Matlab or Maple are great tools for prototyping ideas in applied and
computational mathematics. Your instructor works in Matlab occasionally and used
to use Mathematica extensively. However, your instructor never needed to
purchase a license for these products. These tools were purchased within the
structure of a grant proposal or by USU IT.
%%%%%%%%%%%%%%%%%%%%%%%%%%%%%%%%%%%%%%%%%%%%%%%%%%%%%%%%%%%%%%%%%%%%%%%%%%%%%%%%
%%%%%%%%%%%%%%%%%%%%%%%%%%%%%%%%%%%%%%%%%%%%%%%%%%%%%%%%%%%%%%%%%%%%%%%%%%%%%%%%
\vskip0.1in\hrule\vskip0.1in
\noindent
{\bf Grading:}
\vskip0.1in\hrule\vskip0.1in
\noindent
Your grade in the course will be determined by the following:
\begin{enumerate}
\item Each student is required to meet with the instructor in the instructor's
      office during the first two weeks of the semester and prior to each of the
      exams described below during posted office hours or by making an
      appointment. This will count as one homework assignment.
\item Homework will account for a 35\% of a student's grade. Homework must be
      turned in on time. {\bf Under normal conditions late homework will not be
      accepted.} If you must be gone, any homework must be turned in on or
      before the due date. The only exception to this rule is for a family
      emergency. Also, homework must legible. Much of your work will be done
      using \lq\lq\ git\rq\rq\ and made available for grading using Github. This
      will mean that most of your homework will be typed. There will be times
      when a scanned, handwritten solution will be appropriate. Make sure these
      solutions are legible and the scan of the work is readable.
\item Students will be required to create a software manual to document the
      algorithms developed during the semester. This will account for \%15 of
      the overall grade in the course. The software manual will be comprised of
      short descriptions of the algorithms developed and implemented in this
      class. For each piece of code that is developed in the class there must be
      an entry in your software manual. We will discuss this in some detail
      during the first week of the semester.
\item Two midterms will be given. Each will cover about one half of the content
      of the course. Each of these midterms will account for 15\% or 30\% of the
      grade earned in the course.
\item A comprehensive exam will be given during finals week. The final will
      account for \%20 percent of the overall grade in the course.
\end{enumerate}
If students have questions about assignments, the software manual, midterms, 
and/or final please contact your instructor. One of the assignments will be to
meet with me at least once before each exam to discuss progress in the course.
This means that 3 times during the semester you will need to show up during
office hours or make an appointment to see your instructor.
%%%%%%%%%%%%%%%%%%%%%%%%%%%%%%%%%%%%%%%%%%%%%%%%%%%%%%%%%%%%%%%%%%%%%%%%%%%%%%%%
%%%%%%%%%%%%%%%%%%%%%%%%%%%%%%%%%%%%%%%%%%%%%%%%%%%%%%%%%%%%%%%%%%%%%%%%%%%%%%%%
\vskip0.1in\hrule\vskip0.1in
\newpage
%%%%%%%%%%%%%%%%%%%%%%%%%%%%%%%%%%%%%%%%%%%%%%%%%%%%%%%%%%%%%%%%%%%%%%%%%%%%%%%%
%%%%%%%%%%%%%%%%%%%%%%%%%%%%%%%%%%%%%%%%%%%%%%%%%%%%%%%%%%%%%%%%%%%%%%%%%%%%%%%%
\vskip0.1in\hrule\vskip0.1in
\noindent
{\bf University Policies:}
\vskip0.1in\hrule\vskip0.1in
\paragraph{\underline{Instructor/Department/University Policy Documents:}}
Any policies in this document are superseded by the online policies at the
address given above.

\noindent
There are a number of Utah State University (USU) policies that apply to
students, faculty, and administrators at USU. Students should realize that any
policies stated in this syllabus are included for convenience. The official
policies applied to this course can be found online at the following address.
\begin{verbatim}

http://catalog.usu.edu/content.php?catoid=12&navoid=3587

\end{verbatim}

\paragraph{\underline{Academic Honesty/Integrity}} The University expects that
students and faculty alike maintain the highest standards of academic honesty.
For the benefit of students who may not be aware of specific standards of the
University concerning academic honesty, the following information is quoted from
The Code of Policies and Procedures for Students at Utah State University
(revised September 2009), Article VI, Section 1:

\paragraph{\underline{Section 1. University Standard: Academic Integrity}}
Students have a responsibility to promote academic integrity at the University
by not participating in or facilitating others' participation in any act of
academic dishonesty and by reporting all violations or suspected violations of
the Academic Integrity Standard to their instructors.

\paragraph{\underline{The Honor Pledge:}} To enhance the learning environment at
Utah State University and to develop student academic integrity, each student
agrees to the following Honor Pledge: "I pledge, on my honor, to conduct myself
with the foremost level of academic integrity."

\paragraph{\underline{Violations of the Academic Integrity Standard (academic
violations):}} This includes, but are not limited to:

\vskip0.1in\noindent
Cheating:
\vskip0.1in\noindent
\begin{enumerate}
  \item using or attempting to use or providing others with any unauthorized
        assistance in taking quizzes, tests, examinations, or in any other
        academic exercise or activity, including working in a group when the
        instructor has designated that the quiz, test, examination, or any other
        academic exercise or activity be done "individually";
  \item depending on the aid of sources beyond those authorized by the
        instructor in writing papers, preparing reports, solving problems, or
        carrying out other assignments;
  \item substituting for another student, or permitting another student to
        substitute for oneself, in taking an examination or preparing academic
        work;
  \item acquiring tests or other academic material belonging to a faculty
        member, staff member, or another student without express permission;
  \item continuing to write after time has been called on a quiz, test,
        examination, or any other academic exercise or activity;
  \item submitting substantially the same work for credit in more than one
        class, except with prior approval of the instructor; or
  \item engaging in any form of research fraud.
\end{enumerate}

\paragraph{\underline{Falsification:}} altering or fabricating any information
or citation in an academic exercise or activity.

\paragraph{\underline{Plagiarism:}} representing, by paraphrase or direct
quotation, the published or unpublished work of another person as one's own in
any academic exercise or activity without full and clear acknowledgment. It
also includes using materials prepared by another person or by an agency engaged
in the sale of term papers or other academic materials.

\paragraph{\underline{Section 2. Reporting Violations of Academic Integrity}}
The Academic Integrity Violation Form (AIVF) provides guidance to instructors
and students, ensures minimum due process requirements are met, and allows
tracking of repeat offenders at the University level. The AIVF is available
through the Office of the Vice President for Student Affairs.

Once an instructor has determined that an academic violation has occurred and
that a sanction is appropriate, an AIVF must be submitted prior to application
of the sanction. The student may appeal the determination that an academic
violation occurred if the AIVF is not filed.

All submitted AIVF forms are kept in the Vice President for Student Affairs
Office for the duration of the student's academic career at Utah State
University. When a resolution has been reached between the student and
instructor, a Resolution Report detailing the action taken and agreement of
both parties on that action shall be submitted to the Office of the Vice
President for Student Affairs. If no Resolution Report has been filed for a
submitted AIVF within the semester, the Campus Judicial Officer will investigate
to determine if a solution was reached and why no Resolution Report was filed.

\paragraph{\underline{Section 3. Discipline Regarding Academic Integrity
Violations}} An instructor has full autonomy to evaluate a student's academic
performance in a course. If a student commits an academic violation, the
instructor may sanction the student. Such sanctions may include:
\begin{enumerate}
  \item requiring the student to rewrite a paper/assignment or to retake a
        test/examination;
  \item adjusting the student's grade—for either an assignment/test or the
        course;
  \item giving the student a failing grade for the course; or
  \item taking actions as appropriate. Additional disciplinary action beyond
        instructor sanction shall be determined by the Judicial Officer and the
        University.
\end{enumerate}

The penalty that the University will impose on a student for the first Academic
Integrity violation is placement on academic integrity probation after the first
offense.

The penalties that the University may impose on a student for multiple or
egregious academic integrity violations are:
\begin{enumerate}
  \item Probation: continued participation in an academic program predicated
        upon the student satisfying certain requirements as specified in a
        written notice of probation. Probation is for a designated period of
        time and includes the probability of more severe disciplinary penalties
        if the student does not comply with the specified requirements or is
        found to be committing academic integrity violations during the
        probationary period. The student must request termination of the
        probation in writing.
  \item Performance of community service.
  \item Suspension: temporary dismissal from an academic program or from the
        University for a specified time, after which the student is eligible to
        continue the program or return to the University. Conditions for
        continuance or readmission may be specified.
  \item Expulsion: permanent dismissal either from an academic program or from
        the University.
  \item Assigning a designation with a course grade indicating an academic
        integrity violation involving academic integrity. Conditions for removal
        may be specified, but the designation remains on the student's
        transcript for a minimum of one year; provided however, that once the
        student's degree is posted to the transcript, the designation may not be
        removed thereafter.
  \item Denial or revocation of degrees.
\end{enumerate}
\vskip0.1in\hrule\vskip0.1in\noindent
The complete Code of Policies and Procedures for Students at Utah State
University can be viewed at: The code of policies and procedures for students at
Utah State University.
\vskip0.1in\hrule\vskip0.1in
\newpage
%%%%%%%%%%%%%%%%%%%%%%%%%%%%%%%%%%%%%%%%%%%%%%%%%%%%%%%%%%%%%%%%%%%%%%%%%%%%%%%%
%%%%%%%%%%%%%%%%%%%%%%%%%%%%%%%%%%%%%%%%%%%%%%%%%%%%%%%%%%%%%%%%%%%%%%%%%%%%%%%%
\vskip0.1in\hrule\vskip0.1in
\noindent
{\bf Policies Due to Covid-19:}
\vskip0.1in\hrule\vskip0.1in \noindent
During the Covid-19 pandemic, and prior to resuming on-site operations, all USU
departments must have an approved department operation plan to minimize the
risk of exposure to employees, students, and the public. 

\paragraph{\underline{Classroom Safety Guidelines}}

\noindent
  \underline{Social Distancing of Six Feet:} Students and faculty should
            maintain 6 feet of separation from each other.
    \begin{enumerate}
      \item Each classroom has a social distancing cap (SD cap), indicating the
            greatest number of students allowed in the room.
      \item Seats will be marked indicating where to sit.
      \item An instructional space will be marked at the front of the room.

      \item Instructors are asked to dismiss their students in an orderly manner
            to ensure distancing while leaving the room.
    \end{enumerate}

\noindent
  \underline{Face Coverings Worn by Everyone} All students and faculty are
            required to wear a face covering in the classroom.

    \begin{enumerate}
      \item If students come to class without face coverings, teachers will be
            encouraged to remind students of the university policy.
      \item If students continue to disregard the policy, teachers can submit
            their names to the Provost's office for further steps.
    \end{enumerate}

\noindent
  \underline{Surface Cleaning Before Each Class} All surfaces are to be cleaned
            at the start of each class.
      \begin{enumerate}
        \item Cleaning materials will be provided in each classroom.
        \item Teachers and students are to wipe down chairs, desks, teaching,
              and work stations at the beginning of each class.
        \item USU Facilities staff will clean surfaces more thoroughly after
              hours of operation. 
      \end{enumerate}

\noindent
    \underline{Attendance Tracking} Teachers are to keep track of classroom
              attendance.
      \begin{enumerate}
        \item Attendance is important for purposes of contact tracing in the
              event of a student receiving a positive Covid-19 test result.
        \item Teachers should not pass around a sign-up sheet, but rather mark
              student attendance.
        \item Automated procedures for collecting attendance are being
              developed.
      \end{enumerate}

\noindent
    \underline{Health and hygiene practices} Students and faculty should stay
              home when ill and use good hygiene habits. 
      \begin{enumerate}
        \item Each classroom will have hand sanitizer.
        \item Students should cover coughs and sneezes and wash and disinfect
              their hands regularly.
        \item Students and faculty exhibiting even mild illness symptoms should
              stay home. 
      \end{enumerate}

\noindent
    \underline{Syllabus Language} The Provost's office will be the main resource
              for providing standardized language to include in the course
              syllabus. That language will be posted on the Provost's web page
              when it is available.
%%%%%%%%%%%%%%%%%%%%%%%%%%%%%%%%%%%%%%%%%%%%%%%%%%%%%%%%%%%%%%%%%%%%%%%%%%%%%%%%
%%%%%%%%%%%%%%%%%%%%%%%%%%%%%%%%%%%%%%%%%%%%%%%%%%%%%%%%%%%%%%%%%%%%%%%%%%%%%%%%
\end{document}
