\documentclass[10pt,fleqn]{article}
%\usepackage{graphicx}


\setlength{\topmargin}{-.75in}
\addtolength{\textheight}{2.00in}
\setlength{\oddsidemargin}{.00in}
\addtolength{\textwidth}{.75in}

\title{Math 4610 Lecture Notes \\
            \ \\
       Functions as Arguments in Coding Languages
  \footnote{These notes are part of an Open Resource Educational project
            sponsored by Utah State University}}

\author{Joe Koebbe}

\nofiles

\pagestyle{empty}

\setlength{\parindent}{0in}

% new math commands


\setlength{\oddsidemargin}{-0.25in}
\setlength{\evensidemargin}{-0.25in}
\setlength{\textwidth}{6.75in}
\setlength{\headheight}{0.0in}
\setlength{\topmargin}{-0.25in}
\setlength{\textheight}{9.00in}

\makeindex

\usepackage{mathrsfs}

%\usepackage[pdftex]{graphicx}
\usepackage{epstopdf}

\newcounter{beans}

\newcommand{\ds}{\displaystyle}
\newcommand{\limit}[2]{\displaystyle\lim_{#1\to#2}}

\newcommand{\binomial}[2]{\ \left( \begin{array}{c}
                                  #1 \\
                                  #2
                                 \end{array}
                            \right) \
                         }
\newcommand{\ExampleRule}[2]
  {
  \noindent
  \rule{\linewidth}{1pt}
  \begin{example}
    #1
    \label{#2}
  \end{example}
  \rule{\linewidth}{1pt}
  \vskip0.125in
  }

\newcommand{\defbox}[1]
  {
   \ \\
   \noindent
   \setlength\fboxrule{1pt}
   \fbox{
        \begin{minipage}{6.5in}
          #1
        \end{minipage}
        }
   \ \\
  }
\newcommand{\verysmallworkbox}[1]
  {
   \ \\
   \noindent
   \setlength\fboxrule{1pt}
   \fbox{
        \begin{minipage}{6.5in}
           #1
           \ \\
           \vskip0.5in \ \\
           \ \\
        \end{minipage}
        }
   \ \\
  }
\newcommand{\smallworkbox}[1]
  {
   \ \\
   \noindent
   \setlength\fboxrule{1pt}
   \fbox{
        \begin{minipage}{6.5in}
           #1
           \ \\
           \vskip2.5in \ \\
           \ \\
        \end{minipage}
        }
   \ \\
  }
\newcommand{\halfworkbox}[1]
  {
   \ \\
   \noindent
   \setlength\fboxrule{1pt}
   \fbox{
        \begin{minipage}{6.5in}
           #1 \hfill
           \ \\
           \vskip3.25in \ \\
           \ \\
        \end{minipage}
        }
   \ \\
  }
\newcommand{\largeworkbox}[1]
  {
   \ \\
   \noindent
   \setlength\fboxrule{1pt}
   \fbox{
        \begin{minipage}{6.5in}
           #1
           \ \\
           \vskip7.5in \ \\
           \ \\
        \end{minipage}
        }
   \ \\
  }
\newcommand{\flexworkbox}[2]
  {
   \ \\
   \noindent
   \setlength\fboxrule{1pt}
   \fbox{
        \begin{minipage}{6.5in}
           #1
           \ \\

           \vskip#2 \ \\
           \ \\
        \end{minipage}
        }
   \ \\
  }


% symbols for sets of numbers

\newcommand{\natnumb}{$\cal N$}
\newcommand{\whonumb}{$\cal W$}
\newcommand{\intnumb}{$\cal Z$}
\newcommand{\ratnumb}{$\cal Q$}
\newcommand{\irrnumb}{$\cal I$}
\newcommand{\realnumb}{$\cal R$}
\newcommand{\cmplxnumb}{$\cal C$}

% misc. commands

\newcommand{\mma}{{\it Mathematica}}
\newcommand{\sech}{\mbox{ sech}}
 
\newtheorem{theorem}{Theorem}
\newtheorem{example}{Example}
\newtheorem{definition}{Definition}
\newtheorem{problem}{Problem}

\setcounter{secnumdepth}{2}
\setcounter{tocdepth}{4}


\begin{document}
\maketitle
\newpage

%%%%%%%%%%%%%%%%%%%%%%%%%%%%%%%%%%%%%%%%%%%%%%%%%%%%%%%%%%%%%%%%%%%%%%%%%%%%%%%%
%%%%%%%%%%%%%%%%%%%%%%%%%%%%%%%%%%%%%%%%%%%%%%%%%%%%%%%%%%%%%%%%%%%%%%%%%%%%%%%%
\vskip0.1in\hrule\vskip0.1in
\noindent
{\bf Functions as Arguments in Coding Languages: An Introduction}
\vskip0.1in\hrule\vskip0.1in
\noindent
A common programming problem involves using functions or methods as arguments
in a standalone subroutine, method, or other coding construct. Each and every
language or platform (for example, Matlab) has one or two ways to address this
oissue. In this section, we will go through a few (but not all) programming
languages to document possible ways to use functions as arguments. A simple
example will be used to illustrate how to succeed at this. In particular, a
machine epsilon code will be used. In C, the code will look like the following.
\vskip0.1in\hrule\vskip0.1in
\begin{verbatim}

double maxeps()
{
  double one = 1.0;
}

\end{verbatim}
\vskip0.1in\hrule\vskip0.1in
\newpage
%%%%%%%%%%%%%%%%%%%%%%%%%%%%%%%%%%%%%%%%%%%%%%%%%%%%%%%%%%%%%%%%%%%%%%%%%%%%%%%%
%%%%%%%%%%%%%%%%%%%%%%%%%%%%%%%%%%%%%%%%%%%%%%%%%%%%%%%%%%%%%%%%%%%%%%%%%%%%%%%%
\vskip0.1in\hrule\vskip0.1in
\noindent
{\bf Functions as Arguments in Coding Languages: C}
\vskip0.1in\hrule\vskip0.1in
\noindent
There are a number of ways in the C porgramming language to pass in a function
as an argument. As an 
\newpage
%%%%%%%%%%%%%%%%%%%%%%%%%%%%%%%%%%%%%%%%%%%%%%%%%%%%%%%%%%%%%%%%%%%%%%%%%%%%%%%%
%%%%%%%%%%%%%%%%%%%%%%%%%%%%%%%%%%%%%%%%%%%%%%%%%%%%%%%%%%%%%%%%%%%%%%%%%%%%%%%%
\vskip0.1in\hrule\vskip0.1in
\noindent
{\bf Functions as Arguments in Coding Languages: Java}
\vskip0.1in\hrule\vskip0.1in
\noindent
Due to the object oriented mature of the Java programming language, it is easy
to create objects that contain a function. These notes will point the reader in
a direction that will work. Suppose that we decide to apply Newton's method to
approximate the roots of a function like
\[
  f(x) = sin(\pi + 10\ x^2)
\]
from some initial point. We will need to encode the function above and it's
derivative
\[
  f'(x) = 20\ x\ cos(\pi + 10\ x^2)
\]
which is obtained using an application of the chain rule.

The following code (in Java) implements Newton's method for a generic function.
\vskip0.1in\hrule\vskip0.1in
\begin{verbatim}

     //
     // set the class instantiation
     // ---------------------------
     //
     public class SimpleNewtonMethod extends Object {
       //
       // start with the function evaluation itself
       // -----------------------------------------
       //
       public static double newt(FunctionObject fo,
                                 double x0,
                                 double tol,
                                 int maxit) {
         double error  = 10.0 * tol;
         int iter = 0;
         double xold = x0;
         double xnew = x0;
         //
         // start the iteration
         // -------------------
         //
         while(error > tol  && iter < maxit) {
           iter++;
           xnew = xold - fo.fval(xold) / fo.dfval(xold);
           error = Math.abs(xnew - xold);
           xold = xnew;
         }
         return xnew;
       }
     }

\end{verbatim}
\vskip0.1in\hrule\vskip0.1in
You should notice that there is another object that needs to be included in the
compilation before this code will work. The function object coded up in this
example might look like the following.
\vskip0.1in\hrule\vskip0.1in
\begin{verbatim}

     //
     // set the class instantiation
     // ---------------------------
     //
     public class FunctionObject extends Object {
       //
       // start with the function evaluation itself
       // -----------------------------------------
       //
       public static double fval(double x) {
         fval = Math.sin( Math.PI + 10.0 * x * x );
         return fval;
       }
       //
       // the following method will compute the derivative of the function at an
       // arbitrary point
       // ---------------
       //
       public static double dfval(double x) {
         dfval = 20.0 * x * Math.cos( Math.PI + 10.0 * x * x );
         return dfval;
       }
       //
       // the following method will compute the second derivative of the function at
       // an arbitrary point
       // ------------------
       //
       public static double df2dval(double x) {
         df2val = 20.0 * Math.cos( Math.PI + 10.0 * x * x )
                 - 400.0 * x * x * Math.sin( Math.PI + 10.0 * x * x );
         return df2val;
       }
       //
       // local variables
       // ---------------
       //
       private static double fval;
       private static double dfval;
       private static double df2val;

     }

\end{verbatim}
\vskip0.1in\hrule\vskip0.1in
Notice that the class is used to hold the information about the function being
analyzed. In the code there is (1) a function evaluation, (2) an evaluation of
the derivative, and (3) an evaluation of the second derivative for extra 
information about the function.

It is not difficult to exten this idea to other classes in the Java programming
language. The only thing that needs to be changed in analyzing functions is the
function definition and the definition of the derivative(s). Note that in order
for the Newton method code to see the FunctionObject, you must place the two
files containing the code in the same folder and then compile the
SimpleNewtonMethod object using
\vskip0.1in\hrule\vskip0.1in
\begin{verbatim}

     javac SimpleNewtonMethod.java

\end{verbatim}
The compiler will realize there is a dependency and look for the file containing
the FunctionObject class in an appropriately named file.
\vskip0.1in\hrule\vskip0.1in
\newpage
%%%%%%%%%%%%%%%%%%%%%%%%%%%%%%%%%%%%%%%%%%%%%%%%%%%%%%%%%%%%%%%%%%%%%%%%%%%%%%%%
%%%%%%%%%%%%%%%%%%%%%%%%%%%%%%%%%%%%%%%%%%%%%%%%%%%%%%%%%%%%%%%%%%%%%%%%%%%%%%%%
\vskip0.1in\hrule\vskip0.1in




%%%%%%%%%%%%%%%%%%%%%%%%%%%%%%%%%%%%%%%%%%%%%%%%%%%%%%%%%%%%%%%%%%%%%%%%%%%%%%%%
%%%%%%%%%%%%%%%%%%%%%%%%%%%%%%%%%%%%%%%%%%%%%%%%%%%%%%%%%%%%%%%%%%%%%%%%%%%%%%%%
\vskip0.1in\hrule\vskip0.1in
\noindent
{\bf Vector/Matrix Operations: The Euclidean Length of a Vector}
\vskip0.1in\hrule\vskip0.1in
\noindent
In many problems it will be necessary to measure the length or magnitude of a
vector (at least so say the villian in Despicable Me). So, we need some way to
determine the length. Actually, there are infinitely many ways to do this. The
approach most students see first involves using the Euclidean distance between
two points used to define a vector. From any standard linear algebra course, the
Euclidean length can be defined as follows.
$$
  || v || = ( v_1^2 + v_2^2 + \cdots + v_n^2 )^{1/2}
$$
where the notation, $||*||$, provides a mathematical notation for the magnitude
of the vector. The notation emphasizes the difference between the absolute 
value of the difference of two real numbers and the length of a vector. This can
be implemented easily into a method or routine that will return this length of a
vector.
\begin{verbatim}

     public double l2norm(double[] v) {
       double sum = 0.0;
       //
       // extract the length of the vector
       // --------------------------------
       //
       int n = v.length;
       //
       // compute the sum of squares
       // --------------------------
       //
       for(int i=0; i<n; i++) sum = sum + v[i] * v[i];
       //
       // return the value
       // ----------------
       //
       return Math.sqrt(sum);
       //
     }

\end{verbatim}
Note that the Java package including the square root function defines the last
step in the code. It is best to use an intrinsic for the square root since
implementing our own version of a square root method would take some time.
\newpage
%%%%%%%%%%%%%%%%%%%%%%%%%%%%%%%%%%%%%%%%%%%%%%%%%%%%%%%%%%%%%%%%%%%%%%%%%%%%%%%%
%%%%%%%%%%%%%%%%%%%%%%%%%%%%%%%%%%%%%%%%%%%%%%%%%%%%%%%%%%%%%%%%%%%%%%%%%%%%%%%%
\vskip0.1in\hrule\vskip0.1in
\noindent
{\bf Vector/Matrix Operations: Definition of the Norm of a Vector}
\vskip0.1in\hrule\vskip0.1in
\noindent
There are many ways to compute the length of a vector. In this section we will
consider several ways to do this. However, we need to have a general definition
of what is meant by the length of a vector. To start, the definition of a norm
is given.
\begin{definition}
   Suppose that $V$ is a vector space and ${\bf u}$ and ${\bf v}$ are any two
   vectors in $V$. Also, assume $a$ is an arbitrary number/scalar. The norm of a
   vector is a function
   $$
     || * || : V \rightarrow \Re
   $$
   such that
   \begin{enumerate}
     \item $||v||=0$ if and only if ${\bf v} = {\bf 0}$,
     \item $||a{\bf v}|| = |a|\ ||{\bf v}||$, and
     \item $||{\bf u} + {\bf v} || \leq ||{\bf u}|| + ||{\bf v}||$.
   \end{enumerate}
\end{definition}
The Euclidean magnitude/length of a vector can be shown to satisfy this
definition of a norm. This means we can use the terms length, magnitude, and
norm interchangeably. The term norm is used in the name, {\bf l2norm}, chosen
for the method in the code above.

There are an infinite number of ways to compute norms on vector spaces. A
general definition for a norm on n-dimensional real space is the following.
$$
  ||v||_p = \left( v_1^p + v_2^p + \cdots + v_n^p \right)^{1/p}
              = \left( \sum_{i=1}^n\ |v_i|^p \right)^{1/p}
$$
for any positive integer $p$. The Euclidean norm is obtained when $p=2$. The
reality in computational mathematics is that there are three norms of interest
including the 2-norm. The other two norms are the 1-norm and the other is the
infinity-norm.  The definitions for these two norms are given below.
$$
  ||v||_1 = | v_1 | + | v_2 | + \cdots + | v_n | = \sum_{i=1}^n\ |v_i|
$$
and
$$
  ||v||_\infty = \max_{1\leq i\leq n} | v_i | 
$$
are the definitions we will use in our work. The gold standard for measuring
length is typically the 2-norm. However, in some problems the 1-norm and
infinity-norm are easier to compute in many problems.

The 1-norm and infinity-norm can easily be coded into methods as was done above
to the 2-norm. Using the 2-norm code, the other two methods just need a couple
of minor changes to implement the other two norms into their own methods.
\newpage
%%%%%%%%%%%%%%%%%%%%%%%%%%%%%%%%%%%%%%%%%%%%%%%%%%%%%%%%%%%%%%%%%%%%%%%%%%%%%%%%
%%%%%%%%%%%%%%%%%%%%%%%%%%%%%%%%%%%%%%%%%%%%%%%%%%%%%%%%%%%%%%%%%%%%%%%%%%%%%%%%
\vskip0.1in\hrule\vskip0.1in
\noindent
{\bf Vector/Matrix Operations: Errors in Vector Approximations}
\vskip0.1in\hrule\vskip0.1in
\noindent
Just as the errors in approximating roots of functions of a single variable,
having a way to compute the magnitude of the error in approximating one vector
${\bf v}$ by another vector ${\bf u}$. We can use the definitions of norms
earlier in this section to define a consistent formula for the vector
approximation error. If we use a generic definition of the norm of a vector, we
can define
$$
  \makebox{absolute error} = || {\bf v} - {\bf u} ||
$$
and
$$
  \makebox{relative error} = {{|| {\bf v} - {\bf u} || }\over{ || {\bf u} || }}
$$
These formulas should look familiar when compared to error measurement in root
finding problems.

A code that will implement the absolute error with the 2-norm might look like
\begin{verbatim}

     public double absl2err(double[] u, double[] v) {
       double sum = 0.0;
       //
       // extract the length of the vector
       // --------------------------------
       //
       int n = v.length;
       double diff = 0.0;
       for(int i=0; i<n; i++) {
         diff = u[i] - v[i];
         sum = sum + diff * diff;
       }
       //
       // return the norm of the difference
       // ---------------------------------
       //
       return Math.sqrt(sum);
       //
     }

\end{verbatim}
Of course, we could reuse code that we have already written as follows. Provided
that the methods have been created and test and inserted in some sort of archive
that is available for our use. A first simpler version is the following.
\begin{verbatim}

     public double absl2err(double[] u, double[] v) {
       double [] diff = null;
       int n = v.length;
       diff = vecsub(u, v);
       return l2norm(diff);
     }

\end{verbatim}
An even more concise version of the method might like the following.
\begin{verbatim}

     public double absl2err(double[] u, double[] v) {
       return l2norm(vecsub(u, v));
     }

\end{verbatim}

\end{document}
