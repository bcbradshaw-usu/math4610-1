\documentclass[10pt,fleqn]{article}
\usepackage{graphicx}


\setlength{\topmargin}{-.75in}
\addtolength{\textheight}{2.00in}
\setlength{\oddsidemargin}{.00in}
\addtolength{\textwidth}{.75in}
\title{Math 4610 Lecture Notes \\
           \ \hfill 
            \ \\
      Root Finding Problems for Real Values Function of One Variable
  \footnote{These notes are part of an Open Resource Educational project
            sponsored by Utah State University}}

\author{Joe Koebbe}

\nofiles

\pagestyle{empty}

\setlength{\parindent}{0in}

% new math commands


\setlength{\oddsidemargin}{-0.25in}
\setlength{\evensidemargin}{-0.25in}
\setlength{\textwidth}{6.75in}
\setlength{\headheight}{0.0in}
\setlength{\topmargin}{-0.25in}
\setlength{\textheight}{9.00in}

\makeindex

\usepackage{mathrsfs}

%\usepackage[pdftex]{graphicx}
\usepackage{epstopdf}

\newcounter{beans}

\newcommand{\ds}{\displaystyle}
\newcommand{\limit}[2]{\displaystyle\lim_{#1\to#2}}

\newcommand{\binomial}[2]{\ \left( \begin{array}{c}
                                  #1 \\
                                  #2
                                 \end{array}
                            \right) \
                         }
\newcommand{\ExampleRule}[2]
  {
  \noindent
  \rule{\linewidth}{1pt}
  \begin{example}
    #1
    \label{#2}
  \end{example}
  \rule{\linewidth}{1pt}
  \vskip0.125in
  }

\newcommand{\defbox}[1]
  {
   \ \\
   \noindent
   \setlength\fboxrule{1pt}
   \fbox{
        \begin{minipage}{6.5in}
          #1
        \end{minipage}
        }
   \ \\
  }
\newcommand{\verysmallworkbox}[1]
  {
   \ \\
   \noindent
   \setlength\fboxrule{1pt}
   \fbox{
        \begin{minipage}{6.5in}
           #1
           \ \\
           \vskip0.5in \ \\
           \ \\
        \end{minipage}
        }
   \ \\
  }
\newcommand{\smallworkbox}[1]
  {
   \ \\
   \noindent
   \setlength\fboxrule{1pt}
   \fbox{
        \begin{minipage}{6.5in}
           #1
           \ \\
           \vskip2.5in \ \\
           \ \\
        \end{minipage}
        }
   \ \\
  }
\newcommand{\halfworkbox}[1]
  {
   \ \\
   \noindent
   \setlength\fboxrule{1pt}
   \fbox{
        \begin{minipage}{6.5in}
           #1 \hfill
           \ \\
           \vskip3.25in \ \\
           \ \\
        \end{minipage}
        }
   \ \\
  }
\newcommand{\largeworkbox}[1]
  {
   \ \\
   \noindent
   \setlength\fboxrule{1pt}
   \fbox{
        \begin{minipage}{6.5in}
           #1
           \ \\
           \vskip7.5in \ \\
           \ \\
        \end{minipage}
        }
   \ \\
  }
\newcommand{\flexworkbox}[2]
  {
   \ \\
   \noindent
   \setlength\fboxrule{1pt}
   \fbox{
        \begin{minipage}{6.5in}
           #1
           \ \\

           \vskip#2 \ \\
           \ \\
        \end{minipage}
        }
   \ \\
  }


% symbols for sets of numbers

\newcommand{\natnumb}{$\cal N$}
\newcommand{\whonumb}{$\cal W$}
\newcommand{\intnumb}{$\cal Z$}
\newcommand{\ratnumb}{$\cal Q$}
\newcommand{\irrnumb}{$\cal I$}
\newcommand{\realnumb}{$\cal R$}
\newcommand{\cmplxnumb}{$\cal C$}

% misc. commands

\newcommand{\mma}{{\it Mathematica}}
\newcommand{\sech}{\mbox{ sech}}
 
\newtheorem{theorem}{Theorem}
\newtheorem{example}{Example}
\newtheorem{definition}{Definition}
\newtheorem{problem}{Problem}

\setcounter{secnumdepth}{2}
\setcounter{tocdepth}{4}


\begin{document}
\maketitle

%\begin{figure}
%    \centering
%    \def\svgwidth{\columnwidth}
%    \input{image.pdf_tex}
%\end{figure}
%
%\begin{verbatim}
%<a href="https://www.mathjax.org">
%    <img title="Powered by MathJax"
%    src="https://www.mathjax.org/badge/badge.gif"
%    border="0" alt="Powered by MathJax" />
%</a>
%\end{verbatim}

\newpage
%%%%%%%%%%%%%%%%%%%%%%%%%%%%%%%%%%%%%%%%%%%%%%%%%%%%%%%%%%%%%%%%%%%%%%%%%%%%%%%%
%%%%%%%%%%%%%%%%%%%%%%%%%%%%%%%%%%%%%%%%%%%%%%%%%%%%%%%%%%%%%%%%%%%%%%%%%%%%%%%%
\vskip0.1in\hrule\vskip0.1in
\noindent
{\bf Root Finding Problem: Definition of a Root Finding Problem} 
\vskip0.1in\hrule\vskip0.1in
\noindent
There are many mathematical problems are cast in terms of finding a point in
some interval, \((a,b)\), where a function, \(f\), is zero. Finding such locations
amounts to the solution of a root-finding problem. For example, in a standard
first semester calculus course, the process of finding extreme values of a
real-valued function, \(g\), is presented. The problem in one variable can be
recast or transformed with some work into the problem of determining locations
where the derivative, \(g'\), is zero. This is true since a necessary condition
for the existence of a local minimum or local maximum value of a differentiable
function at a point \(x^*\) is that the derivative be zero. In this case, the
problem of determining the location of a minimum or maximum value of a function
is rewritten as finding the zeros of the derivative of the function. That is,
find all points, \(x^*\), such that
\[
  g'(x^*)=0.
\]
The result is a root finding problem for the derivative of a function.

The following is a general definition of the root finding problem for a
real-valued function of a single real variable.
\vskip0.1in\hrule\vskip0.1in
\begin{definition}
  {\bf The General Root Finding Problem:} Given a real-valued function, \(f\), of
  a single real variable find a point or points, \(x^*\), in the domain of the
  function such that
  \[
    f(x^*) = 0
  \]
  The value, \(x^*\), is called a root or zero of the function \(f\).
\end{definition}
\vskip0.1in\hrule\vskip0.1in
Solution of the general root finding problem seems like it should be easy.
However, there are many sources of error and difficulties that are hidden within
the definition of the function.

There are all kinds of issues that arise in solving root finding problems. For
example, the function may have multiple roots that are close together. This is
an issue if, for example, the multiplicity of the root you are looking for is
in question. It might be the case that roots located close together may appear
as multiple roots due to roundoff error or machine precision issues. In this
case, it could be difficult to detect the difference in the locations of the
roots. In searching for a specific root, say the largest or smallest, we may
find other roots that are not of interest. To deal with all of the issues in
this problem, we will develop a number of algorithms that can be used to
overcome the problems that arise.

More often than not, we will need to locate roots that cannot be represented
exactly due to finite precision in number representation. For example, finding
the roots of
\[
  sin(x)=0
\]
is easy from an analytic point of view. This is a problem covered in all
trigonometry courses in high school and college. The zeros are \(x_n=n\ \pi\)
where \(n\) is an arbitrary integer. If \(n\) is not equal to zero, the root is an
irrational number and cannot be represented exactly in finite precision. So, we
must be prepared to settle for an approximation of the roots of a function. It
should be noted that an algebraic solution will be available only in cases where
\(f(x)\) has a simple definition, say a linear or quadratic polynomial. Also, we
might be able to guarantee a solution exists, but there may be no analytic means
of finding a root or multiple roots for the given function.

As a simple example of proving the existence of roots, consider the function
\[
  p(x) = 1 + 2\ x + 3\ x^2 + 5\ x^3 + \pi\ x^4 + e^{1}\ x^5
\]
This is a polynomial of degree five. For any polynomial of odd degree, we know
from our algebra background there is at least one real root. Since \(p(x)\) is a
polynomial of degree five, there must be at least one real root. However, based
on the coefficients, it will likely be the case that there is no analytic method
for computing a root for this problem.

One very complicated root finding problem involves one of the oldest unsolved
problems in all of mathematics. The problem is the Riemann conjecture or Riemann
hypothesis regarding the distribution of prime numbers in amongst all real
numbers. The Riemann-Zeta function is
\[
  \zeta(s) = \sum_{n=1}^\infty {1\over{n^s}}
\]
where \(s\) represents an arbitrary complex number. This function is at the heart
of the Riemann-conjecture and the distribution of prime numbers. This innocent
looking formula is still not completely understood and the Riemann conjecture
has elluded all efforts at a solution for more than 100 years. It should be
noted that the distribution of primes is central in the development of data
encryption strategies in cyber-security applications.
\vskip0.1in\hrule\vskip0.1in
\newpage
%%%%%%%%%%%%%%%%%%%%%%%%%%%%%%%%%%%%%%%%%%%%%%%%%%%%%%%%%%%%%%%%%%%%%%%%%%%%%%%%
%%%%%%%%%%%%%%%%%%%%%%%%%%%%%%%%%%%%%%%%%%%%%%%%%%%%%%%%%%%%%%%%%%%%%%%%%%%%%%%%
\vskip0.1in\hrule\vskip0.1in
\noindent
{\bf Root Finding Problems: Using Fixed Point Iteration} 
\vskip0.1in\hrule\vskip0.1in
\noindent
As a first attempt at determining the location of a root for a function, we
might consider a modification of the root finding problem as follows. Given a
function, \(f\), we can rewrite the root finding equation
\[
  f(x) = 0
\]
as
\[
  x = x - f(x) = g(x)
\]
The resulting equation is called a fixed point equation or fixed point problem
\[
  x = g(x)
\]
We will use the fixed point problem to define an algorithm for locating roots of
a function. So, suppose we have an initial guess at the solution of the fixed
point equation, \(x_0\), that may or may not satisfy the equation. We can
substitute the value into the fixed point function to obtain
\[
  x_1 = g(x_0)
\]
If \(x_0=x^*\) then the output will be the same as the input, \(x_1=x^*\). If not,
the value can be used as another approximation of \(x^*\). We can repeat this
process ad infinitum. A general formula for the iteration starts by providing an
initial guess, \(x_0\), and then compute
\[
  x_{k+1} = g(x_k)
\]
for \(k=0,1,2,\ldots\)\ This iteration will produce an infinite sequence
\[
  \{ x_k \}_{k=0}^\infty = \{ x_0, x_1, x_2, \cdots \}
\]
of approximations to the solution of the fixed point problem. Since the fixed
point problem is equivalent to the root finding problem, we can treat the
sequence as approximations of the root finding problem.

Even though we can generate any number of approximations of the solution of the
fixed point problem in this way, there is no guarantee that any of these
approximations are close to the solution we desire. If a tolerance is specified
apriori there is no guarantee that the sequence will be close to anything. In
mathmatical terms, what we want is
\[
  \lim_{k\rightarrow\infty} x_k = x^*
\]
That is, we would really like the sequence to converge to a root. We will
return to this topic after writing a bit of code and presenting an example.
\vskip0.1in\hrule\vskip0.1in
\newpage
%%%%%%%%%%%%%%%%%%%%%%%%%%%%%%%%%%%%%%%%%%%%%%%%%%%%%%%%%%%%%%%%%%%%%%%%%%%%%%%%
%%%%%%%%%%%%%%%%%%%%%%%%%%%%%%%%%%%%%%%%%%%%%%%%%%%%%%%%%%%%%%%%%%%%%%%%%%%%%%%%
\vskip0.1in\hrule\vskip0.1in
\noindent
{\bf Root Finding Problems: Coding Fixed Point Iteration} 
\vskip0.1in\hrule\vskip0.1in
\noindent
One can easily write a routine or computer code that implements fixed point
iteration. The following code provides a template of how a reusable routine
might be written:
\vskip0.1in\hrule\vskip0.1in
\begin{verbatim}
     //
     // Author: Joe Koebbe
     //
     // Routine Name:         fproot
     // Programming Language: Java
     // Last Modified:        09/10/19
     //
     // Description/Purpose: The routine will generate a sequence of numbers
     // using fixed point iteration.
     //
     // Input:
     //
     // FunctionObject f - the function defined in the root finding problem
     // double x0 - the initial guess at the location of a fixed point
     // double tol - the error tolerance allowed in the approximation of the
     //              root finding problem
     // int maxit - the maximum number of iterations allowed in the fixed point
     //             iteration.
     //
     // Output:
     //
     // double x1 - the last number in the finite sequence that is an
     //             approximation in the root finding problem
     //
     public double fproot(FunctionObject f, double x0, double tol, int maxit) {
       //
       // initialize the error in the routine so that the iteration loop will be
       // executed at least one time
       // --------------------------
       //
       double error = 10.0 * tol;
       //
       // initialize a counter for the number of iterations
       // -------------------------------------------------
       //
       int iter = 0;
       //
       // loop over the fixed point iterations as long as the error is larger
       // than the tolerance and the number of iterations is less than the
       // maximum number allowed
       // ----------------------
       //
       while(error > tol && iter < maxit) {
         //
         // update the number of iterations performed
         // -----------------------------------------
         //
         iter++;
         //
         // compute the next approximation
         // ------------------------------
         //
         double x1 = x0 - f(x0);
         //
         // compute the error using the difference between the iterates in the
         // loop
         // ----
         //
         error = Math.abs(x1 - x0);
         //
         // reset the input value to be the new approximation
         // --------------------------------------------------
         //
         x0 = x1;
         //
       }
       //
       // return the last value computed
       // -------------------------------
       //
       return x1;
       //
     }

\end{verbatim}
\vskip0.1in\hrule\vskip0.1in
\noindent
There are a couple of features in the code that need to be explained.
\begin{list}{$\bullet$}{\usecounter{beans} \parsep=0pt \listparindent=0pt
\topsep=0pt \rightmargin=.35in \leftmargin=.35in \labelsep=5 pt
\itemsep=2pt}
  \item To make this work in the Java programming language, the method would
        need to be embedded in a class. That is, the code is not a standalone
        code.
  \item The first argument is an Java Object that needs to be created. The
        object is used to provide the function evaluation for any real input.
  \item The second argument is the initial guess at the solution of the problem.
  \item Since we know we are going to end up with at best an approximation of
        a root, the third argument in the function is an error tolerance that is
        acceptable to the calling routine.
  \item The final argument passed in limits the number of iterations allowed in
        the method. Note that if you are not careful, an infinite loop might be
        created due to the approximations used everywhere.
\end{list}
If we apply the code to any problem, we are assuming that the solution will pop
out the end. There is no guarantee that this is the case. It is important to
establish conditions that will guarantee the code will produce an approximate
solution of the fixed point problem and thus provide a root for the original
function, \(f\).
\vskip0.1in\hrule\vskip0.1in
\newpage
%%%%%%%%%%%%%%%%%%%%%%%%%%%%%%%%%%%%%%%%%%%%%%%%%%%%%%%%%%%%%%%%%%%%%%%%%%%%%%%%
%%%%%%%%%%%%%%%%%%%%%%%%%%%%%%%%%%%%%%%%%%%%%%%%%%%%%%%%%%%%%%%%%%%%%%%%%%%%%%%%
\vskip0.1in\hrule\vskip0.1in
\noindent
{\bf Root Finding Problems: Analysis of Functional Iteration Using Taylor
 Series Expansion} 
\vskip0.1in\hrule\vskip0.1in
\noindent
The general iteration formula, given \(x_0\), is the following.
\[
  x_{k+1} = g(x_k)
\]
for \(k=0,1,2,\ldots\). We also know that for the fixed point problem, the
solution satisfies the equation
\[
  x^* = g(x^*)
\]
Subtracting the two equations gives
\[
  x_{k+1} - x^* = g(x_k) - g(x^*)
\]
The Taylor expansion of \(g(x_k)\) about the solution \(x^*\) is given by
\[
  g(x_k) = g(x*) + g'(x^*) ( x_k - x^* ) + {1\over 2} g''(x^*) ( x_k - x^* )^2
              + \ldots
\]
Substituting the expansion into the equation above and truncating the series
gives
\[
  x_{k+1} - x^* \approx g(x*) + g'(x^*) ( x_k - x^* ) - g(x^*)
                       = g'(x^*) ( x_k - x^* )
\]
Taking absolute values the last equation can be written as
\[
  | x_{k+1} - x^* | \leq | g'(x^*) | \cdot | x_k - x^* |
\]
One can read the previous expression as the difference (or error) in \(x_{k+1}\)
is less than the magnitude of the derivative of the fixed point iteration 
function, \(g\), times the difference (or error) in the previous approximation,
\(x_k\). Using
\[
  e_{k} = | x_k - x^* |
\]
allows use to relate the error at successive steps as
\[
  e_{k+1} \leq | g'(x^*) | \cdot | e_{k+1} |
\]
To get convergence to the fixed point (or root) we would like the error to be
reduced at each step. This requires the condition
\[
  | g'(x^*) | < 1
\]
For the general fixed point problem, this condition is required for convergence
to the fixed point, \(x^*\), or solution of the root finding problem. Note that
this is a significant drawback of fixed point iteration as a means of solving
root finding problems.
\vskip0.1in\hrule\vskip0.1in
\newpage
%%%%%%%%%%%%%%%%%%%%%%%%%%%%%%%%%%%%%%%%%%%%%%%%%%%%%%%%%%%%%%%%%%%%%%%%%%%%%%%%
%%%%%%%%%%%%%%%%%%%%%%%%%%%%%%%%%%%%%%%%%%%%%%%%%%%%%%%%%%%%%%%%%%%%%%%%%%%%%%%%
\vskip0.1in\hrule\vskip0.1in
\noindent
{\bf Root Finding Problems: An Example Using Functional Iteration} 
\vskip0.1in\hrule\vskip0.1in
\noindent
Suppose that we are interested in computing the roots of
\[
  f(x) = e^x - \pi
\]
Analytically we can compute the solution by solving for \(x\) in the equation
\[
  e^x - \pi = 0
\]
The value is \(x=ln(\pi)\approx 1.144729886\). This is a very simple problem.
However, it is always a good idea to test general methods on simple problems
while developing algorithms and coding these up for use on real problems.

Let's apply functional iteration to this root finding problem. First, we will
need to create an associated function that defines a fixed point problem. One
possibility is to choose
\[
  g_1(x) = x - f(x) = x - ( e^x - \pi ) = x - e^x + \pi
\]
Let's check the condition for convergence by computing the derivative of \(g\)
near at the solution above.
\[
  g_1'(x) = 1 - e^x = 1 - \pi \approx -2.14159245 \rightarrow 
  | g_1'(x) | \approx 2.14159245
\]
The value is bigger than 1 which means the sequence of iterates is not going to
converge. So, the choice of \(g(x)\) will not work.

As another option, consider a modification of the function. If
\[
  f(x) = e^x - \pi = 0
\]
then
\[
  f(x) = {1\over 5} ( e^x - \pi ) = 0
\]
which allows us to write
\[
  g_2(x) = x - {1\over 5} ( e^x - \pi )
\]
with derivative
\[
  g_2'(x) = 1 - {1\over 5} e^x 
\]
and near the solution
\[
  |g_2'(x)| = | 1 - {1\over 5} \pi | < 1.0
\]
So, we can expect better results in this case.
\vskip0.1in\hrule\vskip0.1in
\newpage
%%%%%%%%%%%%%%%%%%%%%%%%%%%%%%%%%%%%%%%%%%%%%%%%%%%%%%%%%%%%%%%%%%%%%%%%%%%%%%%%
%%%%%%%%%%%%%%%%%%%%%%%%%%%%%%%%%%%%%%%%%%%%%%%%%%%%%%%%%%%%%%%%%%%%%%%%%%%%%%%%
\vskip0.1in\hrule\vskip0.1in
\noindent
{\bf Root Finding Problems: Example Results Tabulated} 
\vskip0.1in\hrule\vskip0.1in
\noindent
For the two examples, the output for the two choices of the iteration function
\(g_1(x)\) or \(g_2(x)\).
\vskip0.1in\hrule\vskip0.1in
\begin{table}[h]
\caption{Results for Functional Iteration for Two Different Iteration Functions}
  \vskip0.1in
  \begin{center}
  \begin{tabular}{|c||c|c||c|c|}
    \hline
    Iteration No. & \(g_2(x)=x-(e^x-\pi)\) & error
                              & \(g_1(x)=x-(e^x-\pi)\) & error \\
    \hline
        01 &  1.08466220  &  8.46621990E-02  & 1.42331100  & 0.423310995 \\
    \hline
        02 &  1.12129271  &  3.66305113E-02  & 0.41406250  & 1.00924850 \\
    \hline
        03 &  1.13584745  &  1.45547390E-02  & 2.04270363  & 1.62864113 \\
    \hline
        04 &  1.14140379  &  5.55634499E-03  &-2.52713394  & 4.56983757 \\
    \hline
        05 &  1.14349020  &  2.08640099E-03  & 0.53457117  & 3.06170511 \\
    \hline
        06 &  1.14426863  &  7.78436661E-04  & 1.96944773  & 1.43487656 \\
    \hline
        07 &  1.14455843  &  2.89797783E-04  &-2.05567694  & 4.02512455 \\
    \hline
        08 &  1.14466619  &  1.07765198E-04  & 0.95790958  & 3.01358652 \\
    \hline
        09 &  1.14470625  &  4.00543213E-05  & 1.49325967  & 0.535350084 \\
    \hline
        10 &  1.14472115  &  1.49011612E-05  & 0.18326997  & 1.30998969 \\
    \hline
        11 &  1.14472663  &  5.48362732E-06  & 2.12372398  & 1.94045401 \\
    \hline
  \end{tabular}
  \end{center}
\end{table}
\vskip0.1in\hrule\vskip0.1in
So, two completely different results are obtained. One converges with a slight
modification to the first. The first function produces a sequence that does not
converge and the second produces the correct result up to machine precision.
That is, \(x^*=1.14472663\) with absolute error \(5.48362732E-06\). This is one of
the reasons why functional iteration is not used as much. The problem is that
there are infinitely many choices for the fixed point equation. Some will
provide convergence and others will not come close.
\vskip0.1in\hrule\vskip0.1in
\newpage
%%%%%%%%%%%%%%%%%%%%%%%%%%%%%%%%%%%%%%%%%%%%%%%%%%%%%%%%%%%%%%%%%%%%%%%%%%%%%%%%
%%%%%%%%%%%%%%%%%%%%%%%%%%%%%%%%%%%%%%%%%%%%%%%%%%%%%%%%%%%%%%%%%%%%%%%%%%%%%%%%
\vskip0.1in\hrule\vskip0.1in
\noindent
{\bf Root Finding Problems: Convergence of Functional Iteration}
\vskip0.1in\hrule\vskip0.1in
\noindent
If we end up using functional iteration, it will also pay to know how fast the
sequence converges. Fewer iterations means faster results with few computations.
The convergence of the sequence is determined by the same calculations as in
the convergence justification above.
\[
  | x_{k+1} - x^* | \leq | g'(x^*) | \cdot| x_k - x^* |
\]
For functional iteration the convergence rate is defined by
\[
  \makebox{rate of convergence} = | g'(x^*) | 
\]
The smaller the magnitude of the derivative, \(|g'(x^*)|\), the faster the
convergence will be.

As an example, consider changing the parameter \({1\over 5}\)used to modify the
iteration function, 
\[
  g_2(x) = x - ( e^x - \pi ) \rightarrow\ g_2(x) = x - {1\over 5} ( e^x - \pi )
\]
to keep the original root the same in the previous section. If the parameter is
decreased, the rate of convergence should be faster. This is covered in one of
the homework tasks.
\vskip0.1in\hrule\vskip0.1in
\newpage
%%%%%%%%%%%%%%%%%%%%%%%%%%%%%%%%%%%%%%%%%%%%%%%%%%%%%%%%%%%%%%%%%%%%%%%%%%%%%%%%
%%%%%%%%%%%%%%%%%%%%%%%%%%%%%%%%%%%%%%%%%%%%%%%%%%%%%%%%%%%%%%%%%%%%%%%%%%%%%%%%
\vskip0.1in\hrule\vskip0.1in
\noindent
{\bf Root Finding Problems: Continuous Functions and the Bisection Method}
\vskip0.1in\hrule\vskip0.1in
\noindent
On the positive side of things, the fixed point approach in the previous section
requires very little of the function in the root finding problem. The only
requirement is that \(f\) is a function at every input value. It is usually very
easy to implement fixed point iteration for this type of problem. It may be
difficult if not impossible to come up with a fixed point problem that will
provide convergence to any fixed point. Due to slow convergence and issues
finding a fixed point equation that works, functional iteration is limited in
applicability in the real world.

So, we need to develop alternative algorithms for the root finding problem. In
this section, we will assume that the function, \(f\), is continuous on a closed
and bounded interval \([a,b]\) where we expect to find a root. The main
mathematical tool used in this case is the Intermediate Value Theorem for
continuous functions.
\vskip0.1in\hrule\vskip0.1in
{\bf Theorem:} Suppose the function, \(f\), is continuous on the closed and
boundaed interval \([a, b]\). If \(M\) is any value between \(f(a)\) and \(f(b)\) then
there exists a value \(c\in(a,b)\) such that \(f(c)=M\),
\vskip0.1in\hrule\vskip0.1in
\noindent
Now, if \(f(a)\geq 0\geq f(b)\) (or vice-versa) then there is at least one value,
\(c\) in the interval \((a,b)\) such that \(f(c)=0\). If we determine end-points of
an interval such that \(f(a)<0\) {\bf and} \(0<f(b)\) (or vice versa), we know there
is also a root of the function somewhere in the interval we have selected. There
is a simple condition that can be test to verify an interval contains an
interval. That is,
\[
   f(a)\cdot f(b) < 0
\]
This is enough to determine that the function crosses the horizontal axis at
at least one point in the interval.
\vskip0.1in\hrule\vskip0.1in
\newpage
%%%%%%%%%%%%%%%%%%%%%%%%%%%%%%%%%%%%%%%%%%%%%%%%%%%%%%%%%%%%%%%%%%%%%%%%%%%%%%%%
%%%%%%%%%%%%%%%%%%%%%%%%%%%%%%%%%%%%%%%%%%%%%%%%%%%%%%%%%%%%%%%%%%%%%%%%%%%%%%%%
\vskip0.1in\hrule\vskip0.1in
\noindent
{\bf Root Finding Problems: Bisection and Convergence}
\vskip0.1in\hrule\vskip0.1in
\noindent
Once we have determined an interval \([a,b]\) such that \(f(a)\ f(b)<0\) we can
start work to determine the location of a root in the initial interval. We
proceed by bisecting the original interval \([a, b]\) into two equal subintervals
\[
  [a, b] = [a, c] \cup [c, b]
\]
where
\[
  c = {{a+b}\over 2}
\]
Since there is at least one root on \([a, b]\) there are three possibilities that
can occur in the bisection. These are:
\begin{list}{$\bullet$}{\usecounter{beans} \parsep=0pt \listparindent=0pt
\topsep=0pt \rightmargin=.35in \leftmargin=.35in \labelsep=5 pt
\itemsep=2pt}
  \item \(f(c) = 0\),
  \item \(f(a)\cdot f(c)<0\) which implies there is a root in \([a,c]\), or
  \item \(f(c)\cdot f(b)<0\) which implies there is a root in \([c,b]\).
\end{list}
If the first condition is true, we have the root, \(x^*=c\) and we are done
searching. In the second case, we can redefine the search interval to \([a,c]\)
and in the third case, the search interval will be redefined to be \([c,b]\). Once
we have redefined the search interval, we repeat the bisection on this new
search interval. The bisection will reduce the size of the search interval by
a factor of two. We just need to translate this idea into a computer code in
some language.
\vskip0.1in\hrule\vskip0.1in
\newpage
%%%%%%%%%%%%%%%%%%%%%%%%%%%%%%%%%%%%%%%%%%%%%%%%%%%%%%%%%%%%%%%%%%%%%%%%%%%%%%%%
%%%%%%%%%%%%%%%%%%%%%%%%%%%%%%%%%%%%%%%%%%%%%%%%%%%%%%%%%%%%%%%%%%%%%%%%%%%%%%%%
\vskip0.1in\hrule\vskip0.1in
\noindent
{\bf Root Finding Problems: A (First) Simple Bisection Code in C}
\vskip0.1in\hrule\vskip0.1in
\noindent
The following routine, written in something like C implements the Bisection
Method.
\vskip0.1in\hrule\vskip0.1in
\begin{verbatim}

     double bisectionMethod(typedef'd f, double a, double b, double tol,
                            int maxiter)
     {
       //
       // set up some parameters and local variables to do the work
       // ---------------------------------------------------------
       //
       double c;
       double error;
       int iter;
       //
       // check the endpoints - if either is 0, we already have a root
       // ------------------------------------------------------------
       //
       if(f(a)==0) return a;
       if(f(b)==0) return b;
       //
       // check for a root in the interval
       // --------------------------------
       //
       if(f(a)*f(b) >= 0.0) throw an error or print a message
       //
       // set the error and iteration counter
       // -----------------------------------
       //
       error = 10.0 * tol;
       iter = 0;
       //
       // use a while loop to go until the tolerance is met or the maximum 
       // number of iterations has been exceeded
       // --------------------------------------
       //
       while(error > tol && iter < maxiter) {
         //
         // update the iteration counter and compute the midpoint of the current
         // interval
         // --------
         //
         iter++;
         c = 0.5 * ( a + b );
         //
         // compute the sign change value
         // -----------------------------
         //
         double val = f(a) * f(c);
         //
         // reassign the end point based on the location of the root
         // --------------------------------------------------------
         //
         if(val<0.0) {
           b = c;
         } else {
           a = c;
         }
         //
         // compute the error in the approximation - this assumes a<b
         // ---------------------------------------------------------
         //
         error = b - a
         //
       }
       //
       // return the midpoint as it is more accurate
       // ------------------------------------------
       //
       return c;
       //
     }

\end{verbatim}
\vskip0.1in\hrule\vskip0.1in
\noindent
The first argument in the C method needs to be changed to a pointer to a
function as an input to the method. This is left up to the reader to do.
It should be noted that once an interval has been determined on which the
function value changes sign, the Bisection Method will continue until a root is
found, at least up to machine precision. We can take advantage of this property
to redesign the algorithm to take a specific number of iterations instead of
checking the error.
\vskip0.1in\hrule\vskip0.1in
\newpage
%%%%%%%%%%%%%%%%%%%%%%%%%%%%%%%%%%%%%%%%%%%%%%%%%%%%%%%%%%%%%%%%%%%%%%%%%%%%%%%%
%%%%%%%%%%%%%%%%%%%%%%%%%%%%%%%%%%%%%%%%%%%%%%%%%%%%%%%%%%%%%%%%%%%%%%%%%%%%%%%%
\vskip0.1in\hrule\vskip0.1in
\noindent
{\bf Root Finding Problems: The Bisection Method and Error Reduction}
\vskip0.1in\hrule\vskip0.1in
\noindent
The fact the the interval size is being reduced in each iteration of bisection
can be used as follows. The length of the original interval can be computed and
used to bound the error in any approximation of a root. That is,
\[
  | x - x^* | \leq | b - a |
\]
A sequence of intervals is created by the Bisection method that contains a root.
We can use subscripts to define the intervals as the bisection proceeds. If we
use \([a_i, b_i]\), for \(i=0,1,\ldots\) where each new interval is selected after
the previous interval is bisected. Note that if we are assuming
\([a_0, b_0]=[a,b]\) in this argument. So, we can write the following set of
inequalities
\[
  | x - x^* | < b_k - a_k
              < {1\over 2} ( b_{k-1} - a_{k-1} )
              < \cdots
              < {1\over{2^k}} ( b_0 - a_0 ) = 2^{-k} ( b - a )  
\] 
This means that once the interval \([a, b]\) has been determined, the reduction in
the error between iterations is computable.

Suppose that we specify an error tolerance that is acceptable, say \(10^{-d}\)
where \(d\) is the number of digits of accuracy. Then we can define the number of
iterations to reduce the error to the desired tolerance as follows.
\[
  2^{-k} ( b - a ) < 10^{-d}
\]
Using a bit of algebra
\[
  2^{-k} < {{10^{-d}}\over{(b-a)}}
   \rightarrow -k < log_2\left( {{10^{-d}}\over{(b-a)}}\right)
\]
or flipping the inequality using a negative multiplier
\[
  - log_2\left( {{10^{-d}}\over{(b-a)}}\right) < k 
\]
This gives us the total number of iterations needed to reduce the error to the
desired tolerance. So, we can rewrite the code to take advantage of this
calculation.
\vskip0.1in\hrule\vskip0.1in
\newpage
%%%%%%%%%%%%%%%%%%%%%%%%%%%%%%%%%%%%%%%%%%%%%%%%%%%%%%%%%%%%%%%%%%%%%%%%%%%%%%%%
%%%%%%%%%%%%%%%%%%%%%%%%%%%%%%%%%%%%%%%%%%%%%%%%%%%%%%%%%%%%%%%%%%%%%%%%%%%%%%%%
\vskip0.1in\hrule\vskip0.1in
\noindent
{\bf Root Finding Problems: An Alternative Bisection Method Code}
\vskip0.1in\hrule\vskip0.1in
\noindent
The alternative C code to implement the alternate Bisection method where the
number of iterations is computed ahead of time is the following.
%%%%%%%%%%%%%%%%%%%%%%%%%%%%%%%%%%%%%%%%%%%%%%%%%%%%%%%%%%%%%%%%%%%%%%%%%%%%%%%%
%%%%%%%%%%%%%%%%%%%%%%%%%%%%%%%%%%%%%%%%%%%%%%%%%%%%%%%%%%%%%%%%%%%%%%%%%%%%%%%%
\vskip0.1in\hrule\vskip0.1in
\begin{verbatim}

     double bisectionMethod(typedef'd f, double a, double b, double tol) {
       //
       // set up some parameters and local variables to do the work
       // ---------------------------------------------------------
       //
       double c;
       double error;
       //
       // check the endpoints - if either is 0, we already have a root
       // ------------------------------------------------------------
       //
       if(f(a)==0) return a;
       if(f(b)==0) return b;
       //
       // check for a root in the interval
       // --------------------------------
       //
       if(f(a)*f(b) >= 0.0) throw an error or print a message
       //
       // compute the number iterations needed to meet the tolerance given
       // ----------------------------------------------------------------
       //
       maxiter = - 2.0 * log2( tol / ( b - a ) );
       //
       // compute the iterations
       // ----------------------
       for(int i=0; i<maxiter; i++) {
         //
         // compute the midpoint of the current interval
         // --------------------------------------------
         //
         c = 0.5 * ( a + b );
         //
         // compute the sign change value
         // -----------------------------
         //
         double val = f(a) * f(c);
         //
         // reassign the end point based on the location of the root
         // --------------------------------------------------------
         //
         if(val<0.0) {
           b = c;
         } else {
           a = c;
         }
         //
       }
       //
       // return the midpoint as it is more accurate
       // ------------------------------------------
       //
       return c;
       //
     }

\end{verbatim}
\vskip0.1in\hrule\vskip0.1in
\noindent
Note that the output value will be an approximation of a root in the original
interval, \([a,b]\) that satisfies the desired tolerance.
\vskip0.1in\hrule\vskip0.1in
\newpage
%%%%%%%%%%%%%%%%%%%%%%%%%%%%%%%%%%%%%%%%%%%%%%%%%%%%%%%%%%%%%%%%%%%%%%%%%%%%%%%%
%%%%%%%%%%%%%%%%%%%%%%%%%%%%%%%%%%%%%%%%%%%%%%%%%%%%%%%%%%%%%%%%%%%%%%%%%%%%%%%%
\vskip0.1in\hrule\vskip0.1in
\noindent
{\bf Root Finding Problems: Bisection Method Examples}
\vskip0.1in\hrule\vskip0.1in
\noindent
It is always a good idea to test the code you write. Using the example from our
tests of functional iteration we can detemine whether or not the Bisection 
Method works and how this compares with the fixed point approach. So, we will
consider the example in the section on functional iteration. That way, we can
compare the results using Bisection to our previous work.

So, we will work with the easy example,
\[
  f(x) = e^x - \pi = 0
\]
and apply the Bisection method on the interval \([-2.2, 6.8]\). Note that we do
not need to come up with an alternate definition of the problem as in the case
of functional iteration. The results shown include functional iteration and the
Bisection method and are computed towards a tolerance of \(10^{-7}\). 
\vskip0.1in\hrule\vskip0.1in
\begin{table}[h]
\caption{Results for Functional Iteration Compared to Bisection}
  \vskip0.1in
  \begin{center}
  \begin{tabular}{|c||c|c||c|c|}
    \hline
    Iteration No. & Bisection & Bisection error & Functional Iteration
                                                & Functional Iteration error \\
    \hline
      01 & 2.30000019  & 1.15527022 & 1.08466220 & 8.46621990E-02  \\
    \hline
      02 & 5.00000715E-02 & 1.09472990 & 1.12129271 & 3.66305113E-02  \\
    \hline
      03 & 1.17500019  &  3.02702188E-02 &  1.1358474 & 1.45547390E-02 \\
    \hline
      04 & 0.612500131 & 0.532229841 & 1.14140379 & 5.55634499E-03  \\
    \hline
      05 & 0.893750191 & 0.250979781 & 1.14349020 & 2.08640099E-03 \\
    \hline
      06 & 1.03437519  & 0.110354781 & 1.14426863 & 7.78436661E-04 \\
    \hline
      07  & 1.10468769  & 4.00422812E-02  &  1.14455843 & 2.89797783E-04 \\
    \hline
      08 & 1.13984394 & 4.88603115E-03  &  1.14466619 & 1.07765198E-04  \\
    \hline
      09 & 1.15742207 & 1.26920938E-02 &  1.14470625 & 4.00543213E-05  \\
    \hline
      10   & 1.14863300  & 3.90303135E-03  &  1.14472115 & 1.49011612E-05  \\
    \hline
      11 & 1.14423847 & 4.91499901E-04 &  1.14472663 & 5.48362732E-06  \\
    \hline
      12 & 1.14643574 & 1.70576572E-03 & No Data & No Data \\
    \hline
      13 & 1.14533710 & 6.07132912E-04 & No Data & No Data \\
    \hline
      14 & 1.14478779 & 5.78165054E-05 & No Data & No Data \\
    \hline
      15 & 1.14451313 & 2.16841698E-04 & No Data & No Data \\
    \hline
      16 & 1.14465046 & 7.95125961E-05 & No Data & No Data \\
    \hline
      17 & 1.14471912 & 1.08480453E-05 & No Data & No Data \\
    \hline
      18 & 1.14475346 & 2.34842300E-05 & No Data & No Data \\
    \hline
      19 & 1.14473629 & 6.31809235E-06 & No Data & No Data \\
    \hline
      20 & 1.14472771 & 2.26497650E-06 & No Data & No Data \\
    \hline
      21 & 1.14473200 & 2.02655792E-06 & No Data & No Data \\
    \hline
      22 & 1.14472985 & 1.19209290E-07 & No Data & No Data \\
    \hline
      23 & 1.14473093 & 9.53674316E-07 & No Data & No Data \\
    \hline
      24 & 1.14473033 & 3.57627869E-07 & No Data & No Data \\
    \hline
  \end{tabular}
  \end{center}
\end{table}
\vskip0.1in\hrule\vskip0.1in
\noindent
The results actually show that the functional iteration approach actually
converges faster and is more efficient. However, as mentioned earlier, the 
functional iteration approach requires the definition of an alternative problem.
Bisection works as long as the function in question is continuous on a closed 
and bounded interval. The guarantee is that once a root is bracketed the method
will trdge along until an approximate value for the root is determined up to a
a given tolerance.
\vskip0.1in\hrule\vskip0.1in
\newpage
%%%%%%%%%%%%%%%%%%%%%%%%%%%%%%%%%%%%%%%%%%%%%%%%%%%%%%%%%%%%%%%%%%%%%%%%%%%%%%%%
%%%%%%%%%%%%%%%%%%%%%%%%%%%%%%%%%%%%%%%%%%%%%%%%%%%%%%%%%%%%%%%%%%%%%%%%%%%%%%%%
\vskip0.1in\hrule\vskip0.1in
\noindent
{\bf Root Finding Problems: Differentiable Functions}
\vskip0.1in\hrule\vskip0.1in
\noindent
The next method that we can cover is Newton's method. This method is based on
using some simple calculus manipulations to determine an iterative method for
approximating roots of a nonlinear function. So, consider a function \(f(x)\) that
is twice differentiable in some open interval containing a root of the function.
There are a couple of ways to develop Newton's method. For this set of notes,
suppose that, \(x_0\), is provided as an approximation of the root. We can expand
the function using the unknown root, \(x^*\), and the approximation, \(x_0\). That
is, using \(x_0\) as the center of the expansion, 
\[
  f(x^*) = f(x_0) + f'(x_0)\ \left( x^* - x_0 \right)
                  + {1\over 2}\ f'(x_0)\ \left( x^* - x_0 \right)^2 + \cdots
\]
The expansion can be truncated using Taylor's theorem with remainder to write
\[
  f(x^*) = f(x_0) + f'(x_0)\ \left( x^* - x_0 \right)
                  + {1\over 2}\ f'(\xi)\ \left( x^* - x_0 \right)^2
\]
where \(\xi\) is between \(x^*\) and \(x_0\). This expansion works as long as the
function, \(f(x)\), is twice continuously differentiable. Mathematically, the
differentiability condition implies that near \(x_0\) (and \(x^*\)) the remainder
term can be bounded as follows.
\[
  {1\over 2}\ f'(\xi)\ \left( x^* - x_0 \right)^2 \leq
                                         C \left( x^* - x_0 \right)^2
\]
If \(x^*\) and \(x_0\) are sufficiently close, we can neglect this term and write
the approximation
\[
  f(x^*) \approx f(x_0) + f'(x_0)\ \left( x^* - x_0 \right)
\]
Recall that \(f(x^*)=0\). So, we can write
\[
  0 \approx f(x_0) + f'(x_0)\ \left( x - x_0 \right)
\]
for any \(x\) near \(x^*\). Using this, we can define another approximation using
\[
  x_1 = x_0 - {{f(x_0)}\over{f'(x_0)}}
\]
This formula suggests an iteration. Given the output, \(x_1\), given the input
\(x_0\), we can continue this process and compute another approximation, \(x_2\),
using
\[
  x_2 = x_1 - {{f(x_1)}\over{f'(x_1)}}
\]
This leads to the definition of Newton's method.
\vskip0.1in\hrule\vskip0.1in
\newpage
%%%%%%%%%%%%%%%%%%%%%%%%%%%%%%%%%%%%%%%%%%%%%%%%%%%%%%%%%%%%%%%%%%%%%%%%%%%%%%%%
%%%%%%%%%%%%%%%%%%%%%%%%%%%%%%%%%%%%%%%%%%%%%%%%%%%%%%%%%%%%%%%%%%%%%%%%%%%%%%%%
\vskip0.1in\hrule\vskip0.1in
\noindent
{\bf Root Finding Problems: Definition of Newton's Method}
\vskip0.1in\hrule\vskip0.1in
\noindent
Given the work in the previous section of these notes, we can define Newton's
method as follows. Given an initial guess, \(x_0\), the sequence of points,
\(x_k\), given by
\[
  x_k = x_{k-1} - {{f(x_{k-1})}\over{f'(x_{k-1})}}
\]
for \(k=1,2,\ldots\) defines Newton's method for finding the roots of a function
of a single variable.

There are certain restrictions that must be met when using Newton's method.
\vskip0.1in\hrule\vskip0.1in
\begin{list}{$\bullet$}{\usecounter{beans} \parsep=0pt \listparindent=0pt
\topsep=0pt \rightmargin=.35in \leftmargin=.35in \labelsep=5 pt
\itemsep=2pt}
  \item The function must be twice continusously differentiable,
  \item the derivative of the function cannot be zero at the root, \(x^*\), and
  \item the initial point must be chosen sufficiently close to the exact value
        of the root.
\end{list}
\vskip0.1in\hrule\vskip0.1in
\newpage
%%%%%%%%%%%%%%%%%%%%%%%%%%%%%%%%%%%%%%%%%%%%%%%%%%%%%%%%%%%%%%%%%%%%%%%%%%%%%%%%
%%%%%%%%%%%%%%%%%%%%%%%%%%%%%%%%%%%%%%%%%%%%%%%%%%%%%%%%%%%%%%%%%%%%%%%%%%%%%%%%
\vskip0.1in\hrule\vskip0.1in
\noindent
{\bf Root Finding Problems: Newton's Method Example and Comparison}
\vskip0.1in\hrule\vskip0.1in
\noindent
As in the development of fixed point iteration and the Bisection method, we will
need to test out any code that we might write for implementing Newton's method.
We will use the same example,
\[
  f(x) = e^x - \pi = 0
\]
to show how this compares to our previous methods we have tried. The simple
problem specified above will not truely test a general Newton code. However, it
makes for a quick comparison of the methods.
\vskip0.1in\hrule\vskip0.1in
\begin{table}[h]
\caption{Results for Newton's Method Iteration Compared to Bisection}
  \vskip0.1in
  \begin{center}
  \begin{tabular}{|c||c|c||c|c|}
    \hline
    Iteration No. & Newton Method & Newton Method Error & Bisection Method
                                                & Bisection Method \\
    \hline
      01 & 0.0000000000000000 & 1.1447299136769349 & 2.30000019 & 1.15527022 \\
    \hline
      02 & 2.1415927410125732 & 0.99686282733563836 & 5.00000715E-02
              & 1.09472990 \\
    \hline
      03 & 1.5106280957127742 & 0.36589818203583935 & 1.17500019
            & 3.02702188E-02 \\
    \hline
      04 & 1.2042015115607474 & 5.9471597883812510E-002 & 0.612500131
            & 0.532229841 \\
    \hline
      05 & 1.1464638070151236 & 1.7338933381887411E-003 & 0.893750191
            & 0.250979781 \\
    \hline
      06 & 1.1447314160015734 & 1.5023246384693323E-006 & 1.03437519 & 0.110354781 \\
    \hline
      07 & 1.1447299136780633 & 1.1284306822290091E-012 & 1.10468769 & 4.00422812E-02 \\
    \hline
      08 & No Data & No Data & 1.13984394 & 4.88603115E-03 \\
    \hline
      09 & No Data & No Data & 1.15742207 & 1.26920938E-02 \\
    \hline
      10 & No Data & No Data   & 1.14863300  & 3.90303135E-03  \\
    \hline
      11 & No Data & No Data & 1.14423847 & 4.91499901E-04 \\
    \hline
      12 & No Data & No Data & 1.14643574 & 1.70576572E-03 \\
    \hline
      13 & No Data & No Data & 1.14533710 & 6.07132912E-04 \\
    \hline
      14 & No Data & No Data & 1.14478779 & 5.78165054E-05 \\
    \hline
      15 & No Data & No Data & 1.14451313 & 2.16841698E-04 \\
    \hline
      16 & No Data & No Data & 1.14465046 & 7.95125961E-05 \\
    \hline
      17 & No Data & No Data & 1.14471912 & 1.08480453E-05 \\
    \hline
      18 & No Data & No Data & 1.14475346 & 2.34842300E-05 \\
    \hline
      19 & No Data & No Data & 1.14473629 & 6.31809235E-06 \\
    \hline
      20 & No Data & No Data & 1.14472771 & 2.26497650E-06 \\
    \hline
      21 & No Data & No Data & 1.14473200 & 2.02655792E-06 \\
    \hline
      22 & No Data & No Data & 1.14472985 & 1.19209290E-07 \\
    \hline
      23 & No Data & No Data & 1.14473093 & 9.53674316E-07 \\
    \hline
      24 & No Data & No Data & 1.14473033 & 3.57627869E-07 \\
    \hline
  \end{tabular}
  \end{center}
\end{table}
\vskip0.1in\hrule\vskip0.1in
\noindent
As we can easily see, Newton's method gives a good approximation within just a
few iterations. In fact, in the next section, we will discuss how Newton's
method converges along with definitions of different orders of convergence.
\vskip0.1in\hrule\vskip0.1in
\newpage
%%%%%%%%%%%%%%%%%%%%%%%%%%%%%%%%%%%%%%%%%%%%%%%%%%%%%%%%%%%%%%%%%%%%%%%%%%%%%%%%
%%%%%%%%%%%%%%%%%%%%%%%%%%%%%%%%%%%%%%%%%%%%%%%%%%%%%%%%%%%%%%%%%%%%%%%%%%%%%%%%
\vskip0.1in\hrule\vskip0.1in
\noindent
{\bf Root Finding Problems: Convergence of Newton's Method}
\vskip0.1in\hrule\vskip0.1in
\noindent
In this section, we are going to show that the sequence of approximations
produced by Newton's method indeed converges to a single value. To start the
analysis, we will define the error in an approximation, \(x_k\), and the exact
value, \(x^*\) by
\[
  e_k = x_k - x^*
\]
Then we can write
\[
  e_{k+1} = x_{k+1} - x^* = x_k - {{f(x_k)}\over{f'(x_k)}} - x^*
            = e_k - {{f(x_k)}\over{f'(x_k)}}
\]
and so,
\[
  e_{k+1} = {{e_k\ f'(x_k) - f(x_k)}\over{f'(x_k)}}
\]
Note that we know \(f(x^*)=0\) since \(x^*\) is a root of \(f\). Now, expanding the
function in a Taylor series at \(x^*\) gives
\[
  0 = f(x^*) = f(x_k-e_k) = f(x_k) - e_k\ f'(x_k) + {1\over 2} e_k^2 f''(\xi) 
\]
where \(\xi\) is a point between \(x_k\) and \(x^*\). We can solve for the numerator
in the error expression as follows.
\[
  e_k\ f'(x_k) - f(x_k) = - {1\over 2} e_k^2 f''(\xi) 
\]
Therefore,
\[
  e_{k+1}
     = {{- {1\over 2} e_k^2 f''(\xi)}\over{f'(x_k)}}
     = {{- {1\over 2} f''(\xi)}\over{f'(x_k)}} e_k^2
\]
Taking absolute values of both sides gives
\[
  | e_{k+1} |
     = | {{{1\over 2} f''(\xi)}\over{f'(x_k)}} | | e_k |^2
     \leq C | e_k |^2
\]
where \(C\) depends on the quotient of the derivative terms.

Note that if \(f\) is twice continuously differentiable and \(f'(x^*)\neq 0\), then
\(C\) is a nonnegative constant. This analysis indicates that the error at the
next step, \(x_{k+1}\), is bounded by the square of the error at the current
approximation, \(x_k\). This is a better result than either functional iteration
or the Bisection method. In the next section we will define different rates of
convergence.
\vskip0.1in\hrule\vskip0.1in
\newpage
%%%%%%%%%%%%%%%%%%%%%%%%%%%%%%%%%%%%%%%%%%%%%%%%%%%%%%%%%%%%%%%%%%%%%%%%%%%%%%%%
%%%%%%%%%%%%%%%%%%%%%%%%%%%%%%%%%%%%%%%%%%%%%%%%%%%%%%%%%%%%%%%%%%%%%%%%%%%%%%%%
\vskip0.1in\hrule\vskip0.1in
\noindent
{\bf Root Finding Problems: Rate of Convergence Definitions}
\vskip0.1in\hrule\vskip0.1in
\noindent
Using iterative methods requires an understanding of convergence of the sequence
of approximations generated. First, it is important to be able to show that the
sequence of approximations converges to the solution of the mathematical problem
that is under consideration. In addition, we should be interested in how fast
the sequence of approximations converges. Now for some definitions.
\vskip0.1in\hrule\vskip0.1in
\begin{definition}
  Suppose that an algorithm generates a sequence of approximations, \(\{x_k\{\),
  that converges to a value, \(x^*\). Define the error in a sequence element by
  \[
    e_k = x_k - x^*
  \]
  If the sequence elements satisfy
  \[
    | e_{k+1} | \leq C\ | e_k |^r
  \]
  for a positive constant, \(C\), not depending on \(k\), the power \(r\) is called
  the rate of convergence of the sequence and thus the algorithm.
\end{definition}
\vskip0.1in\hrule\vskip0.1in
\noindent
Using the definition of rate of convergence, it is easy to see from the analysis
for the three root finding algorithms can be determined. The rate of convergence
for functional iteration and the Bisection method is \(r=1\) and the rate of
convergence for Newton's method is \(r=2\).

The next definition gives terms for different types of convergence.
\vskip0.1in\hrule\vskip0.1in
\begin{definition}
  Suppose that an algorithm produces a sequence of approximations, \(\{x_k\}\),
  thet converges to a value \(x^*\). Then if the rate of convergence is \(r=1\) the
  algorithm is tsaid to be linearly convergent. If the rate of convergence is
  \(r=2\), the algorithm is said to be quadratically convergent. If the rate of
  convergence is between \(r=1\) and \(r=2\), the algorithm is said to be
  super-linearly convergent.
\end{definition}
\vskip0.1in\hrule\vskip0.1in
\noindent
As mentioned above, functional iteration is at best linearly convergent, the
Bisection method is linearly convergent, and Newton's method is quadratically
convergent. At this point none of the algorithms for root finding satisfy the
definition of super-linear convergence. As we will see, the Secant method
presented next will provide an example of a super-linearly convergent algorithm.
\vskip0.1in\hrule\vskip0.1in
\newpage
%%%%%%%%%%%%%%%%%%%%%%%%%%%%%%%%%%%%%%%%%%%%%%%%%%%%%%%%%%%%%%%%%%%%%%%%%%%%%%%%
%%%%%%%%%%%%%%%%%%%%%%%%%%%%%%%%%%%%%%%%%%%%%%%%%%%%%%%%%%%%%%%%%%%%%%%%%%%%%%%%
\vskip0.1in\hrule\vskip0.1in
\noindent
{\bf Root Finding Problems: Approximating Newton's Method - the Secant Method}
\vskip0.1in\hrule\vskip0.1in
\noindent
Newton's method requires the evaluation of both the function and derivative of
the function at each iteration. In many cases it is either difficult, if not
impossible, to evaluate the derivative of the function. In some cases the
derivative may not even be available. In these cases, Newton's method will not
work. To deal with this issue, we can approximate Newton's method by
approximating the derivative of the function. The approximation we will use is
\[
  f'(x_k) \approx {{f(x_k) - f(x_{k-1})}\over{x_k - x_{k-1} }}
\] 
The use of the difference quotient requires only function evaluations to
approximate the derivative. We can substitute this approximation into Newton's
method to obtain
\[
  x_{k+1} = x_k - {{f(x_k)}\over{f'(x_k)}}
          \approx x_k - f(x_k)\ {{
                                  1
                                }\over{
                                  {{
                                    f(x_k) - f(x_{k-1})
                                  }\over{
                                    x_k - x_{k-1}
                                  }}
                                }}
\]
The result of this approximation is the Secant method that can be written in the
folloaing form. Given two initial values, \(x_0\) and \(x_1\), compute the ssequence
\(\{ x_k\}\) using
\[
  x_{k+1} = x_k - f(x_k)\ {{
                            x_k - x_{k-1}
                          }\over{
                            f(x_k) - f(x_{k-1})
                          }}
\]
for \(k=1,2,\ldots\). There are a couple of observations that are easy to see.
First, the approximation of the derivative will result in a loss of accuracy
and likely effect the convergence rate for the sequence. In fact, the
convergence rate of the Secant method is \(r\approx (1+\sqrt{5})/2\). This is
between linear convergence and quadratic convergence. Thus, the Secant method
is a superlinearly convergent method.

The convergence analysis will be given in the next section. Before going into
the analysis of convergence of the Secant method a few remarks are in order.
\vskip0.1in\hrule\vskip0.1in
\begin{list}{$\bullet$}{\usecounter{beans} \parsep=0pt \listparindent=0pt
\topsep=0pt \rightmargin=.35in \leftmargin=.35in \labelsep=5 pt
\itemsep=2pt}
  \item The secant method is used when it is not possible to compoute the
        derivative of the function. It might be the case the no closed formula
        is avaialable for use in Newton's method.
  \item Another issue involves the complexity of the derivative. In some
        problems the derivative may be too complex for evaluation  and the use
        of an approximation may result in a more efficient algorithm.
  \item In multi-dimensional problems, such as multivariable optimization
        problems, extensions of the secant method produce more efficient
        algorithms. In fact, many mutivariable algorithms use some modification
        of the Secant method as a starting point.
\end{list}
\vskip0.1in\hrule\vskip0.1in
With these ideas in mind, the next section will present an analysis of the
convergence of the Secant method.
\vskip0.1in\hrule\vskip0.1in
\newpage
%%%%%%%%%%%%%%%%%%%%%%%%%%%%%%%%%%%%%%%%%%%%%%%%%%%%%%%%%%%%%%%%%%%%%%%%%%%%%%%%
%%%%%%%%%%%%%%%%%%%%%%%%%%%%%%%%%%%%%%%%%%%%%%%%%%%%%%%%%%%%%%%%%%%%%%%%%%%%%%%%
\vskip0.1in\hrule\vskip0.1in
\noindent
{\bf Root Finding Problems: Convergence Analysis for the Secant Method}
\vskip0.1in\hrule\vskip0.1in
\noindent
In this section, the convergence of the Secant method is presented. The analysis
starts by considering the error at each iteration as follows.
\begin{eqnarray*}
  e_{k+1} & = & x_{k+1} - x^* \\
          & = & x_k - f(x_k)\ {{
                            x_k - x_{k-1}
                          }\over{
                            f(x_k) - f(x_{k-1})
                          }} - x^* \\
          & = & e_k - f(x_k)\ {{
                            x_k - x_{k-1}
                          }\over{
                            f(x_k) - f(x_{k-1})
                          }}
\end{eqnarray*}
Next, we can apply a bit of algebra to obtain
\begin{eqnarray*}
  e_{k+1} & = & {{
                  e_k\ \left( f(x_k) - f(x_{k-1}) \right)
                    - f(x_k)\ \left( x_k - x_{k-1} \right)
                }\over{
                  f(x_k) - f(x_{k-1}
                }} \\
          & = & {{
                  e_k\ \left( f(x_k) - f(x_{k-1}) \right)
                    - f(x_k)\ \left( x_k - x^* + x^* - x_{k-1} \right)
                }\over{
                  f(x_k) - f(x_{k-1})
                }}
\end{eqnarray*}
or
\begin{eqnarray*}
  e_{k+1} & = & {{
                  e_k\ \left( f(x_k) - f(x_{k-1}) \right)
                    - f(x_k)\ \left( e_k - e_{k-1} \right)
                }\over{
                  f(x_k) - f(x_{k-1})
                }} \\
          & = & {{
                  f(x_k)\ e_{k-1} - f(x_{k-1}) e_{k-1} 
                }\over{
                  f(x_k) - f(x_{k-1})
                }}
\end{eqnarray*}
The next step in the proof involves factoring out the product of the errors
\(e_k\ e_{k-1}\). The result is the following
\[
  e_{k+1} = {{
              \left( {{
                       f(x_k) 
                     }\over{
                       e_k
                     }}
                   - {{
                     f(x_{k-1})
                     }\over{
                       e_{k-1}
                     }}
              \right)\ e_{k}\ e_{k-1}
            }\over{
              f(x_k) - f(x_{k-1})
            }}
\]
The next step will reduce the two ratios in the denominator of the expression.
The function values will be expanded on a Taylor series about the location of
the root. So, we can expand the first term as follows.
\[
  f(x_k) = f(x^*+e_k) = f(x^*) + f'(x^*)\ e_k + {1\over 2}\ f''(x^*) e_k^2
            + O(e_k^3)
\]
We know that \(f(x^*)=0\) that gets rid of the first term. Then the ratio in our
estimate can be written as follows
\begin{eqnarray*}
  {{ f(x_k) }\over{ e_k }}
    & = & {{
            f'(x^*)\ e_k + {1\over 2}\ f''(x^*) e_k^2 + O(e_k^3)
          }\over{
            e_k
          }} \\
    & = & f'(x^*) + {1\over 2}\ f''(x^*) e_k + O(e_k^2)
\end{eqnarray*}
The same sort of computation can be applied to the other term in the expression
for the error. That is,
\begin{eqnarray*}
  {{ f(x_{k-1}) }\over{ e_{k-1} }}
    & = & {{
        f'(x^*)\ e_{k-1} + {1\over 2}\ f''(x^*) e_{k-1}^2 + O(e_{k-1}^3)
      }\over{
        e_{k-1}
      }} \\
    & = & f'(x^*) + {1\over 2}\ f''(x^*) e_{k-1} + O(e_{k-1}^2)
\end{eqnarray*}
The error term in the denominator of each of these terms has been divided out.
Taking the difference in the equations for \(e_k\) and \(e_{k+1}\) gives
\begin{eqnarray*}
  {{
    f(x_k) 
  }\over{
    e_k
  }}
- {{
    f(x_{k-1})
  }\over{
    e_{k-1}
  }}
  & = & f'(x^*) + {1\over 2}\ f''(x^*) e_k + O(e_k^2)
            - f'(x^*) + {1\over 2}\ f''(x^*) e_{k-1} + O(e_{k-1}^2) \\
  & = & {1\over 2}\ f''(x^*) \left( e_k - e_{k-1} \right) + O(\max_{k} e_k^2)
\end{eqnarray*}
The next step is to go back to the error formula above and do a bit more algebra
on the form. 
\begin{eqnarray*}
  e_{k+1} & = & {{
                  \left( {{
                           f(x_k) 
                         }\over{
                           e_k
                         }}
                       - {{
                         f(x_{k-1})
                         }\over{
                           e_{k-1}
                         }}
                  \right)\ e_{k}\ e_{k-1}
                }\over{
                  f(x_k) - f(x_{k-1})
                }} \\
          & = & {{
                  \left( {{
                           f(x_k) 
                         }\over{
                           e_k
                         }}
                       - {{
                         f(x_{k-1})
                         }\over{
                           e_{k-1}
                         }}
                  \right)\ e_{k}\ e_{k-1}
                }\over{
                  f(x_k) - f(x_{k-1})
                }} {{ x_k - x_{k-1} }\over{ x_k - x_{k-1} }} \\
          & = & {{
                  \left( {{
                           f(x_k) 
                         }\over{
                           e_k
                         }}
                       - {{
                         f(x_{k-1})
                         }\over{
                           e_{k-1}
                         }}
                  \right)\ e_{k}\ e_{k-1}
                }\over{
                  x_k - x_{k-1} 
                }} 
             \times
                \left( {{ 
                         f(x_k) - f(x_{k-1})
                       }\over{
                         x_k - x_{k-1}
                      }}
                \right)^{-1}
\end{eqnarray*}
This amounts to multiplying and dividing by a convenient form of one as seen
above. We can replace the numerator in the first term of the product with the
estimage from above and also notice that $x_k-x_{k-1}=e_k-e_{k-1}$. So,
\begin{eqnarray*}
  e_{k+1} & = & {{
                  \left( 
                    {1\over 2}\ f''(x^*) ( e_k - e_{k-1} ) + O(\max_{k} e_k^2)
                  \right)\ e_{k}\ e_{k-1}
                }\over{
                  e_k - e_{k-1} 
                }} 
             \times
                \left( {{ 
                         f(x_k) - f(x_{k-1})
                       }\over{
                         x_k - x_{k-1}
                      }}
                \right)^{-1} \\
          & \approx & {{
                        \left( 
                          {1\over 2}\ f''(x^*) ( e_k - e_{k-1} )
                        \right)\ e_{k}\ e_{k-1}
                      }\over{
                        e_k - e_{k-1} 
                      }} 
                   \times
                      \left( {{ 
                               f(x_k) - f(x_{k-1})
                             }\over{
                               x_k - x_{k-1}
                            }}
                      \right)^{-1} \\
          & = & \left( {1\over 2}\ f''(x^*) \right)\ e_{k}\ e_{k-1}
                \times
                \left( {{ 
                         f(x_k) - f(x_{k-1})
                       }\over{
                         x_k - x_{k-1}
                       }}
                \right)^{-1}
\end{eqnarray*}
where the term involving $O(\max_{k} e_k^2)$ is assumed to be smaller than the
other terms. Using the mean value theorem for derivatives, we can write
\begin{eqnarray*}
  e_{k+1} & \approx & \left( 
                  {1\over 2}\ f''(x^*) \right)\ e_{k}\ e_{k-1}
                \times
                \left( f'(x^*) \right)^{-1} \\
          & = & {1\over 2}\ {{f''(x^*)}\over{f'(x^*)}}\ e_{k}\ e_{k-1} \\
          & \leq & C\ e_{k}\ e_{k-1}
\end{eqnarray*}
This form looks a bit like the form obtained in the analysis of Newton's method.
The difference is that the errors from the previous two terms appear in the
product. Newton's method depends on $e_k\ e_k=e_k^2$. This will effect the
convergence rate. 

To see the effect on the Secant method, we can set up an asymptotic relationship
of the form
\[
  | e_{k+1} | \approx A | e_k |^r
\]
as $k\rightarrow\infty$. The idea is to determine $r$, the convergence rate for
the method. Note that as $k\rightarrow\infty$ all iterations will behave the
same. So
\[
  | e_k | \approx A | e_{k-1} |^r
   \rightarrow | e_{k-1} | \approx \left( A^{-1} | e_k | \right)^{1/r}
      = A^{-1/r} | e_k |^{1/r}
\]
We are basically solving for the error at the $(k-1)$ iteration in terms of the
error at the $k$ iteration. We can plug the values into the error term for the
secant method as follows. 
\begin{eqnarray*}
  | e_{k+1} | \approx A\ | e^k |^r
          & \approx & C | e_k |\ | e_{k-1} | \\
          & \approx & C | e_k |\ | A^{-1/r}| | | e_k |^{1/r}| \\
          & \approx & C\ A^{-1/r} | e_k |^{1+1/r}
\end{eqnarray*}
This can be translated into the following equation.
\[
  A^{1+1/r}\ |C^{-1}| \approx | e_k |^{-r+1+1/r}
\]
The left hand side of the approximation is a constant that in general is nonzero
while in the right hand side of the equation ${e^K}\rightarrow 0$. The only way
this makes sens is if the exponent in the error is zero. That is,
\[
  1 - r + {1\over r} = 0 \rightarrow r^2-r-1=0
\]
The roots of the quadratic polynomial are 
\[
  r = {{1\pm\sqrt{5}}\over{2}}
\]
One of the roots is negative which means the root is not valid ($r>0$). The 
positive root is $r\approx 1.62$. So, the final result if
\[
   |e_{k+1}| \leq C\ | e_k |^{1.62}
\]
\vskip0.1in\hrule\vskip0.1in
\newpage
%%%%%%%%%%%%%%%%%%%%%%%%%%%%%%%%%%%%%%%%%%%%%%%%%%%%%%%%%%%%%%%%%%%%%%%%%%%%%%%%
%%%%%%%%%%%%%%%%%%%%%%%%%%%%%%%%%%%%%%%%%%%%%%%%%%%%%%%%%%%%%%%%%%%%%%%%%%%%%%%%
\vskip0.1in\hrule\vskip0.1in
\noindent
{\bf Root Finding Problems: Secants Method Example and Comparison}
\vskip0.1in\hrule\vskip0.1in
\noindent
Again, we will use the same example
\[
  f(x) = e^x - \pi = 0
\]
to test the Secant method and compare the results to the results obtained using
Newton's method. It is important that the method will need to initial guesses
to get started, while Newton's method requires a single starting point.
\vskip0.1in\hrule\vskip0.1in
\begin{table}[h]
\caption{Results for the Secant Method Compared to Newton's Method}
  \vskip0.1in
  \begin{center}
  \begin{tabular}{|c||c|c||c|c|}
    \hline
    Iteration No. & Secant Method & Secant Method Error & Newton Method
                                                & Newton Method Error \\
    \hline
      01 & 0.0000000000000000  & 1.1447299136769349 & 0.0000000000000000
                 & 1.1447299136769349 \\
    \hline
      02 & 1.0000000000000000 & 0.14472991367693488 & 2.1415927410125732
                 & 0.99686282733563836 \\
    \hline
      03 & 1.2463570908697517 & 0.10162717719281678 & 1.5106280957127742
                 & 0.36589818203583935 \\
    \hline
      04 & 1.1373319288158861 & 7.3979848610488119E-003 & 1.2042015115607474
                 & 5.9471597883812510E-002 \\
    \hline
      05 & 1.1443599214178914 & 3.6999225904343902E-004 & 1.1464638070151236
                 & 1.7338933381887411E-003 \\
    \hline
      06 & 1.1447312840476851 & 1.3703707502088491E-006 & 1.1447314160015734
                 & 1.5023246384693323E-006 \\
    \hline
      07 & 1.1447299134234061 & 2.5352875354656135E-010 & 1.1447299136780633
                 & 1.1284306822290091E-012 \\
    \hline
  \end{tabular}
  \end{center}
\end{table}
\vskip0.1in\hrule\vskip0.1in
\noindent
Comparing the results in the previous problem we can see that the results are
roughly the same for both methods. The error in the Newton metho approximation
is just a little better than the Secant method. In this example, the computed
convergence rate using the data generated is $r\approx 1.678$ which is a bit
higher than predicted in the previous section.
\vskip0.1in\hrule\vskip0.1in
\newpage
%%%%%%%%%%%%%%%%%%%%%%%%%%%%%%%%%%%%%%%%%%%%%%%%%%%%%%%%%%%%%%%%%%%%%%%%%%%%%%%%
%%%%%%%%%%%%%%%%%%%%%%%%%%%%%%%%%%%%%%%%%%%%%%%%%%%%%%%%%%%%%%%%%%%%%%%%%%%%%%%%
\vskip0.1in\hrule\vskip0.1in
\noindent
{\bf Root Finding Problems: Summary of Results for Basic Methods}
\vskip0.1in\hrule\vskip0.1in
\noindent
In this section the four basic methods are summarized and compared in terms of
the pros and cons of each. The four methods, (1) functional iteration, (2) the
Bisection method, (3) Newton's method, and (4) the Secant method.
\begin{enumerate}
  \item {\bf Functional Iteration:} This method is easy to develop and implement
        into a computer code. The downside is that it is important to create an 
        equivalent fixed point problem that generates approximations that
        converge rapidly to the root of the original problem. In general, this
        may be a very challenging problem. For this reason functional iteration
        is not widely used. In many cases, when the approximations converge to
        the desired solution, the convergence is very slow.
  \item {\bf Bisection Method:} This method requires a continuous function and
        an interval, \([a,b]\), such that $f(a)f(b)<0$. Once an appropriate
        interval is determined Bisection will converge linearly to the a root
        of the original problem. Linear convergence is slow. The algorithm is
        relatively easy to implement into a computer code and given a tolerance
        for the solution, an exact number of iterations can be determined which
        makes for a more efficient computer code. Convergence of the Bisection
        method is linear at best.
  \item {\bf Newton's Method:} The gold standard for finding roots of functions
        that are smooth is Newton's method. This method is quadratically
        convergent which means a small number of iterations are needed to get
        close to the root of a function. The requirements for successful use
        of Newton's method are (a) the function must be twice continuously
        differentiable, (2) the derivative of the function must not be zero near
        the solution, and (3) the initial guess must be sufficiently close to
        a root of the function. All of this restrictions can cause problems in
        real world problem. 
  \item {\bf The Secant Method:} The Secant method is used when the derivative
        of the function is either not available or the derivative is too
        expensive to compute. When this is the case, the derivative is
        approximated by a finite difference which requires two points on the
        secant line joining two points on the graph of the function. The 
        Secant method is simpler than Newton's method, but is restricted in the
        same way as Newton's method. The convergence is superlinear and the
        method is significantly better than Bisection in terms of convergence.
\end{enumerate}
\vskip0.1in\hrule\vskip0.1in
\newpage
%%%%%%%%%%%%%%%%%%%%%%%%%%%%%%%%%%%%%%%%%%%%%%%%%%%%%%%%%%%%%%%%%%%%%%%%%%%%%%%%
%%%%%%%%%%%%%%%%%%%%%%%%%%%%%%%%%%%%%%%%%%%%%%%%%%%%%%%%%%%%%%%%%%%%%%%%%%%%%%%%
\vskip0.1in\hrule\vskip0.1in
\noindent
{\bf Root Finding Problems: General Problems and Hybrid Methods}
\vskip0.1in\hrule\vskip0.1in
\noindent
In a real root finding problem, the location of the root may not be easy to
estimate. This may cause the Newton method and Secant method to fail due to the
requirement of starting sufficiently close to the exact value of the root.
However, if we can determine an interval, \([a,b]\), where $f(a)f(b)<0$, the
Bisection method will converge albeit slowly. A strategy that starts with a few
steps of Bisection to get closer to the root and then trying a Newton or Secant
step might be a good alternatice. This brings us to the idea of Hybrid root
finding methods.

There are any number of strategies for development of Hybrid methods. We will
develop a stratefy based on the following.
\begin{enumerate}
  \item Find an interval \([a,b]\) on which $f(a)f(b)<0$. If $f$ is continuous
        on the interval, there must be root in the interval due to an
        application of the Intermediate Value Theorem. We can compute a bound
        for the initial error in the approximation
        \[
           e_0 = | b - a |
        \]
  \item Next, reduce the error in the approximation one order of magnitdue using
        the Bisection method. This will be fleshed out below.
  \item Try a step of Newton's method.
        \begin{enumerate}
          \item If the Newton step stays in the same interval we can continue
                with Newton steps until convergence.
          \item If the Newton step fails to stay in the interval from the
                Bisection iterations, go back to Bisection.
        \end{enumerate}
  \item Output the value after Newton's method converges.
\end{enumerate}
Now to how to reduce the error by an order of magnitude.
\vskip0.1in\hrule\vskip0.1in
\newpage
%%%%%%%%%%%%%%%%%%%%%%%%%%%%%%%%%%%%%%%%%%%%%%%%%%%%%%%%%%%%%%%%%%%%%%%%%%%%%%%%
%%%%%%%%%%%%%%%%%%%%%%%%%%%%%%%%%%%%%%%%%%%%%%%%%%%%%%%%%%%%%%%%%%%%%%%%%%%%%%%%
\vskip0.1in\hrule\vskip0.1in
\noindent
{\bf Root Finding Problems: Multiple Root Example:}
\vskip0.1in\hrule\vskip0.1in
\noindent
This example,...
\[
   f(x) = sin(10\ x^2 + 3)
\]
%%%%%%%%%%%%%%%%%%%%%%%%%%%%%%%%%%%%%%%%%%%%%%%%%%%%%%%%%%%%%%%%%%%%%%%%%%%%%%%%
%%%%%%%%%%%%%%%%%%%%%%%%%%%%%%%%%%%%%%%%%%%%%%%%%%%%%%%%%%%%%%%%%%%%%%%%%%%%%%%%
\end{document}
