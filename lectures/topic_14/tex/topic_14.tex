\documentclass[10pt,fleqn]{article}
\usepackage{hyperref}
\usepackage{graphicx}


\setlength{\topmargin}{-.75in}
\addtolength{\textheight}{2.00in}
\setlength{\oddsidemargin}{.00in}
\addtolength{\textwidth}{.75in}

\nofiles

\pagestyle{empty}

\setlength{\parindent}{0in}

% new math commands


\setlength{\oddsidemargin}{-0.25in}
\setlength{\evensidemargin}{-0.25in}
\setlength{\textwidth}{6.75in}
\setlength{\headheight}{0.0in}
\setlength{\topmargin}{-0.25in}
\setlength{\textheight}{9.00in}

\makeindex

\usepackage{mathrsfs}

%\usepackage[pdftex]{graphicx}
\usepackage{epstopdf}

\newcounter{beans}

\newcommand{\ds}{\displaystyle}
\newcommand{\limit}[2]{\displaystyle\lim_{#1\to#2}}

\newcommand{\binomial}[2]{\ \left( \begin{array}{c}
                                  #1 \\
                                  #2
                                 \end{array}
                            \right) \
                         }
\newcommand{\ExampleRule}[2]
  {
  \noindent
  \rule{\linewidth}{1pt}
  \begin{example}
    #1
    \label{#2}
  \end{example}
  \rule{\linewidth}{1pt}
  \vskip0.125in
  }

\newcommand{\defbox}[1]
  {
   \ \\
   \noindent
   \setlength\fboxrule{1pt}
   \fbox{
        \begin{minipage}{6.5in}
          #1
        \end{minipage}
        }
   \ \\
  }
\newcommand{\verysmallworkbox}[1]
  {
   \ \\
   \noindent
   \setlength\fboxrule{1pt}
   \fbox{
        \begin{minipage}{6.5in}
           #1
           \ \\
           \vskip0.5in \ \\
           \ \\
        \end{minipage}
        }
   \ \\
  }
\newcommand{\smallworkbox}[1]
  {
   \ \\
   \noindent
   \setlength\fboxrule{1pt}
   \fbox{
        \begin{minipage}{6.5in}
           #1
           \ \\
           \vskip2.5in \ \\
           \ \\
        \end{minipage}
        }
   \ \\
  }
\newcommand{\halfworkbox}[1]
  {
   \ \\
   \noindent
   \setlength\fboxrule{1pt}
   \fbox{
        \begin{minipage}{6.5in}
           #1 \hfill
           \ \\
           \vskip3.25in \ \\
           \ \\
        \end{minipage}
        }
   \ \\
  }
\newcommand{\largeworkbox}[1]
  {
   \ \\
   \noindent
   \setlength\fboxrule{1pt}
   \fbox{
        \begin{minipage}{6.5in}
           #1
           \ \\
           \vskip7.5in \ \\
           \ \\
        \end{minipage}
        }
   \ \\
  }
\newcommand{\flexworkbox}[2]
  {
   \ \\
   \noindent
   \setlength\fboxrule{1pt}
   \fbox{
        \begin{minipage}{6.5in}
           #1
           \ \\

           \vskip#2 \ \\
           \ \\
        \end{minipage}
        }
   \ \\
  }


% symbols for sets of numbers

\newcommand{\natnumb}{$\cal N$}
\newcommand{\whonumb}{$\cal W$}
\newcommand{\intnumb}{$\cal Z$}
\newcommand{\ratnumb}{$\cal Q$}
\newcommand{\irrnumb}{$\cal I$}
\newcommand{\realnumb}{$\cal R$}
\newcommand{\cmplxnumb}{$\cal C$}

% misc. commands

\newcommand{\mma}{{\it Mathematica}}
\newcommand{\sech}{\mbox{ sech}}
 
\newtheorem{theorem}{Theorem}
\newtheorem{example}{Example}
\newtheorem{definition}{Definition}
\newtheorem{problem}{Problem}

\setcounter{secnumdepth}{2}
\setcounter{tocdepth}{4}


\begin{document}
%%%%%%%%%%%%%%%%%%%%%%%%%%%%%%%%%%%%%%%%%%%%%%%%%%%%%%%%%%%%%%%%%%%%%%%%%%%%%%%%
%%%%%%%%%%%%%%%%%%%%%%%%%%%%%%%%%%%%%%%%%%%%%%%%%%%%%%%%%%%%%%%%%%%%%%%%%%%%%%%%
\vskip0.1in\hrule\vskip0.1in \noindent
{\bf Math 4610 Fundamentals of Computational Mathematics  - Topic 14.}
\vskip0.1in\hrule\vskip0.1in \noindent
In this section we will analyze approximation of derivatives using various
difference quotients. Recall that the derivative of a function \(f(x)\) exists
if the following limit exists.
\[
  f'(a) = {{df}\over{dx}}(a) = \lim_{h\rightarrow 0}\ {{f(a+h)-f(a)}\over{h}}
\]
We then turn around and use the approximation
\[
  f'(a) \approx {{f(a+h)-f(a)}\over{h}}
\]
There are at least a couple of questions that we should want the answers to.
First, how accurate is the approximation? Another question to answer might be
are there better approximations we can come up with?

To complete the analysis in this section we won't need a computer. Instead we
will resort to Taylor series expansions.
%%%%%%%%%%%%%%%%%%%%%%%%%%%%%%%%%%%%%%%%%%%%%%%%%%%%%%%%%%%%%%%%%%%%%%%%%%%%%%%%
%%%%%%%%%%%%%%%%%%%%%%%%%%%%%%%%%%%%%%%%%%%%%%%%%%%%%%%%%%%%%%%%%%%%%%%%%%%%%%%%
\vskip0.1in\hrule\vskip0.1in \noindent
{\bf Some Useful Taylor Series Expansions:}
\vskip0.1in\hrule\vskip0.1in \noindent
In this section several expansions will be computed.
\vskip0.1in\hrule\vskip0.1in \noindent
{\bf Approximation To the Left Side:}
\vskip0.1in\hrule\vskip0.1in \noindent
If we need to approximate using \(f(a+h)\) we can write an expansion about \(a\)
as follows.
\[
  f(a+h) = f(a) + h\ f'(a) + {1\over 2}\ h^2 f''(a) + {1\over 6}] h^3 f'''(a)
               + \cdots
\]
For the other side we can write
\[
  f(a-h) = f(a) - h\ f'(a) + {1\over 2}\ h^2 f''(a) - {1\over 6}] h^3 f'''(a)
               + \cdots
\]
Note that the expansion of \(f(a)=f(a)\). Typically, we will look to expand any
terms about the same point, say \(a\).

We can continue in this manner to write
\[
  f(a+2h) = f(a) + 2h\ f'(a) + 2h^2 f''(a) + {4\over 3}] h^3 f'''(a) + \cdots
\]
and
\[
  f(a-2h) = f(a) - 2h\ f'(a) + 2h^2 f''(a) - {4\over 3}] h^3 f'''(a) + \cdots
\]
to expand the function at a point a distance of \(2h\) away. It is not hard to
imageine using more and more values.

As a simple example, consider the approximation of \(f'(a)\) in the following
manner.
\[
  f'(a) \approx {{f(a)-f(a-h)}\over{h}}
\]
The goal is to expand everything in the expression on the right hand side of the
approximation at the point, \(a\). Using the expansions above we can write
\begin{eqnarray*}
  {{f(a)-f(a-h)}\over{h}} & = & {1\over h} \left( f(a)-f(a-h) \right) \\
       & = & {1\over h}
        \left( f(a)-( f(a) - h f'(a)
               + {1\over 2} h^2 f''(a) - \cdots ) \right) \\
       & = & {1\over h}
        \left( h f'(a) - {1\over 2} h^2 f''(a) + \cdots ) \right) \\
       & = & f'(a) - {1\over 2} h f''(a) + \cdots ) 
\end{eqnarray*}
As mentioned in the last topic, this can be interpretted as follows. The
one-sided difference quotient used in this example is equal to \(f'(a)\) plus
an infinite number of terms from the Taylor series expansion about \(a\).

We can actually be a bit more precise in the estimation of these quantities. The
Taylor series with remainder will do the trick.
%%%%%%%%%%%%%%%%%%%%%%%%%%%%%%%%%%%%%%%%%%%%%%%%%%%%%%%%%%%%%%%%%%%%%%%%%%%%%%%%
%%%%%%%%%%%%%%%%%%%%%%%%%%%%%%%%%%%%%%%%%%%%%%%%%%%%%%%%%%%%%%%%%%%%%%%%%%%%%%%%
\vskip0.1in\hrule\vskip0.1in \noindent
{\bf Using the Taylor Series with Remainder}
\vskip0.1in\hrule\vskip0.1in \noindent
Using the Taylor series with remainder we can write
\[
  f(a-h) = f(a) - h\ f'(a) + {1\over 2}\ h^2 f''(\xi)
\]
where \(\xi\) is between \(a\) and \(a+h\). The difference is that the expansion
now has a finite number of terms. We can perform the same analysis as above we
can write
\[
  {{f(a)-f(a-h)}\over{h}} = f'(a) - {1\over 2} h f''(\xi)
\]
As long as the function, \(f(x)\), has two continuous derivatives, we can bound
the remainder term and as \(h\rightarrow 0\) the remainder term will approach
zero. Now, the statement is that the difference quotient is equal to \(f'(a)\)
plus an error term that goes to zero as \(h\) goes to zero.

Note that all of this agrees with what students are taught in a calculus course.
%%%%%%%%%%%%%%%%%%%%%%%%%%%%%%%%%%%%%%%%%%%%%%%%%%%%%%%%%%%%%%%%%%%%%%%%%%%%%%%%
%%%%%%%%%%%%%%%%%%%%%%%%%%%%%%%%%%%%%%%%%%%%%%%%%%%%%%%%%%%%%%%%%%%%%%%%%%%%%%%%
\vskip0.1in\hrule\vskip0.1in \noindent
{\bf Writing The Expansions As An Error:}
\vskip0.1in\hrule\vskip0.1in \noindent
Another way to write the expansions and such is to use an error form for the
equation. For the example in the last section, we can write
\[
  | {{f(a)-f(a-h)}\over{h}} - f'(a) | = | {1\over 2} h f''(\xi) |
\]
This is a more accurate statement of what we are after. That is, what is the
error in using the difference quotient as an approximation of the derivative?

If \(|f(x)|\) is continuous in an interval around \(a\), then there is a
positive constant, \(B>0\), such that
\[
  |f''(\xi)| \leq B
\]
and thus we can write the error as
\[
  | {{f(a)-f(a-h)}\over{h}} - f'(a) | = | {1\over 2} h f''(\xi) |
         \leq {B\over 2} h \leq Ch
\]
where \(C=B/2\). For this sort of result, we would say that the approximation
linearly converges to zero as \(h\) approaches zero.
%%%%%%%%%%%%%%%%%%%%%%%%%%%%%%%%%%%%%%%%%%%%%%%%%%%%%%%%%%%%%%%%%%%%%%%%%%%%%%%%
%%%%%%%%%%%%%%%%%%%%%%%%%%%%%%%%%%%%%%%%%%%%%%%%%%%%%%%%%%%%%%%%%%%%%%%%%%%%%%%%
\vskip0.1in\hrule\vskip0.1in\noindent
{\bf Order of Accuracy in an Approximation:} 
\vskip0.1in\hrule\vskip0.1in\noindent
In the previous section, the error in the difference quotient approximation
satisfied the bound
\[
  E = | {{f(a)-f(a-h)}\over{h}} - f'(a) | \leq C h 
\]
In general, we can write this approximation in the form
\[
  E \leq C h^r
\]
where \(r>0\). For the approximation in the previous section, \(r=1\). For the
general formula above, \(r\), is called the order of convergence or the order of
the approximation. If \(r=1\) the approximation is said to be linear or give
linear convergence and if \(r=1\), the approximation is said to converge
quadratically.
%%%%%%%%%%%%%%%%%%%%%%%%%%%%%%%%%%%%%%%%%%%%%%%%%%%%%%%%%%%%%%%%%%%%%%%%%%%%%%%%
%%%%%%%%%%%%%%%%%%%%%%%%%%%%%%%%%%%%%%%%%%%%%%%%%%%%%%%%%%%%%%%%%%%%%%%%%%%%%%%%
\vskip0.1in\hrule\vskip0.1in\noindent
{\bf A Better Approximation Using a Centered Difference Quotient:} 
\vskip0.1in\hrule\vskip0.1in\noindent
The derivative can also be approximated by a centeral difference,
\[
  f'(a) \approx {{f(a+h)-f(a-h)}\over{2h}}
\]
To determine the error in this approximation we will expand the function values
in the numerator of the difference quotient. Using the expanstions above, the
difference quotient becomes
\begin{eqnarray*}
  {{f(a+h)-f(a-h)}\over{2h}}
     & = & {1\over{2h}} \left( f(a+h) - f(a-h) \right) \\
     & = & {1\over{2h}}
       \left(
         f(a) + h f'(a) + {1\over 2} h^2 f''(a) + {1\over 6} h^3 f'''(\xi)
        - ( f(a) - h f'(a) + {1\over 2} h^2 f''(a) - {1\over 6} h^3 f'''(\eta))
       \right) \\
     & = & {1\over{2h}}
       \left(
         h f'(a) + {1\over 6} h^3 f'''(\xi)
       + h f'(a) + {1\over 6} h^3 f'''(\eta))
       \right) \\
     & = & {1\over{2h}}
       \left(
         2 h f'(a) + {1\over 6} h^3 ( f'''(\xi) + f'''(\eta) )
       \right) \\
     & = & f'(a) + {1\over 3} h^3 ( f'''(\xi) + f'''(\eta) )
\end{eqnarray*}
Notice the two different points, \(\xi\), that resides somewhere between \(a\)
and \(a+h\) and, \(eta\), that resides between \(a\) and \(a-h\). Now, using an
application of the mean value theorem for derivatives, we can write
\[
  {{f(a+h)-f(a-h)}\over{2h}} = f'(a) + {1\over 3} h^2 f'''(\gamma)
\]
So, if the function has three continous derivatives near the point \(a\) then
we can write an error formula as follows.
\[
  E = | {{f(a)-f(a-h)}\over{h}} - f'(a) | \leq C h^2 
\]
where the constant \(C\) depends on \(f'''(x)\). This method is a quadratic
method of approximating the derivative of a function since \(r=2\).

Now to a comparison: If \(h=0.01=10^{-2}\), then
\[
  E \leq C 10^{-2}
\]
for the first approximation and for the central difference
\[
  E \leq C (10^{-2})^2 = C 10^{-4}
\]
Comparing these expressions, it is clear that the central difference
approximation is much better for the same value of the increment, \(h\). So,
there is at least one approximation that is better than the first approximation
we tried.
%%%%%%%%%%%%%%%%%%%%%%%%%%%%%%%%%%%%%%%%%%%%%%%%%%%%%%%%%%%%%%%%%%%%%%%%%%%%%%%%
%%%%%%%%%%%%%%%%%%%%%%%%%%%%%%%%%%%%%%%%%%%%%%%%%%%%%%%%%%%%%%%%%%%%%%%%%%%%%%%%
\vskip0.1in\hrule\vskip0.1in\noindent
{\bf Using Linear Combinations to Compute Higher Order Approximations:} 
\vskip0.1in\hrule\vskip0.1in\noindent
The last question is can we increase the order of accuracy to whatever level we
would like - at least theoretically. The answer is yes. If we consider the forms
for the difference quotients, we can treat the difference quotients as linear
combinations of function values near the point of interest, \(a\). The central
difference approximation can be written as
\[
  f'(a) = {1\over{2h}} f(a+h) - {1\over{2h}} f(a-h)
          = a_1 f(a+h) + a_{-1} f(a-h)
\]
where \(a_1=1/2h\) and \(a_{-1}=-1/2h\).

Maybe we can do better. Let's try a more general lienar combination. That is,
\[
  f'(a) \approx a_{-2} f(a-2h) + a_{-1} f(a-h) + a_0 f(a) + a_1 f(a+h)
              + a_2 f(a+2h)
\]
This leaves us with five unknowns to determine.
%%%%%%%%%%%%%%%%%%%%%%%%%%%%%%%%%%%%%%%%%%%%%%%%%%%%%%%%%%%%%%%%%%%%%%%%%%%%%%%%
%%%%%%%%%%%%%%%%%%%%%%%%%%%%%%%%%%%%%%%%%%%%%%%%%%%%%%%%%%%%%%%%%%%%%%%%%%%%%%%%
\end{document}
