\documentclass[10pt,fleqn]{article}
\usepackage{hyperref}
\usepackage{graphicx}


\setlength{\topmargin}{-.75in}
\addtolength{\textheight}{2.00in}
\setlength{\oddsidemargin}{.00in}
\addtolength{\textwidth}{.75in}

\nofiles

\pagestyle{empty}

\setlength{\parindent}{0in}

% new math commands


\setlength{\oddsidemargin}{-0.25in}
\setlength{\evensidemargin}{-0.25in}
\setlength{\textwidth}{6.75in}
\setlength{\headheight}{0.0in}
\setlength{\topmargin}{-0.25in}
\setlength{\textheight}{9.00in}

\makeindex

\usepackage{mathrsfs}

%\usepackage[pdftex]{graphicx}
\usepackage{epstopdf}

\newcounter{beans}

\newcommand{\ds}{\displaystyle}
\newcommand{\limit}[2]{\displaystyle\lim_{#1\to#2}}

\newcommand{\binomial}[2]{\ \left( \begin{array}{c}
                                  #1 \\
                                  #2
                                 \end{array}
                            \right) \
                         }
\newcommand{\ExampleRule}[2]
  {
  \noindent
  \rule{\linewidth}{1pt}
  \begin{example}
    #1
    \label{#2}
  \end{example}
  \rule{\linewidth}{1pt}
  \vskip0.125in
  }

\newcommand{\defbox}[1]
  {
   \ \\
   \noindent
   \setlength\fboxrule{1pt}
   \fbox{
        \begin{minipage}{6.5in}
          #1
        \end{minipage}
        }
   \ \\
  }
\newcommand{\verysmallworkbox}[1]
  {
   \ \\
   \noindent
   \setlength\fboxrule{1pt}
   \fbox{
        \begin{minipage}{6.5in}
           #1
           \ \\
           \vskip0.5in \ \\
           \ \\
        \end{minipage}
        }
   \ \\
  }
\newcommand{\smallworkbox}[1]
  {
   \ \\
   \noindent
   \setlength\fboxrule{1pt}
   \fbox{
        \begin{minipage}{6.5in}
           #1
           \ \\
           \vskip2.5in \ \\
           \ \\
        \end{minipage}
        }
   \ \\
  }
\newcommand{\halfworkbox}[1]
  {
   \ \\
   \noindent
   \setlength\fboxrule{1pt}
   \fbox{
        \begin{minipage}{6.5in}
           #1 \hfill
           \ \\
           \vskip3.25in \ \\
           \ \\
        \end{minipage}
        }
   \ \\
  }
\newcommand{\largeworkbox}[1]
  {
   \ \\
   \noindent
   \setlength\fboxrule{1pt}
   \fbox{
        \begin{minipage}{6.5in}
           #1
           \ \\
           \vskip7.5in \ \\
           \ \\
        \end{minipage}
        }
   \ \\
  }
\newcommand{\flexworkbox}[2]
  {
   \ \\
   \noindent
   \setlength\fboxrule{1pt}
   \fbox{
        \begin{minipage}{6.5in}
           #1
           \ \\

           \vskip#2 \ \\
           \ \\
        \end{minipage}
        }
   \ \\
  }


% symbols for sets of numbers

\newcommand{\natnumb}{$\cal N$}
\newcommand{\whonumb}{$\cal W$}
\newcommand{\intnumb}{$\cal Z$}
\newcommand{\ratnumb}{$\cal Q$}
\newcommand{\irrnumb}{$\cal I$}
\newcommand{\realnumb}{$\cal R$}
\newcommand{\cmplxnumb}{$\cal C$}

% misc. commands

\newcommand{\mma}{{\it Mathematica}}
\newcommand{\sech}{\mbox{ sech}}
 
\newtheorem{theorem}{Theorem}
\newtheorem{example}{Example}
\newtheorem{definition}{Definition}
\newtheorem{problem}{Problem}

\setcounter{secnumdepth}{2}
\setcounter{tocdepth}{4}


\begin{document}
%%%%%%%%%%%%%%%%%%%%%%%%%%%%%%%%%%%%%%%%%%%%%%%%%%%%%%%%%%%%%%%%%%%%%%%%%%%%%%%%
%%%%%%%%%%%%%%%%%%%%%%%%%%%%%%%%%%%%%%%%%%%%%%%%%%%%%%%%%%%%%%%%%%%%%%%%%%%%%%%%
\vskip0.1in\hrule\vskip0.1in \noindent
{\bf Math 4610 Fundamentals of Computational Mathematics  - Topic 16.}
\vskip0.1in\hrule\vskip0.1in \noindent
In this section we will present a couple of examples of Markdown and HTML for
displaying web content. We will start with an example of a web page written in
the two different mark up languages, one in HTML and one in Markdown to compare
how Markdown and HTML choose to format items on a web page. We will also work
through the page source in each case to compare the effort needed to format
pages in these two web development languages.

In a second example, we will consider the use of MathJax to display mathematical
formulas. Being a quick and slick markup language, Markdown does not play well
with Javascript or similar scripting languages. There are always ways around 
these problems. However, if standardization is important it is important to keep
things simple.
%%%%%%%%%%%%%%%%%%%%%%%%%%%%%%%%%%%%%%%%%%%%%%%%%%%%%%%%%%%%%%%%%%%%%%%%%%%%%%%%
%%%%%%%%%%%%%%%%%%%%%%%%%%%%%%%%%%%%%%%%%%%%%%%%%%%%%%%%%%%%%%%%%%%%%%%%%%%%%%%%
\vskip0.1in\hrule\vskip0.1in \noindent
{\bf The Documents For This Course:}
\vskip0.1in\hrule\vskip0.1in \noindent
Most of the notes in this course can be separated into one of three types of
files. These are (1) Portable Document Format (pdf), (2) Hypertext Markup
Language (html), and (3) Markdown (md) files. Each of these formatting languages
has speccific advantages over the others. pdf files are ubiquitus in working
with computers. There are many billions of documents on the internet that are
pdf documents. It would be surprising if any student registering for a computer
course has not dealy with pdf files. Anyone using the internet for any purpose
is using html files by the hundreds if not thousands per hour. md files are a
recent streamlined display language. If should be noted that most html tags can
be used within individual md files.

Most documents on the repositry for the course have been developed in one, two,
or all three of the formats. There are some text files for code, but the vast
majority of documents are of thre three types mentioned. One last note is that
all browsers give the users access to the page source. Viewing the source of a 
particular page is common. It is relatively easy to see what things others have
done and use or improve formatting using pages as examples. Right clicking on a
Firefox and other browser page will produce a menu of items that include a
\lq\lq View Page Source\rq\rq\ choice. When you click on this choice the actual
code/source is displayed.
%%%%%%%%%%%%%%%%%%%%%%%%%%%%%%%%%%%%%%%%%%%%%%%%%%%%%%%%%%%%%%%%%%%%%%%%%%%%%%%%
%%%%%%%%%%%%%%%%%%%%%%%%%%%%%%%%%%%%%%%%%%%%%%%%%%%%%%%%%%%%%%%%%%%%%%%%%%%%%%%%
\vskip0.1in\hrule\vskip0.1in \noindent
{\bf An Example Comparing Markdown and HTML:}
\vskip0.1in\hrule\vskip0.1in \noindent
The table of contents of the topics covered in the course (including this topic)
are displayed when the following
\href{https://jvkoebbe.github.io/math4610/lectures/toc/md/topic\_toc.md}{link}
is put into your favorite browser.
\begin{verbatim}

      https://jvkoebbe.github.io/math4610/lectures/toc/md/topic_toc.md

\end{verbatim}
You can right click in the background of the page and a drop-down menu will
appear. If you click on the \lq\lq View Page Source\rq\rq\ item the Markdown
used to display the page are shown. 
\begin{verbatim}

# Math 4610 Table of Contents for Lectures by Topic:

The following is the Markdown code for the
\href{../../topic_01/md/topic_01.md}{first topic}) in the course. You will
notice that it is pretty simple to understand and prodouces formatted output on
the web page the is ok.
\begin{verbatim}

        # Topic 1: Math 4610 Fundamentals of Numerical Analysis 

        <hr>

        ## Introduction to the Course

        <hr>

        This course presents fundamental content from numerical methods/analysis and
        introductory computational skills concepts. Numerical methods and some analysis
        of the numerical methods will be presented in the lectures. The content for each
        lecture will be presented from a [list of topics](../../toc/md/topic_toc.md)
        that bounces around between advanced computational skills, numerical methods,
        and some mathematics to prove basic results. The topics will be covered in order
        as a continuous stream of material until the end of the semester. So, we may
        cover one or two topics during a single lecture or we may take multiple lectures
        to cover a single topic.

        The course will not cover a long and thorough list of topics from computational
        mathematics. Instead of covering a long list of numerical methods, we will weave
        in computational skills like parallel programming, data visualization, and a few
        topics that are not ordinarily covered in a numerical analysis or numerical
        methods course.

        The next step by going over a syllabus for the course.

        <hr>

        [Previous](../../toc/md/topic_toc.md) | 
        [Table of Contents](../../toc/md/topic_toc.md) | 
        [Next](../../topic_02/md/topic_02.md)

        <hr>

\end{verbatim}




Link to a couple of table of contents pages and also pages with graphics and
code in them.


%%%%%%%%%%%%%%%%%%%%%%%%%%%%%%%%%%%%%%%%%%%%%%%%%%%%%%%%%%%%%%%%%%%%%%%%%%%%%%%%
%%%%%%%%%%%%%%%%%%%%%%%%%%%%%%%%%%%%%%%%%%%%%%%%%%%%%%%%%%%%%%%%%%%%%%%%%%%%%%%%
\vskip0.1in\hrule\vskip0.1in \noindent
{\bf Displaying Mathematics in HTML:}
\vskip0.1in\hrule\vskip0.1in \noindent
%%%%%%%%%%%%%%%%%%%%%%%%%%%%%%%%%%%%%%%%%%%%%%%%%%%%%%%%%%%%%%%%%%%%%%%%%%%%%%%%
%%%%%%%%%%%%%%%%%%%%%%%%%%%%%%%%%%%%%%%%%%%%%%%%%%%%%%%%%%%%%%%%%%%%%%%%%%%%%%%%
\vskip0.1in\hrule\vskip0.1in \noindent
{\bf Github Pages and Markdown:}
\vskip0.1in\hrule\vskip0.1in \noindent
Talk about accessing html pages through github pages ... usually not a problem.
%%%%%%%%%%%%%%%%%%%%%%%%%%%%%%%%%%%%%%%%%%%%%%%%%%%%%%%%%%%%%%%%%%%%%%%%%%%%%%%%
%%%%%%%%%%%%%%%%%%%%%%%%%%%%%%%%%%%%%%%%%%%%%%%%%%%%%%%%%%%%%%%%%%%%%%%%%%%%%%%%
\end{document}

