\documentclass[10pt,fleqn]{article}
%\usepackage{graphicx}


\setlength{\topmargin}{-.75in}
\addtolength{\textheight}{2.00in}
\setlength{\oddsidemargin}{.00in}
\addtolength{\textwidth}{.75in}

\title{Math 4610 Lecture Notes \\
            \ \\
      Convergence Rates for Root Finding Methods
  \footnote{These notes are part of an Open Resource Educational project
            sponsored by Utah State University}}

\author{Joe Koebbe}

\nofiles

\pagestyle{empty}

\setlength{\parindent}{0in}

% new math commands


\setlength{\oddsidemargin}{-0.25in}
\setlength{\evensidemargin}{-0.25in}
\setlength{\textwidth}{6.75in}
\setlength{\headheight}{0.0in}
\setlength{\topmargin}{-0.25in}
\setlength{\textheight}{9.00in}

\makeindex

\usepackage{mathrsfs}

%\usepackage[pdftex]{graphicx}
\usepackage{epstopdf}

\newcounter{beans}

\newcommand{\ds}{\displaystyle}
\newcommand{\limit}[2]{\displaystyle\lim_{#1\to#2}}

\newcommand{\binomial}[2]{\ \left( \begin{array}{c}
                                  #1 \\
                                  #2
                                 \end{array}
                            \right) \
                         }
\newcommand{\ExampleRule}[2]
  {
  \noindent
  \rule{\linewidth}{1pt}
  \begin{example}
    #1
    \label{#2}
  \end{example}
  \rule{\linewidth}{1pt}
  \vskip0.125in
  }

\newcommand{\defbox}[1]
  {
   \ \\
   \noindent
   \setlength\fboxrule{1pt}
   \fbox{
        \begin{minipage}{6.5in}
          #1
        \end{minipage}
        }
   \ \\
  }
\newcommand{\verysmallworkbox}[1]
  {
   \ \\
   \noindent
   \setlength\fboxrule{1pt}
   \fbox{
        \begin{minipage}{6.5in}
           #1
           \ \\
           \vskip0.5in \ \\
           \ \\
        \end{minipage}
        }
   \ \\
  }
\newcommand{\smallworkbox}[1]
  {
   \ \\
   \noindent
   \setlength\fboxrule{1pt}
   \fbox{
        \begin{minipage}{6.5in}
           #1
           \ \\
           \vskip2.5in \ \\
           \ \\
        \end{minipage}
        }
   \ \\
  }
\newcommand{\halfworkbox}[1]
  {
   \ \\
   \noindent
   \setlength\fboxrule{1pt}
   \fbox{
        \begin{minipage}{6.5in}
           #1 \hfill
           \ \\
           \vskip3.25in \ \\
           \ \\
        \end{minipage}
        }
   \ \\
  }
\newcommand{\largeworkbox}[1]
  {
   \ \\
   \noindent
   \setlength\fboxrule{1pt}
   \fbox{
        \begin{minipage}{6.5in}
           #1
           \ \\
           \vskip7.5in \ \\
           \ \\
        \end{minipage}
        }
   \ \\
  }
\newcommand{\flexworkbox}[2]
  {
   \ \\
   \noindent
   \setlength\fboxrule{1pt}
   \fbox{
        \begin{minipage}{6.5in}
           #1
           \ \\

           \vskip#2 \ \\
           \ \\
        \end{minipage}
        }
   \ \\
  }


% symbols for sets of numbers

\newcommand{\natnumb}{$\cal N$}
\newcommand{\whonumb}{$\cal W$}
\newcommand{\intnumb}{$\cal Z$}
\newcommand{\ratnumb}{$\cal Q$}
\newcommand{\irrnumb}{$\cal I$}
\newcommand{\realnumb}{$\cal R$}
\newcommand{\cmplxnumb}{$\cal C$}

% misc. commands

\newcommand{\mma}{{\it Mathematica}}
\newcommand{\sech}{\mbox{ sech}}
 
\newtheorem{theorem}{Theorem}
\newtheorem{example}{Example}
\newtheorem{definition}{Definition}
\newtheorem{problem}{Problem}

\setcounter{secnumdepth}{2}
\setcounter{tocdepth}{4}


\begin{document}
\maketitle
\newpage
%%%%%%%%%%%%%%%%%%%%%%%%%%%%%%%%%%%%%%%%%%%%%%%%%%%%%%%%%%%%%%%%%%%%%%%%%%%%%%%%
%%%%%%%%%%%%%%%%%%%%%%%%%%%%%%%%%%%%%%%%%%%%%%%%%%%%%%%%%%%%%%%%%%%%%%%%%%%%%%%%
\vskip0.1in\hrule\vskip0.1in
\noindent
{\bf Convergence Rates for Root Finding Problem: Definitions} 
\vskip0.1in\hrule\vskip0.1in
\noindent
The root finding methods that have been covered produce a possibly infinite
sequence of approximations for a root of a function. For example, in the
Bisection method the approximations can be defined as the midpoint of each
interval. This is the same point that is used to define the new interval. If
we start with the notation
$$
  [ a_0, b_0 ] = [a, b]
$$
the first subdivision of the interval gives
$$
  c = {{a + b}\over 2} = {{a_0 + b_0}\over 2}
$$
Once the midpoint is computed, the new interval, which is half the size of the
previous interval, can be written as $[a_1,b_1]$ where one of the two of
endpoints has been reassigned to be an endpoint. As presented earlier in
course, these computations go on until the interval size is reduced to a value
smaller than a specified tolerance.
%%%%%%%%%%%%%%%%%%%%%%%%%%%%%%%%%%%%%%%%%%%%%%%%%%%%%%%%%%%%%%%%%%%%%%%%%%%%%%%%
%%%%%%%%%%%%%%%%%%%%%%%%%%%%%%%%%%%%%%%%%%%%%%%%%%%%%%%%%%%%%%%%%%%%%%%%%%%%%%%%
We can use the algorithm to define a relationship involving a reduction from one
step to the next. The error in using any point in the original interval as an
approximation is at most the length of the interval. That is,
$$
  | x_0 - x^* | = | e_0 | \leq | b_0 - a_0 | = | b - a |
$$
After one iteration, the error in using any point in the interval, $[a_1, b_1]$,
can be defined in the same way. That is,
$$
  | e_1 | = | x_1 - x^* | \leq | b_1 - a_1 | = {1\over 2} | b_0 - a_0 |
                                       = {1\over 2} e_0
$$
The last equality relates the error to the original interval. After $k$
bisections the error in using an arbitrary point in the interval
$[a_{k+1}, b_{k+1}]$ is given by
$$
  | e_{k+1} | = | x_{k+1} - x^* | \leq | b_{k+1} - a_{k+1} |
                          = {1\over 2} | b_k - a_k | = {1\over 2} | e_k |
$$
This defines a relationship between the error after $k+1$ iterations and $k$
iterations. The final error relationship is $e_{k+1}={1\over 2} e_k$.
%%%%%%%%%%%%%%%%%%%%%%%%%%%%%%%%%%%%%%%%%%%%%%%%%%%%%%%%%%%%%%%%%%%%%%%%%%%%%%%%
%%%%%%%%%%%%%%%%%%%%%%%%%%%%%%%%%%%%%%%%%%%%%%%%%%%%%%%%%%%%%%%%%%%%%%%%%%%%%%%%
In the development of the Bisection method, the relationship above was used to
determine the number of bisections needed to reduce the interval size to a
length that was less than a specified tolerance. We can use the error reduction
relationship to relate the error after $k+1$ iterations and the original
interval size. We can also use the error relationship to define how fast the
convergence will be for a given method. For the Bisection method we can write
$$
  | e_{k+1} | \leq C | e_k |
$$
where $C=1/2$ for the Bisection method. In general, we can write
$$
  | e_{k+1} | \leq C | e_k |^r
$$
where $r$ is called the convergence rate. For the Bisection method, the rate of
convergence is $r=1$. Also, for the Bisection method, there is a linear
relationship between successive errors. This motiviates the following
definition.
%%%%%%%%%%%%%%%%%%%%%%%%%%%%%%%%%%%%%%%%%%%%%%%%%%%%%%%%%%%%%%%%%%%%%%%%%%%%%%%%
%%%%%%%%%%%%%%%%%%%%%%%%%%%%%%%%%%%%%%%%%%%%%%%%%%%%%%%%%%%%%%%%%%%%%%%%%%%%%%%%
\begin{definition}
   Suppose that a sequence of numbers is generated by an iterative algorithm,
   say the Bisection method. If the error in successive approximations satisfies
   the relationship
   $$
     | e_{k+1} | \leq C | e_k |^r
   $$
   where $C$ does not depend on $k$. Then $r$ defines the rate of convergence of
   the method. If $r=1$, the method used is said to be linearly convergence. If
   $r=2$, the method used is said to be quadratically convergence. If $1<r<2$,
   the method is said to be super-linearly convergent.
\end{definition}
%%%%%%%%%%%%%%%%%%%%%%%%%%%%%%%%%%%%%%%%%%%%%%%%%%%%%%%%%%%%%%%%%%%%%%%%%%%%%%%%
%%%%%%%%%%%%%%%%%%%%%%%%%%%%%%%%%%%%%%%%%%%%%%%%%%%%%%%%%%%%%%%%%%%%%%%%%%%%%%%%
Of course, in terms of error convergence, the methods already discussed in the
course can be categorized as follows. (1) functional iteration is at best
linearly convergence depending on the choice of the function in the fixed point
iteration, (2) the Bisection method is linearly convergent, (3) Newton's method
is quadratically convergent, and the Secant method is super-linearly convergent.
%%%%%%%%%%%%%%%%%%%%%%%%%%%%%%%%%%%%%%%%%%%%%%%%%%%%%%%%%%%%%%%%%%%%%%%%%%%%%%%%
%%%%%%%%%%%%%%%%%%%%%%%%%%%%%%%%%%%%%%%%%%%%%%%%%%%%%%%%%%%%%%%%%%%%%%%%%%%%%%%%
As a rule of thumb, it is better to use a quadratically convergent method over
a linear or super-linearly convergent method. To see this, suppose that from one
iteration to the next, the error in the previous approximation is $10^{-1}$ and
we know that the errors are devreasing. Then for a linearly convergent method
the error obeys
$$
  | e_{k+1} | \leq C | e_k | = C\cdot 10^{-1}
$$
where $C$ must be less than one in magnitude. For Bisection, $C={1/2}$. For a
quadratically convergent method, the error relationship is
$$
  | e_{k+1} | \leq C | e_k |^2 = C\cdot (10^{-1})^2 = C\cdot 10^{-2}
$$
After two iterations, the relationships look like
$$
  | e_{k+2} | \leq C | e_{k+1} | \leq C^2 | e_k | = C^2\cdot 10^{-1}
                  \approx C^2 (0.1)
$$
and for quadratic convergence
$$
  | e_{k+2} | \leq C | e_{k+1} |^2 \leq C^2\cdot | e_{k} |^4
               \leq C^2 (10^{-1})^4 \leq C^2 (0.0001)
$$
It is easy to see that after two iterations, the error is being reduced much
faster than a linear method. In some problems, the constant, $C$, may cause a
problem. However, in most problems, this constant is bounded and the error
reduction is dependent on the previous error.
%%%%%%%%%%%%%%%%%%%%%%%%%%%%%%%%%%%%%%%%%%%%%%%%%%%%%%%%%%%%%%%%%%%%%%%%%%%%%%%%
%%%%%%%%%%%%%%%%%%%%%%%%%%%%%%%%%%%%%%%%%%%%%%%%%%%%%%%%%%%%%%%%%%%%%%%%%%%%%%%%















%%%%%%%%%%%%%%%%%%%%%%%%%%%%%%%%%%%%%%%%%%%%%%%%%%%%%%%%%%%%%%%%%%%%%%%%%%%%%%%%
%%%%%%%%%%%%%%%%%%%%%%%%%%%%%%%%%%%%%%%%%%%%%%%%%%%%%%%%%%%%%%%%%%%%%%%%%%%%%%%%
\vskip0.1in\hrule\vskip0.1in
\noindent
The general root finding problem can be written as follows: For a given
real-valued function, $f$, of a single real variable find a real number, $x^*$,
such that
$$
  f(x^*) = 0
$$
There are all kinds of issues that arise in solving these types of problems.
For example, the function may have multiple roots. In searching for a specific
root, we may find other roots that are not of interest. To deal with all of the
issues in this problem, we will develop a number of algorithms that can be used
in a variety of root finding problems.
%%%%%%%%%%%%%%%%%%%%%%%%%%%%%%%%%%%%%%%%%%%%%%%%%%%%%%%%%%%%%%%%%%%%%%%%%%%%%%%%
%%%%%%%%%%%%%%%%%%%%%%%%%%%%%%%%%%%%%%%%%%%%%%%%%%%%%%%%%%%%%%%%%%%%%%%%%%%%%%%%
\vskip0.1in\hrule\vskip0.1in
\noindent
{\bf Root Finding Problems: Using Fixed Point Iteration} 
\vskip0.1in\hrule\vskip0.1in
\noindent
As a first attempt at determining the location of a root for a function, we
might consider a modification of the root finding problem as follows. Given a
function, $f$, we can rewrite the equation
$$
  f(x^*) = 0
$$
as
$$
  x = x - f(x^*) = g(x^*)
$$
The resulting equation is called a fixed point equation and the equation
suggests an iteration of the form
$$
  x_1 = g(x_0), x_2 = f(x_1), \ldots
$$
where $x_0$ must be supplied to start the iteration. This iteration formula will
produce a sequence of real numbers. The hope is that the sequence will converge
to a solution of the fixed point equation and also a solution of the root
finding problem.
%%%%%%%%%%%%%%%%%%%%%%%%%%%%%%%%%%%%%%%%%%%%%%%%%%%%%%%%%%%%%%%%%%%%%%%%%%%%%%%%
%%%%%%%%%%%%%%%%%%%%%%%%%%%%%%%%%%%%%%%%%%%%%%%%%%%%%%%%%%%%%%%%%%%%%%%%%%%%%%%%
\vskip0.1in\hrule\vskip0.1in
\noindent
{\bf Root Finding Problems: Coding Fixed Point Iteration} 
\vskip0.1in\hrule\vskip0.1in
\noindent
One can easily write a routine or computer code that implements fixed point
iteration. The following code provides a template of how a reusable routine
might be written:
\vskip0.1in\hrule\vskip0.1in
\begin{verbatim}
     //
     // Author: Joe Koebbe
     //
     // Routine Name:         fproot
     // Programming Language: Java
     // Last Modified:        09/10/19
     //
     // Description/Purpose: The routine will generate a sequence of numbers
     // using fixed point iteration.
     //
     // Input:
     //
     // FunctionObject f - the function defined in the root finding problem
     // double x0 - the initial guess at the location of a fixed point
     // double tol - the error tolerance allowed in the approximation of the
     //              root finding problem
     // int maxit - the maximum number of iterations allowed in the fixed point
     //             iteration.
     //
     // Output:
     //
     // double x1 - the last number in the finite sequence that is an
     //             approximation in the root finding problem
     //
     public double fproot(FunctionObject f, double x0, double tol, int maxit) {
       //
       // initialize the error in the routine so that the iteration loop will be
       // executed at least one time
       // --------------------------
       //
       double error = 10.0 * tol;
       //
       // initialize a counter for the number of iterations
       // -------------------------------------------------
       //
       int iter = 0;
       //
       // loop over the fixed point iterations as long as the error is larger
       // than the tolerance and the number of iterations is less than the
       // maximum number allowed
       // ----------------------
       //
       while(error > tol && iter < maxit) {
         //
         // update the number of iterations performed
         // -----------------------------------------
         //
         iter++;
         //
         // compute the next approximation
         // ------------------------------
         //
         double x1 = x0 - f(x0);
         //
         // compute the error using the difference between the iterates in the
         // loop
         // ----
         //
         error = Math.abs(x1 - x0);
         //
         // reset the input value to be the new approximation
         // --------------------------------------------------
         //
         x0 = x1;
         //
       }
       //
       // return the last value computed
       // -------------------------------
       //
       return x1;
       //
     }

\end{verbatim}
\vskip0.1in\hrule\vskip0.1in
\noindent
There are a couple of features in the code that need to be explained.
\begin{list}{$\bullet$}{\usecounter{beans} \parsep=0pt \listparindent=0pt
\topsep=0pt \rightmargin=.35in \leftmargin=.35in \labelsep=5 pt
\itemsep=2pt}
  \item To make this work in the Java programming language, the method would
        need to be embedded in a class. That is, the code is not a standalone
        code.
  \item The first argument is an Java Object that needs to be created. The
        object is used to provide the function evaluation for any real input.
  \item The second argument is the initial guess at the solution of the problem.
  \item Since we know we are going to end up with at best an approximation of
        a root, the third argument in the function is an error tolerance that is
        acceptable to the calling routine.
  \item The final argument passed in limits the number of iterations allowed in
        the method. Note that if you are not careful, an infinite loop might be
        created due to the approximations used everywhere.
\end{list}
If we apply the code to any problem, we are assuming that the solution will pop
out the end. There is no guarantee that this is the case. It is important to
establish conditions that will guarantee the code will produce an approximate
solution of the fixed point problem and thus provide a root for the original
function, $f$.
%%%%%%%%%%%%%%%%%%%%%%%%%%%%%%%%%%%%%%%%%%%%%%%%%%%%%%%%%%%%%%%%%%%%%%%%%%%%%%%%
%%%%%%%%%%%%%%%%%%%%%%%%%%%%%%%%%%%%%%%%%%%%%%%%%%%%%%%%%%%%%%%%%%%%%%%%%%%%%%%%
\vskip0.1in\hrule\vskip0.1in
\noindent
{\bf Root Finding Problems: Analysis of Functional Iteration Using Taylor
 Series Expansion} 
\vskip0.1in\hrule\vskip0.1in
\noindent
The general iteration formula, given $x_0$, is the following.
$$
  x_{k+1} = g(x_k)
$$
for $k=0,1,2,\ldots$. We also know that for the fixed point problem, the
solution satisfies the equation
$$
  x^* = g(x^*)
$$
Subtracting the two equations gives
$$
  x_{k+1} - x^* = g(x_k) - g(x^*)
$$
The Taylor expansion of $g(x_k)$ about the solution $x^*$ is given by
$$
  g(x_k) = g(x*) + g'(x^*) ( x_k - x^* ) + {1\over 2} g''(x^*) ( x_k - x^* )^2
              + \ldots
$$
Substituting the expansion into the equation above and truncating the series
gives
$$
  x_{k+1} - x^* \approx g(x*) + g'(x^*) ( x_k - x^* ) - g(x^*)
                       = g'(x^*) ( x_k - x^* )
$$
Taking absolute values the last equation can be written as
$$
  | x_{k+1} - x^* | \leq | g'(x^*) | \cdot | x_k - x^* |
$$
One can read the previous expression as the difference (or error) in $x_{k+1}$
is less than the magnitude of the derivative of the fixed point iteration 
function, $g$, times the difference (or error) in the previous approximation,
$x_k$. Using
$$
  e_{k} = | x_k - x^* |
$$
allows use to relate the error at successive steps as
$$
  e_{k+1} \leq | g'(x^*) | \cdot | e_{k+1} |
$$
To get convergence to the fixed point (or root) we would like the error to be
reduced at each step. This requires the condition
$$
  | g'(x^*) | < 1
$$
For the general fixed point problem, this condition is required for convergence
to the fixed point, $x^*$, or solution of the root finding problem. Note that
this is a significant drawback of fixed point iteration as a means of solving
root finding problems.
%%%%%%%%%%%%%%%%%%%%%%%%%%%%%%%%%%%%%%%%%%%%%%%%%%%%%%%%%%%%%%%%%%%%%%%%%%%%%%%%
%%%%%%%%%%%%%%%%%%%%%%%%%%%%%%%%%%%%%%%%%%%%%%%%%%%%%%%%%%%%%%%%%%%%%%%%%%%%%%%%
\vskip0.1in\hrule\vskip0.1in
\noindent
{\bf Root Finding Problems: An Example Using Functional Iteration} 
\vskip0.1in\hrule\vskip0.1in
\noindent
Suppose that we are interested in computing the roots of
$$
  f(x) = e^x - \pi
$$
Analytically we can compute the solution by solving for $x$ in the equation
$$
  e^x - \pi = 0
$$
The value is $x=ln(\pi)\approx 1.144729886$. This is a very simple problem.
However, it is always a good idea to test general methods on simple problems
while developing algorithms and coding these up for use on real problems.

Let's apply functional iteration to this root finding problem. First, we will
need to create an associated function that defines a fixed point problem. One
possibility is to choose
$$
  g_1(x) = x - f(x) = x - ( e^x - \pi ) = x - e^x + \pi
$$
Let's check the condition for convergence by computing the derivative of $g$
near at the solution above.
$$
  g_1'(x) = 1 - e^x = 1 - \pi \approx -2.14159245 \rightarrow 
  | g_1'(x) | \approx 2.14159245
$$
The value is bigger than 1 which means the sequence of iterates is not going to
converge. So, the choice of $g(x)$ will not work.

As another option, consider a modification of the function. If
$$
  f(x) = e^x - \pi = 0
$$
then
$$
  f(x) = {1\over 5} ( e^x - \pi ) = 0
$$
which allows us to write
$$
  g_2(x) = x - {1\over 5} ( e^x - \pi )
$$
with derivative
$$
  g_2'(x) = 1 - {1\over 5} e^x 
$$
and near the solution
$$
  |g_2'(x)| = | 1 - {1\over 5} \pi | < 1.0
$$
So, we can expect better results in this case.

For the two examples, the output for the two choices of the iteration function
$g_1(x)$ or $g_2(x)$.
\vskip0.1in\hrule\vskip0.1in
\begin{table}
\caption{Results for Functional Iteration for Two Different Iteration Functions}
  \vskip0.1in
  \begin{center}
  \begin{tabular}{|c||c|c||c|c|}
    \hline
    Iteration No. & $g_2(x)=x-(e^x-\pi)$ & error
                              & $g_1(x)=x-(e^x-\pi)$ & error \\
    \hline
        01 &  1.08466220  &  8.46621990E-02  & 1.42331100  & 0.423310995 \\
    \hline
        02 &  1.12129271  &  3.66305113E-02  & 0.41406250  & 1.00924850 \\
    \hline
        03 &  1.13584745  &  1.45547390E-02  & 2.04270363  & 1.62864113 \\
    \hline
        04 &  1.14140379  &  5.55634499E-03  &-2.52713394  & 4.56983757 \\
    \hline
        05 &  1.14349020  &  2.08640099E-03  & 0.53457117  & 3.06170511 \\
    \hline
        06 &  1.14426863  &  7.78436661E-04  & 1.96944773  & 1.43487656 \\
    \hline
        07 &  1.14455843  &  2.89797783E-04  &-2.05567694  & 4.02512455 \\
    \hline
        08 &  1.14466619  &  1.07765198E-04  & 0.95790958  & 3.01358652 \\
    \hline
        09 &  1.14470625  &  4.00543213E-05  & 1.49325967  & 0.535350084 \\
    \hline
        10 &  1.14472115  &  1.49011612E-05  & 0.18326997  & 1.30998969 \\
    \hline
        11 &  1.14472663  &  5.48362732E-06  & 2.12372398  & 1.94045401 \\
    \hline
  \end{tabular}
  \end{center}
\end{table}
\vskip0.1in\hrule\vskip0.1in
So, two completely different results are obtained. One converges with a slight
modification to the first. The first function produces a sequence that does not
converge and the second produces the correct result up to machine precision.
That is, $x^*=1.14472663$ with absolute error $5.48362732E-06$. This is one of
the reasons why functional iteration is not used as much. The problem is that
there are infinitely many choices for the fixed point equation. Some will
provide convergence and others will not come close.
%%%%%%%%%%%%%%%%%%%%%%%%%%%%%%%%%%%%%%%%%%%%%%%%%%%%%%%%%%%%%%%%%%%%%%%%%%%%%%%%
%%%%%%%%%%%%%%%%%%%%%%%%%%%%%%%%%%%%%%%%%%%%%%%%%%%%%%%%%%%%%%%%%%%%%%%%%%%%%%%%
\vskip0.1in\hrule\vskip0.1in
\noindent
{\bf Root Finding Problems: Convergence of Functional Iteration}
\vskip0.1in\hrule\vskip0.1in
\noindent
If we end up using functional iteration, it will also pay to know how fast the
sequence converges. Fewer iterations means faster results with few computations.
The convergence of the sequence is determined by the same calculations as in
the convergence justification above.
$$
  | x_{k+1} - x^* | \leq | g'(x^*) | \cdot| x_k - x^* |
$$
For functional iteration the convergence rate is defined by
$$
  \makebox{rate of convergence} = | g'(x^*) | 
$$
The smaller the magnitude of the derivative, $|g'(x^*)|$, the faster the
convergence will be.

As an example, consider changing the parameter ${1\over 5}$ used to modify the
iteration function in the previous section. If the parameter is decreased, what
happens to the convergence? This is covered in the homework tasks.
%%%%%%%%%%%%%%%%%%%%%%%%%%%%%%%%%%%%%%%%%%%%%%%%%%%%%%%%%%%%%%%%%%%%%%%%%%%%%%%%
%%%%%%%%%%%%%%%%%%%%%%%%%%%%%%%%%%%%%%%%%%%%%%%%%%%%%%%%%%%%%%%%%%%%%%%%%%%%%%%%
\vskip0.1in\hrule\vskip0.1in
\noindent
{\bf Root Finding Problems: Continuous Functions and the Bisection Method}
\vskip0.1in\hrule\vskip0.1in
\noindent
Functional iteration is limited in applicability in the real world. So, we move
on to the next algorithm for the root finding problem. In this section, we will
assume that the function, $f$, is continuous on a closed and bounded interval
$[a,b]$. The main mathematical tool used in this case is the Intermediate Value
Theorem for continuous functions.
\vskip0.1in\hrule\vskip0.1in
{\bf Theorem:} Suppose the function, $f$, is continuous on the closed and
boundaed interval $[a, b]$. If $M$ is any value between $f(a)$ and $f(b)$ then
there exists a value $c\in(a,b)$ such that $f(c)=M$,
\vskip0.1in\hrule\vskip0.1in
\noindent
Now, if $f(a)\geq 0\geq f(b)$ (or vice-versa) then there is at least one value,
$c$ in the interval $(a,b)$ such that $f(c)=0$. If we determine end-points of
an interval such that $f(a)\leq 0$ {\bf and} $0\leq f(b)$ (or vice versa), we
know there is also a root of the function on the interval we have determined.  
There is a simple condition that can be test to verify an interval will work.
That is,
$$
   f(a)\cdot f(b) < 0
$$
This is enough to determine that the interval straddles the horizontal axis.
%%%%%%%%%%%%%%%%%%%%%%%%%%%%%%%%%%%%%%%%%%%%%%%%%%%%%%%%%%%%%%%%%%%%%%%%%%%%%%%%
%%%%%%%%%%%%%%%%%%%%%%%%%%%%%%%%%%%%%%%%%%%%%%%%%%%%%%%%%%%%%%%%%%%%%%%%%%%%%%%%
\vskip0.1in\hrule\vskip0.1in
\noindent
{\bf Root Finding Problems: Bisection and Convergence}
\vskip0.1in\hrule\vskip0.1in
\noindent
If the original interval $[a, b]$ is bisected into two equal subintervals
$$
  [a, b] = [a, c] \cup [c, b]
$$
where
$$
  c = {{a+b}\over 2}
$$
Since there is at least one root on the interval there are three possibilities
\begin{list}{$\bullet$}{\usecounter{beans} \parsep=0pt \listparindent=0pt
\topsep=0pt \rightmargin=.35in \leftmargin=.35in \labelsep=5 pt
\itemsep=2pt}
  \item $f(c) = 0$,
  \item $f(a)\cdot f(c)<0$ which implies there is a root in $[a,c]$, or
  \item $f(c)\cdot f(b)<0$ which implies there is a root in $[c,b]$.
\end{list}
If the first condition is true, twe have the root, $x^*=c$ and we are done
searching. In the second case, we redefine the search interval to $[a,c]$ and
in the third case, the search interval is redefined to be $[c,b]$. Once we have
redefined the interval, we repeat the bisection on the new interval. The
bisection will reduce the size of the search interval each time through. We just
need to translate this into code.
%%%%%%%%%%%%%%%%%%%%%%%%%%%%%%%%%%%%%%%%%%%%%%%%%%%%%%%%%%%%%%%%%%%%%%%%%%%%%%%%
%%%%%%%%%%%%%%%%%%%%%%%%%%%%%%%%%%%%%%%%%%%%%%%%%%%%%%%%%%%%%%%%%%%%%%%%%%%%%%%%
\vskip0.1in\hrule\vskip0.1in
\noindent
{\bf Root Finding Problems: Bisection and Convergence}
\vskip0.1in\hrule\vskip0.1in
\noindent
The following routine, written in something like C will implement the Bisection
Method.
\vskip0.1in\hrule\vskip0.1in
\begin{verbatim}

double bisectionMethod(typedef'd f, double a, double b, double tol, int maxiter)
{
  // set up some parameters
  double c;
  double error;
  int iter;
 
  // check the endpoints
  if(f(a)==0) return a;
  if(f(b)==0) return b;

  // check for a root in the interval
  if(f(a)*f(b) >= 0.0) throw an error or print a message

  //this part of the code will perform the iterations
  error = 10.0 * tol;
  iter = 0;
  while(error > tol && iter < maxiter) {
    iter++;
    c = 0.5 * ( a + b );

    // compute the sign change value
    double val = f(a) * f(c);

    // reassign the end point based on the location of the root
    if(val<0.0) {
      b = c;
    } else {
      a = c;
    }

    // compute the error in the approximation - this assumes a<b
    error = b - a

  }
  
  // return the midpoint as it is more accurate
  return c;
  
}

\end{verbatim}
\vskip0.1in\hrule\vskip0.1in
\noindent
It should be noted that once an interval has been determined for which the
function value changes sign, the Bisection Method will continue until a root is
found, at least up to machine precision. We can take advantage of this property
to redesign the algorithm to take a specific number of iterations instead of
checking the error.
%%%%%%%%%%%%%%%%%%%%%%%%%%%%%%%%%%%%%%%%%%%%%%%%%%%%%%%%%%%%%%%%%%%%%%%%%%%%%%%%
%%%%%%%%%%%%%%%%%%%%%%%%%%%%%%%%%%%%%%%%%%%%%%%%%%%%%%%%%%%%%%%%%%%%%%%%%%%%%%%%
\vskip0.1in\hrule\vskip0.1in
\noindent
{\bf Root Finding Problems: The Bisection Method and Error Reduction}
\vskip0.1in\hrule\vskip0.1in
\noindent
The fact the the interval size is being reduced in each iteration of bisection
can be used as follows. The length of the original interval can be used to
bound the error in any approximation of a root. That is,
$$
  | x - x^* | \leq | b - a |
$$
A sequence of intervals is created by the Bisection method that we can write as
$[a_i, b_i]$, for $i=0,1,\ldots$ where each new interval shares an endpoint from
the previous interval and the other end point is the midpoint of the previous
interval. Note that $[a_0, b_0]=[a,b]$ in this argument. So, we can write the
following set of inequalities
$$
  | x - x^* | < b_k - a_k < {1\over 2} ( b_{k-1} - a_{k-1} ) < \cdots
                    < {1\over{2^k}} ( b_0 - a_0 ) = 2^{-k} ( b - a )  
$$ 
This means that once the interval $[a, b]$ has been determined, the amount of
error in the approximation is computable at each iteration.

Suppose that we specify an error tolerance that is acceptable, say $10^{-d}$
where $d$ is the number of digits of accuracy. Then define the number of
iterations to reduce the error to the desired tolerance
$$
  2^{-k} ( b - a ) < 10^{-d}
$$
Using a bit of algebra
$$
  2^{-k} < {{10^{-d}}\over{(b-a)}}
   \rightarrow -k < log_2\left( {{10^{-d}}\over{(b-a)}}\right)
$$
or using the negattion of the inequality
$$
  - log_2\left( {{10^{-d}}\over{(b-a)}}\right) < k 
$$
This gives us the total number of iterations needed to reduce the error to the
desired tolerance. So, we can rewrite the code.
%%%%%%%%%%%%%%%%%%%%%%%%%%%%%%%%%%%%%%%%%%%%%%%%%%%%%%%%%%%%%%%%%%%%%%%%%%%%%%%%
%%%%%%%%%%%%%%%%%%%%%%%%%%%%%%%%%%%%%%%%%%%%%%%%%%%%%%%%%%%%%%%%%%%%%%%%%%%%%%%%
\vskip0.1in\hrule\vskip0.1in
\noindent
{\bf Root Finding Problems: An Alternative Bisection Method Code}
\vskip0.1in\hrule\vskip0.1in
\noindent
The alternative code is the following.
%%%%%%%%%%%%%%%%%%%%%%%%%%%%%%%%%%%%%%%%%%%%%%%%%%%%%%%%%%%%%%%%%%%%%%%%%%%%%%%%
%%%%%%%%%%%%%%%%%%%%%%%%%%%%%%%%%%%%%%%%%%%%%%%%%%%%%%%%%%%%%%%%%%%%%%%%%%%%%%%%
\vskip0.1in\hrule\vskip0.1in
\begin{verbatim}

double bisectionMethod(typedef'd f, double a, double b, double tol) {
  // set up some parameters
  double c;
  double error;
 
  // check the endpoints
  if(f(a)==0) return a;
  if(f(b)==0) return b;

  // check for a root in the interval
  if(f(a)*f(b) >= 0.0) throw an error or print a message

  //this part of the code will perform the iterations
  maxiter = - 2.0 * log2( tol / ( b - a ) );
  for(int i=0; i<maxiter; i++) {
    c = 0.5 * ( a + b );
    // compute the sign change value
    double val = f(a) * f(c);
    // reassign the end point based on the location of the root
    if(val<0.0) {
      b = c;
    } else {
      a = c;
    }
  }
  
  // return the midpoint as it is more accurate
  return c;
  
}

\end{verbatim}
\vskip0.1in\hrule\vskip0.1in
\noindent
Note that the output value will be an approximation of a root in the original
interval, $[a,b]$ that satisfies the desired tolerance.
%%%%%%%%%%%%%%%%%%%%%%%%%%%%%%%%%%%%%%%%%%%%%%%%%%%%%%%%%%%%%%%%%%%%%%%%%%%%%%%%
%%%%%%%%%%%%%%%%%%%%%%%%%%%%%%%%%%%%%%%%%%%%%%%%%%%%%%%%%%%%%%%%%%%%%%%%%%%%%%%%
\vskip0.1in\hrule\vskip0.1in
\noindent
{\bf Root Finding Problems: Bisection Method Examples}
\vskip0.1in\hrule\vskip0.1in
\noindent
It is always a good idea to test the code you write. Using the example from our
tests of functional iteration we can detemine whether or not the Bisection 
Method works and how this compares with the fixed point approach.
%%%%%%%%%%%%%%%%%%%%%%%%%%%%%%%%%%%%%%%%%%%%%%%%%%%%%%%%%%%%%%%%%%%%%%%%%%%%%%%%
%%%%%%%%%%%%%%%%%%%%%%%%%%%%%%%%%%%%%%%%%%%%%%%%%%%%%%%%%%%%%%%%%%%%%%%%%%%%%%%%
\end{document}
