\documentclass[10pt,fleqn]{article}
%\usepackage{graphicx}


\setlength{\topmargin}{-.75in}
\addtolength{\textheight}{2.00in}
\setlength{\oddsidemargin}{.00in}
\addtolength{\textwidth}{.75in}

\title{Math 4610 Lecture Notes \\
            \ \\
            Machine Epsilon Code 
  \footnote{These notes are part of an Open Resource Educational project
            sponsored by Utah State University}}

\author{Joe Koebbe}

\nofiles

\pagestyle{empty}

\setlength{\parindent}{0in}

% new math commands


\setlength{\oddsidemargin}{-0.25in}
\setlength{\evensidemargin}{-0.25in}
\setlength{\textwidth}{6.75in}
\setlength{\headheight}{0.0in}
\setlength{\topmargin}{-0.25in}
\setlength{\textheight}{9.00in}

\makeindex

\usepackage{mathrsfs}

%\usepackage[pdftex]{graphicx}
\usepackage{epstopdf}

\newcounter{beans}

\newcommand{\ds}{\displaystyle}
\newcommand{\limit}[2]{\displaystyle\lim_{#1\to#2}}

\newcommand{\binomial}[2]{\ \left( \begin{array}{c}
                                  #1 \\
                                  #2
                                 \end{array}
                            \right) \
                         }
\newcommand{\ExampleRule}[2]
  {
  \noindent
  \rule{\linewidth}{1pt}
  \begin{example}
    #1
    \label{#2}
  \end{example}
  \rule{\linewidth}{1pt}
  \vskip0.125in
  }

\newcommand{\defbox}[1]
  {
   \ \\
   \noindent
   \setlength\fboxrule{1pt}
   \fbox{
        \begin{minipage}{6.5in}
          #1
        \end{minipage}
        }
   \ \\
  }
\newcommand{\verysmallworkbox}[1]
  {
   \ \\
   \noindent
   \setlength\fboxrule{1pt}
   \fbox{
        \begin{minipage}{6.5in}
           #1
           \ \\
           \vskip0.5in \ \\
           \ \\
        \end{minipage}
        }
   \ \\
  }
\newcommand{\smallworkbox}[1]
  {
   \ \\
   \noindent
   \setlength\fboxrule{1pt}
   \fbox{
        \begin{minipage}{6.5in}
           #1
           \ \\
           \vskip2.5in \ \\
           \ \\
        \end{minipage}
        }
   \ \\
  }
\newcommand{\halfworkbox}[1]
  {
   \ \\
   \noindent
   \setlength\fboxrule{1pt}
   \fbox{
        \begin{minipage}{6.5in}
           #1 \hfill
           \ \\
           \vskip3.25in \ \\
           \ \\
        \end{minipage}
        }
   \ \\
  }
\newcommand{\largeworkbox}[1]
  {
   \ \\
   \noindent
   \setlength\fboxrule{1pt}
   \fbox{
        \begin{minipage}{6.5in}
           #1
           \ \\
           \vskip7.5in \ \\
           \ \\
        \end{minipage}
        }
   \ \\
  }
\newcommand{\flexworkbox}[2]
  {
   \ \\
   \noindent
   \setlength\fboxrule{1pt}
   \fbox{
        \begin{minipage}{6.5in}
           #1
           \ \\

           \vskip#2 \ \\
           \ \\
        \end{minipage}
        }
   \ \\
  }


% symbols for sets of numbers

\newcommand{\natnumb}{$\cal N$}
\newcommand{\whonumb}{$\cal W$}
\newcommand{\intnumb}{$\cal Z$}
\newcommand{\ratnumb}{$\cal Q$}
\newcommand{\irrnumb}{$\cal I$}
\newcommand{\realnumb}{$\cal R$}
\newcommand{\cmplxnumb}{$\cal C$}

% misc. commands

\newcommand{\mma}{{\it Mathematica}}
\newcommand{\sech}{\mbox{ sech}}
 
\newtheorem{theorem}{Theorem}
\newtheorem{example}{Example}
\newtheorem{definition}{Definition}
\newtheorem{problem}{Problem}

\setcounter{secnumdepth}{2}
\setcounter{tocdepth}{4}


\begin{document}
\maketitle
\newpage
%%%%%%%%%%%%%%%%%%%%%%%%%%%%%%%%%%%%%%%%%%%%%%%%%%%%%%%%%%%%%%%%%%%%%%%%%%%%%%%%
%%%%%%%%%%%%%%%%%%%%%%%%%%%%%%%%%%%%%%%%%%%%%%%%%%%%%%%%%%%%%%%%%%%%%%%%%%%%%%%%
\vskip0.1in\hrule\vskip0.1in
\noindent
{\bf Math 4610: Development of a Test for Machine Precision.}
\vskip0.1in\hrule\vskip0.1in
\noindent
To determine machine precision for your computer we need a strategy that will
display accuracy. We know that the number of digits kept in a machine precision
number is finite. In addition, we know that the numbers are stored in a binary
number system. As scientists, engineers, and computer scientists the results
that we want to see are decimal (base 10) numbers. Our goal should be to
determine the resolution of our numbers in base 10.

One way to proceed would be to try to compare a sequence of numbers converging
to zero until the machine numbers are internally no different than zero. The
problem with this strategy is that we will need to compute the difference
between a term in the sequence and the number zero. That is,
\begin{verbatim}
 
    error = number - zero

\end{verbatim}
The trouble with this approach is that in the difference of the two numbers
that are close to zero may cancel and in floating point arithmetic, there is
no guarantee as to what values will be returned due to the rounding algorithm
used by any computer. That is, if digits that are beyond the finite
representation of the computer numbers are included in the computation, there is
no guarantee the values returned. This is the old computer programming adage of
garbage in equals garbage out.

A more stable approach is to compare a sequence of numbers converging to one
meaning consider
\begin{verbatim}

     error = 1 - ( 1 + epslion )

\end{verbatim}
where epsilon is represents a decreasing sequence of values. Note also that the
number one is represented exactly in binary arithmetic. Notice that there are
parentheses around the last two terms. This will force a code to evaluate the
term in parentheses first before computing the difference. 

So, let's write a code that will compute this difference as a parameter,
$\epsilon$, is made closer and closer to zero.
\begin{verbatim}

     // intialize the constant one and the small value

     one = 1.0;
     eps = 1.0

     // loop over the computing the difference between 1 and 1 plus a bit

     for i=1:20
       diff = one - ( 1 + eps );
       if(diff == 0) {
       eps = 0.5 * eps;
     end

\end{verbatim}




%%%%%%%%%%%%%%%%%%%%%%%%%%%%%%%%%%%%%%%%%%%%%%%%%%%%%%%%%%%%%%%%%%%%%%%%%%%%%%%%
%%%%%%%%%%%%%%%%%%%%%%%%%%%%%%%%%%%%%%%%%%%%%%%%%%%%%%%%%%%%%%%%%%%%%%%%%%%%%%%%
\end{document}














**Routine Name:**           smaceps

**Author:** Joe Koebbe

**Language:** Fortran. The code can be compiled using the GNU Fortran compiler (gfortran).

For example,

    gfortran smaceps.f

will produce an executable **./a.exe** than can be executed. If you want a different name, the following will work a bit
better

    gfortran -o smaceps smaceps.f

**Description/Purpose:** This routine will compute the single precision value for the machine epsilon or the number of digits
in the representation of real numbers in single precision. This is a routine for analyzing the behavior of any computer. This
usually will need to be run one time for each computer.

**Input:** There are no inputs needed in this case. Even though there are arguments supplied, the real purpose is to
return values in those variables.

**Output:** This routine returns a single precision value for the number of decimal digits that can be represented on the
computer being queried.

**Usage/Example:**

The routine has two arguments needed to return the values of the precision in terms of the smallest number that can be
represented. Since the code is written in terms of a Fortran subroutine, the values of the machine machine epsilon and
the power of two that gives the machine epsilon. Due to implicit Fortran typing, the first argument is a single precision
value and the second is an integer.

      call smaceps(sval, ipow)
      print *, ipow, sval

Output from the lines above:

      24   5.96046448E-08

The first value (24) is the number of binary digits that define the machine epsilon and the second is related to the
decimal version of the same value. The number of decimal digits that can be represented is roughly eight (E-08 on the
end of the second value).

**Implementation/Code:** The following is the code for smaceps()

      subroutine smaceps(seps, ipow)
    c
    c set up storage for the algorithm
    c --------------------------------
    c
          real seps, one, appone
    c
    c initialize variables to compute the machine value near 1.0
    c ----------------------------------------------------------
    c
          one = 1.0
          seps = 1.0
          appone = one + seps
    c
    c loop, dividing by 2 each time to determine when the difference between one and
    c the approximation is zero in single precision
    c --------------------------------------------- 
    c
          ipow = 0
          do 1 i=1,1000
             ipow = ipow + 1
    c
    c update the perturbation and compute the approximation to one
    c ------------------------------------------------------------
    c
            seps = seps / 2
            appone = one + seps
    c
    c do the comparison and if small enough, break out of the loop and return
    c control to the calling code
    c ---------------------------
    c
            if(abs(appone-one) .eq. 0.0) return
    c
        1 continue
    c
    c if the code gets to this point, there is a bit of trouble
    c ---------------------------------------------------------
    c
          print *,"The loop limit has been exceeded"
    c
    c done
    c ----
    c
          return
    end

**Last Modified:** September/2017
