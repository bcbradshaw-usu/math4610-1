\documentclass[10pt,fleqn]{article}
%\usepackage{graphicx}


\setlength{\topmargin}{-.75in}
\addtolength{\textheight}{2.00in}
\setlength{\oddsidemargin}{.00in}
\addtolength{\textwidth}{.75in}

\title{Math 4610 Lecture Notes \\
            \ \\
      Measurement of Error
  \footnote{These notes are part of an Open Resource Educational project
            sponsored by Utah State University}}

\author{Joe Koebbe}

\nofiles

\pagestyle{empty}

\setlength{\parindent}{0in}

% new math commands


\setlength{\oddsidemargin}{-0.25in}
\setlength{\evensidemargin}{-0.25in}
\setlength{\textwidth}{6.75in}
\setlength{\headheight}{0.0in}
\setlength{\topmargin}{-0.25in}
\setlength{\textheight}{9.00in}

\makeindex

\usepackage{mathrsfs}

%\usepackage[pdftex]{graphicx}
\usepackage{epstopdf}

\newcounter{beans}

\newcommand{\ds}{\displaystyle}
\newcommand{\limit}[2]{\displaystyle\lim_{#1\to#2}}

\newcommand{\binomial}[2]{\ \left( \begin{array}{c}
                                  #1 \\
                                  #2
                                 \end{array}
                            \right) \
                         }
\newcommand{\ExampleRule}[2]
  {
  \noindent
  \rule{\linewidth}{1pt}
  \begin{example}
    #1
    \label{#2}
  \end{example}
  \rule{\linewidth}{1pt}
  \vskip0.125in
  }

\newcommand{\defbox}[1]
  {
   \ \\
   \noindent
   \setlength\fboxrule{1pt}
   \fbox{
        \begin{minipage}{6.5in}
          #1
        \end{minipage}
        }
   \ \\
  }
\newcommand{\verysmallworkbox}[1]
  {
   \ \\
   \noindent
   \setlength\fboxrule{1pt}
   \fbox{
        \begin{minipage}{6.5in}
           #1
           \ \\
           \vskip0.5in \ \\
           \ \\
        \end{minipage}
        }
   \ \\
  }
\newcommand{\smallworkbox}[1]
  {
   \ \\
   \noindent
   \setlength\fboxrule{1pt}
   \fbox{
        \begin{minipage}{6.5in}
           #1
           \ \\
           \vskip2.5in \ \\
           \ \\
        \end{minipage}
        }
   \ \\
  }
\newcommand{\halfworkbox}[1]
  {
   \ \\
   \noindent
   \setlength\fboxrule{1pt}
   \fbox{
        \begin{minipage}{6.5in}
           #1 \hfill
           \ \\
           \vskip3.25in \ \\
           \ \\
        \end{minipage}
        }
   \ \\
  }
\newcommand{\largeworkbox}[1]
  {
   \ \\
   \noindent
   \setlength\fboxrule{1pt}
   \fbox{
        \begin{minipage}{6.5in}
           #1
           \ \\
           \vskip7.5in \ \\
           \ \\
        \end{minipage}
        }
   \ \\
  }
\newcommand{\flexworkbox}[2]
  {
   \ \\
   \noindent
   \setlength\fboxrule{1pt}
   \fbox{
        \begin{minipage}{6.5in}
           #1
           \ \\

           \vskip#2 \ \\
           \ \\
        \end{minipage}
        }
   \ \\
  }


% symbols for sets of numbers

\newcommand{\natnumb}{$\cal N$}
\newcommand{\whonumb}{$\cal W$}
\newcommand{\intnumb}{$\cal Z$}
\newcommand{\ratnumb}{$\cal Q$}
\newcommand{\irrnumb}{$\cal I$}
\newcommand{\realnumb}{$\cal R$}
\newcommand{\cmplxnumb}{$\cal C$}

% misc. commands

\newcommand{\mma}{{\it Mathematica}}
\newcommand{\sech}{\mbox{ sech}}
 
\newtheorem{theorem}{Theorem}
\newtheorem{example}{Example}
\newtheorem{definition}{Definition}
\newtheorem{problem}{Problem}

\setcounter{secnumdepth}{2}
\setcounter{tocdepth}{4}


\begin{document}
\maketitle
\newpage
%%%%%%%%%%%%%%%%%%%%%%%%%%%%%%%%%%%%%%%%%%%%%%%%%%%%%%%%%%%%%%%%%%%%%%%%%%%%%%%%
%%%%%%%%%%%%%%%%%%%%%%%%%%%%%%%%%%%%%%%%%%%%%%%%%%%%%%%%%%%%%%%%%%%%%%%%%%%%%%%%
\vskip0.1in\hrule\vskip0.1in
\noindent
{\bf Absolute and Relative Error: Introduction.} 
\vskip0.1in\hrule\vskip0.1in
\noindent
In most cases, the solution of a mathematical problem can only be approximated.
So, it would be a good idea to have a way to measure the error between the exact
solution and the approximation of that solution. We will define two types of
error. These are absolute error and relative error. The {\bf absolute error} is
the absolute value of the difference between the approximation and the exact
value for the solution. That is, if $x^*$ is the exact value approximated by
$x$, then
$$ e_{abs} = | x - x^{*} | $$
defines the absolute error. The {\bf relative error} is a scaled error defined
by
$$ e_{rel} = {{| x - x^{*} |}\over{|x^*|}} $$
So, the relative error is a scaled or percent error based on the magnitude of
the exact value.
%%%%%%%%%%%%%%%%%%%%%%%%%%%%%%%%%%%%%%%%%%%%%%%%%%%%%%%%%%%%%%%%%%%%%%%%%%%%%%%%
%%%%%%%%%%%%%%%%%%%%%%%%%%%%%%%%%%%%%%%%%%%%%%%%%%%%%%%%%%%%%%%%%%%%%%%%%%%%%%%%

%%%%%%%%%%%%%%%%%%%%%%%%%%%%%%%%%%%%%%%%%%%%%%%%%%%%%%%%%%%%%%%%%%%%%%%%%%%%%%%%
%%%%%%%%%%%%%%%%%%%%%%%%%%%%%%%%%%%%%%%%%%%%%%%%%%%%%%%%%%%%%%%%%%%%%%%%%%%%%%%%
\vskip0.1in\hrule\vskip0.1in
\noindent
{\bf Absolute and Relative Error: Examples.} 
\vskip0.1in\hrule\vskip0.1in
\noindent
If we are trying to find the roots of the polynomial
$$
  p(x) = x^5 + x^3 - 2\ x^2 + 5\ x
$$
we can see that $x=0$ is one solution. To find other roots we can use Newton's
method to generate a sequence of approximations given a starting point. That is,
we will generate a sequence
$$
   S = \left\lbrace x_k \right\rbrace_{k=0}^{\infty}
$$
Newton's method will be covered a bit later in this course. However, what we
will want is for the sequence to converge to a root, say $x^*$. This can be
rephrased as
$$ | x_k - x^* | \rightarrow 0 $$
which implies the absolute error will tend to zero as $k$ tends to $\infty$.
Using the relative error we want
$$ {{ | x_k - x^* | }\over{ | x^* | }} \rightarrow 0 $$
For the polynomial define in this example, there will be problems in using the
relative error near the zero roots. So, if the sequence starts to converg to the
zero root, we would need to use the absolute error as a measure.
%%%%%%%%%%%%%%%%%%%%%%%%%%%%%%%%%%%%%%%%%%%%%%%%%%%%%%%%%%%%%%%%%%%%%%%%%%%%%%%%
%%%%%%%%%%%%%%%%%%%%%%%%%%%%%%%%%%%%%%%%%%%%%%%%%%%%%%%%%%%%%%%%%%%%%%%%%%%%%%%%

%%%%%%%%%%%%%%%%%%%%%%%%%%%%%%%%%%%%%%%%%%%%%%%%%%%%%%%%%%%%%%%%%%%%%%%%%%%%%%%%
%%%%%%%%%%%%%%%%%%%%%%%%%%%%%%%%%%%%%%%%%%%%%%%%%%%%%%%%%%%%%%%%%%%%%%%%%%%%%%%%
\vskip0.1in\hrule\vskip0.1in
\noindent
{\bf Absolute and Relative Error: A Numerical Example.} 
\vskip0.1in\hrule\vskip0.1in
\noindent
As an illustration of how absolute and relative errors compare, we can consider
some numerical examples. The following table gives a pair of numbers along with
both the absolute and relative errors.

\begin{table}[h!]
   \caption{Absolute and Relative Error Values}
   \begin{center}
   \begin{tabular}{c|c|c|c}
     \hline
     \textbf{$x$} & \textbf{$x^*$} & \textbf{abs. err.} & \textbf{rel. err.} \\
     \hline
     0.01 & 0.1 & 0.09 & 0.9 \\
     \hline
     1.01 & 1.0 & 0.01 & 0.01 \\
     \hline
     2.0   & 3.0  & 1.0 & 0.5 \\
     \hline
     10.0  & 9.0  & 1.0 & 0.1 \\
     \hline
     100.0 & 99.0 & 1.0 & 0.01 \\
     \hline
   \end{tabular}
   \end{center}
\end{table}

Note that when both the approximation and exact value are close to zero, the
absolute error becomes a better measure of error than the relative error and
for large values (at least greater than one) one should expect that the better
measure of the error is the relative error. We will use both in the development
of algorithms to numerically solve mathematical problems.






The following is a list of sources for error that need to be taken into account
by compoutational scientists.
\begin{enumerate}
  \item {\bf Modeling Errors} These errors can occur when assumptions are made
        about the phenomena being studied. For example, one may consider a model
        of the solar system where the planets are assumed to be spheres, which
        is not the case.
  \item {\bf Measurement Errors:} These errors occur when instruments are used
        to measure physical quantities. For example, the temperature of molten
        lava might be measured to within one or two degrees based on the
        magnitude of the exact temperature. The fractional part of the 
        measurement would characterize the error.
  \item {\bf Discretization Error:} In order to computer solutions to
        mathematical problems using computers, it necessary that the model be
        finite and discrete. For example, weather models based on systems of
        partial differential equations require a discretization of the continuous model to fit in the
        discrete framework of a computer simulation.
\end{enumerate}
Note that Github will not allow you the luxury of creating empty folders. This
is an advantage in using \lq\lq git\rq\rq\ on a local machine. When changes are
\lq\lq pushed\rq\rq\ to Github, empty folders are ignored. So, let's get started
on the formatting the homework solutions portion of the repository for the
class.
%%%%%%%%%%%%%%%%%%%%%%%%%%%%%%%%%%%%%%%%%%%%%%%%%%%%%%%%%%%%%%%%%%%%%%%%%%%%%%%%
%%%%%%%%%%%%%%%%%%%%%%%%%%%%%%%%%%%%%%%%%%%%%%%%%%%%%%%%%%%%%%%%%%%%%%%%%%%%%%%%
\end{document}
