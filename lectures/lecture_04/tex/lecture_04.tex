\documentclass[10pt,fleqn]{article}
\usepackage{hyperref}

\setlength{\topmargin}{-.75in}
\addtolength{\textheight}{2.00in}
\setlength{\oddsidemargin}{.00in}
\addtolength{\textwidth}{.75in}

\nofiles

\pagestyle{empty}

\setlength{\parindent}{0in}

% new math commands


\setlength{\oddsidemargin}{-0.25in}
\setlength{\evensidemargin}{-0.25in}
\setlength{\textwidth}{6.75in}
\setlength{\headheight}{0.0in}
\setlength{\topmargin}{-0.25in}
\setlength{\textheight}{9.00in}

\makeindex

\usepackage{mathrsfs}

%\usepackage[pdftex]{graphicx}
\usepackage{epstopdf}

\newcounter{beans}

\newcommand{\ds}{\displaystyle}
\newcommand{\limit}[2]{\displaystyle\lim_{#1\to#2}}

\newcommand{\binomial}[2]{\ \left( \begin{array}{c}
                                  #1 \\
                                  #2
                                 \end{array}
                            \right) \
                         }
\newcommand{\ExampleRule}[2]
  {
  \noindent
  \rule{\linewidth}{1pt}
  \begin{example}
    #1
    \label{#2}
  \end{example}
  \rule{\linewidth}{1pt}
  \vskip0.125in
  }

\newcommand{\defbox}[1]
  {
   \ \\
   \noindent
   \setlength\fboxrule{1pt}
   \fbox{
        \begin{minipage}{6.5in}
          #1
        \end{minipage}
        }
   \ \\
  }
\newcommand{\verysmallworkbox}[1]
  {
   \ \\
   \noindent
   \setlength\fboxrule{1pt}
   \fbox{
        \begin{minipage}{6.5in}
           #1
           \ \\
           \vskip0.5in \ \\
           \ \\
        \end{minipage}
        }
   \ \\
  }
\newcommand{\smallworkbox}[1]
  {
   \ \\
   \noindent
   \setlength\fboxrule{1pt}
   \fbox{
        \begin{minipage}{6.5in}
           #1
           \ \\
           \vskip2.5in \ \\
           \ \\
        \end{minipage}
        }
   \ \\
  }
\newcommand{\halfworkbox}[1]
  {
   \ \\
   \noindent
   \setlength\fboxrule{1pt}
   \fbox{
        \begin{minipage}{6.5in}
           #1 \hfill
           \ \\
           \vskip3.25in \ \\
           \ \\
        \end{minipage}
        }
   \ \\
  }
\newcommand{\largeworkbox}[1]
  {
   \ \\
   \noindent
   \setlength\fboxrule{1pt}
   \fbox{
        \begin{minipage}{6.5in}
           #1
           \ \\
           \vskip7.5in \ \\
           \ \\
        \end{minipage}
        }
   \ \\
  }
\newcommand{\flexworkbox}[2]
  {
   \ \\
   \noindent
   \setlength\fboxrule{1pt}
   \fbox{
        \begin{minipage}{6.5in}
           #1
           \ \\

           \vskip#2 \ \\
           \ \\
        \end{minipage}
        }
   \ \\
  }


% symbols for sets of numbers

\newcommand{\natnumb}{$\cal N$}
\newcommand{\whonumb}{$\cal W$}
\newcommand{\intnumb}{$\cal Z$}
\newcommand{\ratnumb}{$\cal Q$}
\newcommand{\irrnumb}{$\cal I$}
\newcommand{\realnumb}{$\cal R$}
\newcommand{\cmplxnumb}{$\cal C$}

% misc. commands

\newcommand{\mma}{{\it Mathematica}}
\newcommand{\sech}{\mbox{ sech}}
 
\newtheorem{theorem}{Theorem}
\newtheorem{example}{Example}
\newtheorem{definition}{Definition}
\newtheorem{problem}{Problem}

\setcounter{secnumdepth}{2}
\setcounter{tocdepth}{4}


\begin{document}
%%%%%%%%%%%%%%%%%%%%%%%%%%%%%%%%%%%%%%%%%%%%%%%%%%%%%%%%%%%%%%%%%%%%%%%%%%%%%%%%
%%%%%%%%%%%%%%%%%%%%%%%%%%%%%%%%%%%%%%%%%%%%%%%%%%%%%%%%%%%%%%%%%%%%%%%%%%%%%%%%
\vskip0.1in\hrule\vskip0.1in
\noindent
{\bf Math 4610 Fundamentals of Computational Mathematics  - Lecture 4.} 
\vskip0.1in\hrule\vskip0.1in
\noindent
In the last lesson algorithms for computing limitations on representation of
numbers. This can be done for arbitrary precision. However, for most coding
languages, like C, C++, Fortran, and so on, it is enough to establish the
limitation for single and double precision. In the language of coding and
computing it is typical to associate 32-bit word lengths with single precision
and 64-bit word lengths with double precision. It should be noted that
interpretted languages like Python allow users to work with arbitrary precision.
However, the work is done with software and will produce inefficient computer
codes.

Another thing to note is in some programming languages, single and double
precision may not be the same. For example, in older versions of C compilers
the {\bf float} declaration may provide exactly the same precision as a double
declaration. So, in spite of standardization and other assumptions, it is a
good idea to test any computer that you will be using to perform lots of
calculations. So, it makes sense to write a single code that you can use on any
computer that will display how accurate numbers can be represented.

\end{document}



\begin{verbatim}

  1. Explicit Euler method and error

\end{verbatim}



Representation of round off error and the analysis of how errors accumulate in
complex computations. One way to proceed is through interval analysis applied to
numbers that are approximated in machine arithmetic. A simple 
example of how to build a shared library will also be discussed. This is an
important skill to have in computational mathematics.
%%%%%%%%%%%%%%%%%%%%%%%%%%%%%%%%%%%%%%%%%%%%%%%%%%%%%%%%%%%%%%%%%%%%%%%%%%%%%%%%
%%%%%%%%%%%%%%%%%%%%%%%%%%%%%%%%%%%%%%%%%%%%%%%%%%%%%%%%%%%%%%%%%%%%%%%%%%%%%%%%
\vskip0.1in\hrule\vskip0.1in
\noindent
{\bf Content Items:}
\begin{list}{$\bullet$}{\usecounter{beans} \parsep=0pt \listparindent=0pt
\topsep=0pt \rightmargin=.35in \leftmargin=.35in \labelsep=5 pt
\itemsep=2pt}
  \item {\bf Homework repositories on Github:} - We will discuss how to create a
     homework repository that each student must have to complete homework. The
     homework repository should be a private repository with you instructor as
     the only collaborator. Homework will be graded by first cloning the
     repository and then your constructor can create a branch for grading. This
     can be repeated by creating a pull request on the repository.
     [> > go there](https://www.github.io/jvkoebbe/math4610/lectures/lecture_02/homework_structure)
     | [(pdf)](https://jvkoebbe.github.io/math4610/lecture_02/pdf/homework.pdf)

        \href{https://jvkoebbe.github.io/math4610/syllabus/md/syllabus}

  \item
        \href{https://jvkoebbe.github.io/math4610/lectures/lecture_01/pdf/cygwin_primer.pdf}

  \item {\bf Using Version Control Systems (VCS) - git:} A brief discussion of
     \lq\lq git\rq\rq\ will be taken up in class to show how to work and
     collaborate with other students and your instructor. The \lq\lq git\rq\rq\ 
     platform allows you to work on a laptop at home and then \lq\lq push\rq\rq\ 
     your work and any modifications to Github where the instructor can get to
     the work.
       [> > go there](https://jvkoebbe.github.io/math4610/lectures/lecture_02/md/git_primer)
       | [(pdf)](https://jvkoebbe.github.io/math4610/lecture_02/pdf/git_primer.pdf)
  \item {\bf Code for Determining Machine Precision:} A simple algorithm for
     determining the precision for number representation will be covered. Two
     different versions of the code will be presented that determine single and
     double precision for your computer. Note that these two codes will be used
     to show students how to build a shared library.
       [> > go there](https://jvkoebbe.github.io/math4610/lecture_02/html/finite_precision)
       | [(pdf)](https://jvkoebbe.github.io/math4610/lecture_02/pdf/git_primer.pdf)

  \item {\bf Building a Shared Library for Reusing Code:} In important skill
     that computational mathematicians should learn is the ability to take
     relatively small codes and turning these into reusable object files and
     then collecting the object files into a shared library. 
       [> > go there](https://jvkoebbe.github.io/math4610/lecture_02/html/finite_precision)
       | [(pdf)](https://jvkoebbe.github.io/math4610/lecture_02/pdf/git_primer.pdf)
        \href{https://jvkoebbe.github.io/math4610/lectures/lecture_01/pdf/github_primer.pdf}

  \item {\bf Testing the Codes in a Shared Library:} We will go over how to
     write a code that uses the object routines in a shared library. Linking a
     main code to a library will be covered.
       [> > go there](https://jvkoebbe.github.io/math4610/lecture_02/html/finite_precision)
       | [(pdf)](https://jvkoebbe.github.io/math4610/lecture_02/pdf/git_primer.pdf)

  \item {\bf Floating Point Representation of Numbers on Computers:} The IEEE
     standard for number representation will be covered. Number precision for
     32-bit and 64-bit systems will be covered. 

  \item {\bf Wrap up and Questions:} If there is time and anyone has questions
        about the lecture, these will be addressed.

\end{list}
\vskip0.1in\hrule\vskip0.1in
%%%%%%%%%%%%%%%%%%%%%%%%%%%%%%%%%%%%%%%%%%%%%%%%%%%%%%%%%%%%%%%%%%%%%%%%%%%%%%%%
%%%%%%%%%%%%%%%%%%%%%%%%%%%%%%%%%%%%%%%%%%%%%%%%%%%%%%%%%%%%%%%%%%%%%%%%%%%%%%%%
\end{document}
# Math 4610 Fundamentals of Computational Mathematics  - Lecture 3

The accumulation of round off error can be a serious problem in computational
mathematics. The idea in this lecture is to focus on errors that occur in every
arithmetic operation performed by a computer.
Some simple examples of how to
mitigate inexact arithmetic interval analysis will
be covered. Documentation of algorithms and computer codes is essential to any
successful attempt to write reusable code. Finally, we will introduce the
general mathematical problem of finding the roots of a nonlinear function of a
single variable.

<hr>

## Content Items:

  1. **Building a Shared Library for Reusing Code:** An important skill that
     computational mathematicians should learn is to take relatively small codes
     and turn these into reusable object files. These can be collected into a
     shared library. A shared library can be linked to by other programs from
     this point on. Note that the equivalent of a shared library in a Windows
     setting is called a DLL.
       [> > go there](https://jvkoebbe.github.io/math4610/lecture_03/md/shared_library_example)
       | [(pdf)](https://jvkoebbe.github.io/math4610/lecture_03/pdf/git_primer.pdf)

  2. **Roundoff Errors and Arithmetic Operations:** - Every computation done
     combines two numbers that in general are only approximations of the exact
     values. The errors in some cases result in catastrophic cancellation that
     result in output that is garbage. We will look at the sum, difference,
     product, and ratio of numbers. Note that all operations involve some
     combination of these four operations.
     [> > go there](https://jvkoebbe.github.io/math4610/lectures/lecture_02/md/precision_00)
     | [(pdf)](https://jvkoebbe.github.io/math4610/lectures/lecture_02/pdf/precision_00.pdf)

     * [**Intro:**](https://jvkoebbe.github.io/math4610/lectures/lecture_02/md/precision_00)
     * [**Part 1:**](https://jvkoebbe.github.io/math4610/lectures/lecture_02/html/precision_01.html)
     * [**Part 2:**](https://jvkoebbe.github.io/math4610/lectures/lecture_02/html/precision_02.html)
     * [**Part 3:**](https://jvkoebbe.github.io/math4610/lectures/lecture_02/html/precision_03.html)
     * [**Part 4:**](https://jvkoebbe.github.io/math4610/lectures/lecture_02/html/precision_04.html)

  3. **Testing the Codes in a Shared Library:** We will go over how to write a
     code that uses the object routines in a shared library. Linking a main
     code to a library will be covered.
       [> > go there](https://jvkoebbe.github.io/math4610/lecture_03/html/finite_precision)
       | [(pdf)](https://jvkoebbe.github.io/math4610/lecture_03/pdf/git_primer.p

  4. **Analytic Fixes for Cancellation Errors:** In some cases, the arithmetic
     operations needed in an algorithm can be rearranged through the use of
     some mathematical manipulation to obtain an equivalent algorithm that is
     more stable. A couple of examples will be treated.
       [> > go there](https://jvkoebbe.github.io/math4610/lectures/lecture_02/md/git_primer)
       | [(pdf)](https://jvkoebbe.github.io/math4610/lecture_02/pdf/git_primer.pdf)

  5. **An Introduction to Interval Analysis:** Interval analysis involves doing
     arithmetic on intervals instead of numbers. This will be made a bit more
     clear in the lecture. Sums, differences, products, and quotients are all
     treated within the structre of intervals. 
       [> > go there](https://jvkoebbe.github.io/math4610/lecture_02/html/finite_precision)
       | [(pdf)](https://jvkoebbe.github.io/math4610/lecture_02/pdf/git_primer.pdf)

  4. **Documentation of Reusable Code:**  Each student will need to write a
     software manual for the algorithms implemented in class. A template for the
     software manual has been created by your instructor using Jekyll on Github.
     Students are required to use the template and can look at the template and
     download a copy of the template for their own use.
       [> > go there](https://jvkoebbe.github.io/math4610/lecture_02/html/finite_precision)
       | [(pdf)](https://jvkoebbe.github.io/math4610/lecture_02/pdf/git_primer.pdf)

  5. **Statement of the Root Finding Problem:**  Many applied mathematics
     problems can be stated in the form of a general root finding problem. In
     this lecture the general root finding problem will be stated. Algorithms
     for the root finding problem will be presented in the next few lectures.
       [> > go there](https://jvkoebbe.github.io/math4610/lecture_02/html/finite_precision)
       | [(pdf)](https://jvkoebbe.github.io/math4610/lecture_02/pdf/git_primer.pdf)

  6. **Wrap up and Questions:**  If there is time and anyone has questions about
     the lecture, these will be addressed.

---

[prev](https://jvkoebbe.github.io/math4610/lectures/lecture_02/md/lecture_02) |
[toc](https://jvkoebbe.github.io/math4610/lectures/toc_lectures) |
[next](https://jvkoebbe.github.io/math4610/lectures/lecture_04/md/lecture_04)

---
