\documentclass[10pt,fleqn]{article}
%\usepackage{graphicx}


\setlength{\topmargin}{-.75in}
\addtolength{\textheight}{2.00in}
\setlength{\oddsidemargin}{.00in}
\addtolength{\textwidth}{.75in}

\nofiles

\pagestyle{empty}

\setlength{\parindent}{0in}

% new math commands


\setlength{\oddsidemargin}{-0.25in}
\setlength{\evensidemargin}{-0.25in}
\setlength{\textwidth}{6.75in}
\setlength{\headheight}{0.0in}
\setlength{\topmargin}{-0.25in}
\setlength{\textheight}{9.00in}

\makeindex

\usepackage{mathrsfs}

%\usepackage[pdftex]{graphicx}
\usepackage{epstopdf}

\newcounter{beans}

\newcommand{\ds}{\displaystyle}
\newcommand{\limit}[2]{\displaystyle\lim_{#1\to#2}}

\newcommand{\binomial}[2]{\ \left( \begin{array}{c}
                                  #1 \\
                                  #2
                                 \end{array}
                            \right) \
                         }
\newcommand{\ExampleRule}[2]
  {
  \noindent
  \rule{\linewidth}{1pt}
  \begin{example}
    #1
    \label{#2}
  \end{example}
  \rule{\linewidth}{1pt}
  \vskip0.125in
  }

\newcommand{\defbox}[1]
  {
   \ \\
   \noindent
   \setlength\fboxrule{1pt}
   \fbox{
        \begin{minipage}{6.5in}
          #1
        \end{minipage}
        }
   \ \\
  }
\newcommand{\verysmallworkbox}[1]
  {
   \ \\
   \noindent
   \setlength\fboxrule{1pt}
   \fbox{
        \begin{minipage}{6.5in}
           #1
           \ \\
           \vskip0.5in \ \\
           \ \\
        \end{minipage}
        }
   \ \\
  }
\newcommand{\smallworkbox}[1]
  {
   \ \\
   \noindent
   \setlength\fboxrule{1pt}
   \fbox{
        \begin{minipage}{6.5in}
           #1
           \ \\
           \vskip2.5in \ \\
           \ \\
        \end{minipage}
        }
   \ \\
  }
\newcommand{\halfworkbox}[1]
  {
   \ \\
   \noindent
   \setlength\fboxrule{1pt}
   \fbox{
        \begin{minipage}{6.5in}
           #1 \hfill
           \ \\
           \vskip3.25in \ \\
           \ \\
        \end{minipage}
        }
   \ \\
  }
\newcommand{\largeworkbox}[1]
  {
   \ \\
   \noindent
   \setlength\fboxrule{1pt}
   \fbox{
        \begin{minipage}{6.5in}
           #1
           \ \\
           \vskip7.5in \ \\
           \ \\
        \end{minipage}
        }
   \ \\
  }
\newcommand{\flexworkbox}[2]
  {
   \ \\
   \noindent
   \setlength\fboxrule{1pt}
   \fbox{
        \begin{minipage}{6.5in}
           #1
           \ \\

           \vskip#2 \ \\
           \ \\
        \end{minipage}
        }
   \ \\
  }


% symbols for sets of numbers

\newcommand{\natnumb}{$\cal N$}
\newcommand{\whonumb}{$\cal W$}
\newcommand{\intnumb}{$\cal Z$}
\newcommand{\ratnumb}{$\cal Q$}
\newcommand{\irrnumb}{$\cal I$}
\newcommand{\realnumb}{$\cal R$}
\newcommand{\cmplxnumb}{$\cal C$}

% misc. commands

\newcommand{\mma}{{\it Mathematica}}
\newcommand{\sech}{\mbox{ sech}}
 
\newtheorem{theorem}{Theorem}
\newtheorem{example}{Example}
\newtheorem{definition}{Definition}
\newtheorem{problem}{Problem}

\setcounter{secnumdepth}{2}
\setcounter{tocdepth}{4}


\begin{document}
%%%%%%%%%%%%%%%%%%%%%%%%%%%%%%%%%%%%%%%%%%%%%%%%%%%%%%%%%%%%%%%%%%%%%%%%%%%%%%%%
%%%%%%%%%%%%%%%%%%%%%%%%%%%%%%%%%%%%%%%%%%%%%%%%%%%%%%%%%%%%%%%%%%%%%%%%%%%%%%%%
\vskip0.1in\hrule\vskip0.1in
\noindent
{\bf Math 4610 Fundamentals of Computational Mathematics  -  Single and Double
 Precision Numbers.}
\vskip0.1in\hrule\vskip0.1in
\noindent
As a way to make sure that finite precision numbers and the arithmetic
operations used to work with these numbers produce correct results we need a
consistent/accepted format for numbers used on any/all computers. It is typical
to define numbers in a fixed format that can be manipulated in predictable ways.
A standard way to classify the storage of numbers is in terms of whether the
number representation is of single or double precision. Single precision is a
number format that occupies 32 bits and double precision is a number format that
typically uses 64 bits. Note that the use of these formats also dictates memory
limitations in terms of the total amount of memory and disk space available to
actually do work.
%%%%%%%%%%%%%%%%%%%%%%%%%%%%%%%%%%%%%%%%%%%%%%%%%%%%%%%%%%%%%%%%%%%%%%%%%%%%%%%%
%%%%%%%%%%%%%%%%%%%%%%%%%%%%%%%%%%%%%%%%%%%%%%%%%%%%%%%%%%%%%%%%%%%%%%%%%%%%%%%%

%%%%%%%%%%%%%%%%%%%%%%%%%%%%%%%%%%%%%%%%%%%%%%%%%%%%%%%%%%%%%%%%%%%%%%%%%%%%%%%%
%%%%%%%%%%%%%%%%%%%%%%%%%%%%%%%%%%%%%%%%%%%%%%%%%%%%%%%%%%%%%%%%%%%%%%%%%%%%%%%%
Note that for integers the story is a little different. Integers can be
represented exactly given enough bits. For example, if we work with a
signed 32 bit integer, the largest integer that can be represented in memory is
$2^{31}-1=2,147,483,647$. For real numbers, we need more information to be able
to represent a number. Real numbers are typically described by (1) the mantissa
or fractional part of a number, (2) an (integer) exponent for the base being
used, and (3) a plus or minus sign represented by a zero for a positive number
or a one for negative numbers.
%%%%%%%%%%%%%%%%%%%%%%%%%%%%%%%%%%%%%%%%%%%%%%%%%%%%%%%%%%%%%%%%%%%%%%%%%%%%%%%%
%%%%%%%%%%%%%%%%%%%%%%%%%%%%%%%%%%%%%%%%%%%%%%%%%%%%%%%%%%%%%%%%%%%%%%%%%%%%%%%%

%%%%%%%%%%%%%%%%%%%%%%%%%%%%%%%%%%%%%%%%%%%%%%%%%%%%%%%%%%%%%%%%%%%%%%%%%%%%%%%%
%%%%%%%%%%%%%%%%%%%%%%%%%%%%%%%%%%%%%%%%%%%%%%%%%%%%%%%%%%%%%%%%%%%%%%%%%%%%%%%%
Given these considerations for the format of computer numbers and a specified
number of bits to make available for numbers, the only thing left is to
determine how to divide up the bits. It makes sense to use the same number of
bits for each number used on a computer. For single precision numbers the IEEE
754 standard for this representation is defined by 32 bits organized as follows.

\begin{list}{$\bullet$}{\usecounter{beans} \parsep=0pt \listparindent=0pt
\topsep=0pt \rightmargin=.35in \leftmargin=.35in \labelsep=5 pt
\itemsep=2pt}
  \item bits 0 through 22 (23 bits) define the mantissa or fraction
  \item bits 23 through 30 (8 bits) define the integer exponent
  \item bit 31 (1 bit) for the sign of the number
\end{list}

So, a standard single precision number can be written in the form
\[
  x_{base 2} = (-1)^{b_{31}} \times 2^{b_{30}b_{29}\cdots b_{32}-127}
     \times (1.b_{22}b_{21}\ldots b_{0})_2
\]
Note that the terms, $b_i$, $i=0,1,\ldots,31$ are binary digits. These digits
are either 0 or 1. Also, the exponent is shifted to be symmetric about the
origin. That way the exponent can be used to resolve very small numbers
(e.g, $2^{-127}$) and very large numbers (e.g, $2^{128}$). This gives a complete
description of the 32-bit format. Note that the bits are indexed from zero and
not one. 
%%%%%%%%%%%%%%%%%%%%%%%%%%%%%%%%%%%%%%%%%%%%%%%%%%%%%%%%%%%%%%%%%%%%%%%%%%%%%%%%
%%%%%%%%%%%%%%%%%%%%%%%%%%%%%%%%%%%%%%%%%%%%%%%%%%%%%%%%%%%%%%%%%%%%%%%%%%%%%%%%

%%%%%%%%%%%%%%%%%%%%%%%%%%%%%%%%%%%%%%%%%%%%%%%%%%%%%%%%%%%%%%%%%%%%%%%%%%%%%%%%
%%%%%%%%%%%%%%%%%%%%%%%%%%%%%%%%%%%%%%%%%%%%%%%%%%%%%%%%%%%%%%%%%%%%%%%%%%%%%%%%
In a previous part of this lecture, an algorithm was coded (maceps) that returns
the number of digits expected to be correct for a machine number. We can
actually verify that this makes sense using the following list of values in
32-bit numbers.

\begin{list}{$\bullet$}{\usecounter{beans} \parsep=0pt \listparindent=0pt
\topsep=0pt \rightmargin=.35in \leftmargin=.35in \labelsep=5 pt
\itemsep=2pt}
  \item $1 + 2^{-23} \approx 1.000000119$
  \item $2 - 2^{-23} \approx 1.999999881$
  \item $2^{-126} \approx 1.175549435 \times 10^{-38}$
  \item $2^{127} \approx 1.70141183 \times 10^{38}$
\end{list}

The importance of these specific numbers is the following. The first number is
exactly the value we should see from the single precision machine epsilon code
before the number 1 is subtracted. The number of digits of accuracy is then
available using the fractional piece of the approximation.
%%%%%%%%%%%%%%%%%%%%%%%%%%%%%%%%%%%%%%%%%%%%%%%%%%%%%%%%%%%%%%%%%%%%%%%%%%%%%%%%
%%%%%%%%%%%%%%%%%%%%%%%%%%%%%%%%%%%%%%%%%%%%%%%%%%%%%%%%%%%%%%%%%%%%%%%%%%%%%%%%

%%%%%%%%%%%%%%%%%%%%%%%%%%%%%%%%%%%%%%%%%%%%%%%%%%%%%%%%%%%%%%%%%%%%%%%%%%%%%%%%
%%%%%%%%%%%%%%%%%%%%%%%%%%%%%%%%%%%%%%%%%%%%%%%%%%%%%%%%%%%%%%%%%%%%%%%%%%%%%%%%
The term double precision refers to a 64 bit format with the bits distributed
as described below.

\begin{list}{$\bullet$}{\usecounter{beans} \parsep=0pt \listparindent=0pt
\topsep=0pt \rightmargin=.35in \leftmargin=.35in \labelsep=5 pt
\itemsep=2pt}
  \item bits 0 through 51 (52 bits) define the mantissa or fraction
  \item bits 52 through 62 (11 bits) define the integer exponent
  \item bit 63 (1 bit) for the sign of the number
\end{list}

More bits in the mantissa and exponent means exact represntation of more real
numbers. Note tha there will still be lots of holes in the numbers represented
by a 64-bit machine number. The machine epsilon code should indicate the
accuracy of the approximation just as in the 32-bit case.
%%%%%%%%%%%%%%%%%%%%%%%%%%%%%%%%%%%%%%%%%%%%%%%%%%%%%%%%%%%%%%%%%%%%%%%%%%%%%%%%
%%%%%%%%%%%%%%%%%%%%%%%%%%%%%%%%%%%%%%%%%%%%%%%%%%%%%%%%%%%%%%%%%%%%%%%%%%%%%%%%

%%%%%%%%%%%%%%%%%%%%%%%%%%%%%%%%%%%%%%%%%%%%%%%%%%%%%%%%%%%%%%%%%%%%%%%%%%%%%%%%
%%%%%%%%%%%%%%%%%%%%%%%%%%%%%%%%%%%%%%%%%%%%%%%%%%%%%%%%%%%%%%%%%%%%%%%%%%%%%%%%
Computers would be incredibly useless if all a computer could do is store
approximations of numbers. What makes a computer extremely useful is the ability
to combine numbers through arithmetic operations and do the work accurately (no
human error) and efficiently. In this part of the course we will use the
notation
\[
  fl(x) \approx x
\]
where $fl$ is associated with the term floating point. Most operations done by
computers are the four standard binary operations of addition, subtraction,
division, and multiplication. The term floating point comes from process of
lining up the decimal points before performing the operation. This is done with
a shify in the exponent of the numbers.
%%%%%%%%%%%%%%%%%%%%%%%%%%%%%%%%%%%%%%%%%%%%%%%%%%%%%%%%%%%%%%%%%%%%%%%%%%%%%%%%
%%%%%%%%%%%%%%%%%%%%%%%%%%%%%%%%%%%%%%%%%%%%%%%%%%%%%%%%%%%%%%%%%%%%%%%%%%%%%%%%

%%%%%%%%%%%%%%%%%%%%%%%%%%%%%%%%%%%%%%%%%%%%%%%%%%%%%%%%%%%%%%%%%%%%%%%%%%%%%%%%
%%%%%%%%%%%%%%%%%%%%%%%%%%%%%%%%%%%%%%%%%%%%%%%%%%%%%%%%%%%%%%%%%%%%%%%%%%%%%%%%
For the moment, suppose that we have two numbers and want to look at the output
of a single binary operation. Without loss of generality, we can assume the
numbers are both positive. If the numbers are $x$ and $y$, then we can write
\[
   x = fl(x) + \epsilon_x
\]
and
\[
   y = fl(y) + \epsilon_y
\]
Adding the two results gives
\[
   x + y = ( fl(x) + \epsilon_x ) + ( fl(y) + \epsilon_y )
         = ( fl(x) + fl(y) ) + ( \epsilon_x + \epsilon_y )
\]
Since the values are positive, one would expect that the error in this case
will be small.
%%%%%%%%%%%%%%%%%%%%%%%%%%%%%%%%%%%%%%%%%%%%%%%%%%%%%%%%%%%%%%%%%%%%%%%%%%%%%%%%
%%%%%%%%%%%%%%%%%%%%%%%%%%%%%%%%%%%%%%%%%%%%%%%%%%%%%%%%%%%%%%%%%%%%%%%%%%%%%%%%

%%%%%%%%%%%%%%%%%%%%%%%%%%%%%%%%%%%%%%%%%%%%%%%%%%%%%%%%%%%%%%%%%%%%%%%%%%%%%%%%
%%%%%%%%%%%%%%%%%%%%%%%%%%%%%%%%%%%%%%%%%%%%%%%%%%%%%%%%%%%%%%%%%%%%%%%%%%%%%%%%
The binary of operation of subtracting two numbers is not as stable. For
\[
   x - y = ( fl(x) - \epsilon_x ) + ( fl(y) - \epsilon_y )
         = ( fl(x) - fl(y) ) + ( \epsilon_x - \epsilon_y )
\]
The problem is that if the two numbers are within machine precision of each
other, then the difference will produce a value that is of the same order of
magnitude as the difference
\[
  \makebox{ error } = \epsilon_x - \epsilon_y
\]
In floating point arithmetic, the two numbers are aligned at the decimal point
and when the difference is computed all significant digits can canel out. Since
there are no extra digits to use in these representations, it is usual that the
output from this operation is garbage. This is called catastrophic cancellation.
The real issue is if a result like this is used in a later computation. This
means the output cannot be trusted.
%%%%%%%%%%%%%%%%%%%%%%%%%%%%%%%%%%%%%%%%%%%%%%%%%%%%%%%%%%%%%%%%%%%%%%%%%%%%%%%%
%%%%%%%%%%%%%%%%%%%%%%%%%%%%%%%%%%%%%%%%%%%%%%%%%%%%%%%%%%%%%%%%%%%%%%%%%%%%%%%%

%%%%%%%%%%%%%%%%%%%%%%%%%%%%%%%%%%%%%%%%%%%%%%%%%%%%%%%%%%%%%%%%%%%%%%%%%%%%%%%%
%%%%%%%%%%%%%%%%%%%%%%%%%%%%%%%%%%%%%%%%%%%%%%%%%%%%%%%%%%%%%%%%%%%%%%%%%%%%%%%%
Multiplication and division behave similarly to addition and subtraction,
respectively. That is. the multiplication of two numbers is a stable operation
while the ratio of two numbers is not stable for all computations. We can write
the product of two numbers as
\[
   x * y = ( fl(x) + \epsilon_x ) * ( fl(y) + \epsilon_y )
         = fl(x) * fl(y) + \epsilon_x fl(y) + \epsilon_y * fl(x)
                                          + \epsilon_x * \epsilon_y
\]
and for the ratio of numbers
\[
   {x \over y} = {{ fl(x) - \epsilon_x }\over{ fl(y) - \epsilon_y }}
\]
The expression for the product is not too difficult to analyze. The last three
terms in the expansion of the product are, in general, small relative to the
first term. So, the product is relatively stable. As for the ratio of numbers,
the analysis is a bit more complicated. A typical way to handle these types of
errors is to use interval analysis. We will take up an introduction to interval
analysis in the near future. For now, we will move on to the types of error one
encounters in using computers to solve problems.
%%%%%%%%%%%%%%%%%%%%%%%%%%%%%%%%%%%%%%%%%%%%%%%%%%%%%%%%%%%%%%%%%%%%%%%%%%%%%%%%
%%%%%%%%%%%%%%%%%%%%%%%%%%%%%%%%%%%%%%%%%%%%%%%%%%%%%%%%%%%%%%%%%%%%%%%%%%%%%%%%

%%%%%%%%%%%%%%%%%%%%%%%%%%%%%%%%%%%%%%%%%%%%%%%%%%%%%%%%%%%%%%%%%%%%%%%%%%%%%%%%
%%%%%%%%%%%%%%%%%%%%%%%%%%%%%%%%%%%%%%%%%%%%%%%%%%%%%%%%%%%%%%%%%%%%%%%%%%%%%%%%
From the start there have been cases where machine precision errors have caused
disasters. There are a couple of web sites that document these types of failures
The sites can be found at:
\begin{verbatim}

     http://www.ima.umn.edu/~arnold/disasters

\end{verbatim}
and
\begin{verbatim}

     http://wwwzenger.informatik.tu-muenchen.de/persons/huckle/bugse.html

\end{verbatim}
It is informative to check these types of cautionary tales so that we do not
reproduce the same types of errors. There is also a book one can read entitled
\lq\lq\ Set Phasers on Stun\rq\rq\ with similar types of real world problems
that were caused by not paying attention to the details.
%%%%%%%%%%%%%%%%%%%%%%%%%%%%%%%%%%%%%%%%%%%%%%%%%%%%%%%%%%%%%%%%%%%%%%%%%%%%%%%%
%%%%%%%%%%%%%%%%%%%%%%%%%%%%%%%%%%%%%%%%%%%%%%%%%%%%%%%%%%%%%%%%%%%%%%%%%%%%%%%%

%%%%%%%%%%%%%%%%%%%%%%%%%%%%%%%%%%%%%%%%%%%%%%%%%%%%%%%%%%%%%%%%%%%%%%%%%%%%%%%%
%%%%%%%%%%%%%%%%%%%%%%%%%%%%%%%%%%%%%%%%%%%%%%%%%%%%%%%%%%%%%%%%%%%%%%%%%%%%%%%%










With this finite discrete representation there is a limit as to the set of
numbers that can be represented.














Any work that is done on a computer boils down to manipulating numbers. A
problem with this is that computers have finite resources and the representation
of many numbers requires the use of an infinite number of decimal digits. For
example, given a circle, the formula for the circumference is
$$C = 2 \times \pi \times r = \pi \times d$$
where $r$ is the radius of the circle and $d$ is the diameter of the circle. The
number $\pi$ is not a rational number. That is, the decimal expansion of this
value has an infinite fractional part. The value can be represented as follows:
$$\pi\approx 3.141592653589793...$$
where the ellipsis notation, $...$, means the digits never repeat. So, to get an
exact representation of $\pi$ it is necessary to have an infinite number of
digits available. Since computer resources are finite, we must settle for an
approximation.

In this part of the lecture, we will use a few examples that should motivate us
to spend some time on this issue and more fully understand the implications of
finite precision of number representation.

For the first example, we could use the approximation
$$\pi\approx 3.141592653589793$$
without including an infinite number of digits. One question that should arise
is how many digits will provide us with an accurate enough approximation. One
of the programs used over the past few decades to \lq\lq\ burn in machines was
an algorithm to compute more and more digits of $\pi$. This means that is
possible to determine $\pi$ to any degree of accuracy that we want. However, it
is not practical for real problems.

In some cases, a very crude approximation is enough. In some of our United
States, laws have been passed to legally approximate $\pi$ using a rational
number. For example,
$$\pi\approx {{22}\over 7}$$
provides an approximation that will hold up in a courts of law. If you are
pouring a circular concrete slab for a water tank it is a good idea to have an
estimate of the amount of concrete based on an accepted value for the number
$\pi$.

Basically, numbers are best represented on a computer using zeros and ones -
or in a binary number system. Other common number systems used in computer
architecture/hardware are in octal (or base 8) and hexadecimal (or base 16).
Another issue that arises in the representation of numbers is numbers that are
relatively prime to base 2. As a simple example, consider the representation of
the number $1/3$ in base 2.  The value is
$${1\over 3} = 0.01010101.....$$
where the last pair of digits repeats forever. If a finite number of binary
digits are used to represent $1/3$, the result is an approximation of the exact
value. Note that a base 10 representation of $1/3$ is given by the decimal
representation
$${1\over 3} = 0.3333333333333.........$$
Even if computers worked in a base 10 system, we would necessarily have to
settle for approximate number representation.

Since there are an uncountable number of irrational numbers, it is impossible to
imagine a computer that would not suffer the same issue. So, the best we can
hope for is that there is an accepted number representation that will work on
all computers. There is a standard (IEEE standard reference here) for number
representation that we will look into later in the course. For now, we will
assume that all of the computers we will use will behave the same way. From a
practical point of view, it would be nice to be able to compute the limits of
the accuracy of machine numbers. Fortunately, we can write a little program that
will do the trick for us.
%%%%%%%%%%%%%%%%%%%%%%%%%%%%%%%%%%%%%%%%%%%%%%%%%%%%%%%%%%%%%%%%%%%%%%%%%%%%%%%%
%%%%%%%%%%%%%%%%%%%%%%%%%%%%%%%%%%%%%%%%%%%%%%%%%%%%%%%%%%%%%%%%%%%%%%%%%%%%%%%%
\end{document}
