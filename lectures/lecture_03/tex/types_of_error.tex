\documentclass[10pt,fleqn]{article}
%\usepackage{graphicx}


\setlength{\topmargin}{-.75in}
\addtolength{\textheight}{2.00in}
\setlength{\oddsidemargin}{.00in}
\addtolength{\textwidth}{.75in}

\title{Math 4610 Lecture Notes \\
            \ \\
       Types of Errors in Scientific Computing
  \footnote{These notes are part of an Open Resource Educational project
            sponsored by Utah State University}}

\author{Joe Koebbe}

\nofiles

\pagestyle{empty}

\setlength{\parindent}{0in}

% new math commands


\setlength{\oddsidemargin}{-0.25in}
\setlength{\evensidemargin}{-0.25in}
\setlength{\textwidth}{6.75in}
\setlength{\headheight}{0.0in}
\setlength{\topmargin}{-0.25in}
\setlength{\textheight}{9.00in}

\makeindex

\usepackage{mathrsfs}

%\usepackage[pdftex]{graphicx}
\usepackage{epstopdf}

\newcounter{beans}

\newcommand{\ds}{\displaystyle}
\newcommand{\limit}[2]{\displaystyle\lim_{#1\to#2}}

\newcommand{\binomial}[2]{\ \left( \begin{array}{c}
                                  #1 \\
                                  #2
                                 \end{array}
                            \right) \
                         }
\newcommand{\ExampleRule}[2]
  {
  \noindent
  \rule{\linewidth}{1pt}
  \begin{example}
    #1
    \label{#2}
  \end{example}
  \rule{\linewidth}{1pt}
  \vskip0.125in
  }

\newcommand{\defbox}[1]
  {
   \ \\
   \noindent
   \setlength\fboxrule{1pt}
   \fbox{
        \begin{minipage}{6.5in}
          #1
        \end{minipage}
        }
   \ \\
  }
\newcommand{\verysmallworkbox}[1]
  {
   \ \\
   \noindent
   \setlength\fboxrule{1pt}
   \fbox{
        \begin{minipage}{6.5in}
           #1
           \ \\
           \vskip0.5in \ \\
           \ \\
        \end{minipage}
        }
   \ \\
  }
\newcommand{\smallworkbox}[1]
  {
   \ \\
   \noindent
   \setlength\fboxrule{1pt}
   \fbox{
        \begin{minipage}{6.5in}
           #1
           \ \\
           \vskip2.5in \ \\
           \ \\
        \end{minipage}
        }
   \ \\
  }
\newcommand{\halfworkbox}[1]
  {
   \ \\
   \noindent
   \setlength\fboxrule{1pt}
   \fbox{
        \begin{minipage}{6.5in}
           #1 \hfill
           \ \\
           \vskip3.25in \ \\
           \ \\
        \end{minipage}
        }
   \ \\
  }
\newcommand{\largeworkbox}[1]
  {
   \ \\
   \noindent
   \setlength\fboxrule{1pt}
   \fbox{
        \begin{minipage}{6.5in}
           #1
           \ \\
           \vskip7.5in \ \\
           \ \\
        \end{minipage}
        }
   \ \\
  }
\newcommand{\flexworkbox}[2]
  {
   \ \\
   \noindent
   \setlength\fboxrule{1pt}
   \fbox{
        \begin{minipage}{6.5in}
           #1
           \ \\

           \vskip#2 \ \\
           \ \\
        \end{minipage}
        }
   \ \\
  }


% symbols for sets of numbers

\newcommand{\natnumb}{$\cal N$}
\newcommand{\whonumb}{$\cal W$}
\newcommand{\intnumb}{$\cal Z$}
\newcommand{\ratnumb}{$\cal Q$}
\newcommand{\irrnumb}{$\cal I$}
\newcommand{\realnumb}{$\cal R$}
\newcommand{\cmplxnumb}{$\cal C$}

% misc. commands

\newcommand{\mma}{{\it Mathematica}}
\newcommand{\sech}{\mbox{ sech}}
 
\newtheorem{theorem}{Theorem}
\newtheorem{example}{Example}
\newtheorem{definition}{Definition}
\newtheorem{problem}{Problem}

\setcounter{secnumdepth}{2}
\setcounter{tocdepth}{4}


\begin{document}
\maketitle
\newpage
%%%%%%%%%%%%%%%%%%%%%%%%%%%%%%%%%%%%%%%%%%%%%%%%%%%%%%%%%%%%%%%%%%%%%%%%%%%%%%%%
%%%%%%%%%%%%%%%%%%%%%%%%%%%%%%%%%%%%%%%%%%%%%%%%%%%%%%%%%%%%%%%%%%%%%%%%%%%%%%%%
\vskip0.1in\hrule\vskip0.1in
\noindent
{\bf Types of Errors: Summary.} 
\vskip0.1in\hrule\vskip0.1in
\noindent
Weather forcasting has improved over the past several decades due to significant
increases in computer speeds and mathematical modeling techniques. Forcasts
involve predicting future events based on present data and past experience. The
accuracy of weather predictions can mean the difference between life and death
in terms of major weather events like hurricanes and tornados. If the forcast is
wrong it could mean that people will not have enough time to seek shelter. A
typical weather model will involve systems of differential equations and input
parameters based current conditions. In mathematical terms, this is called an
initial value problem. 

Problems in the real world typically involve making observations and posing
questions about the physical processes being observed. For example, wildlife
biologists may decide to fit collars equipped with GPS devices on migrating deet
to find out how many times those deer cross highways or fences as they move
from winter to summer feeding grounds. Locations for the deer are taken at 
discrete intervals to track the path deer take.  


The following is a list of sources for error that need to be taken into account
by compoutational scientists.
\begin{enumerate}
  \item {\bf Modeling Errors} These errors can occur when assumptions are made
        about the phenomena being studied. For example, one may consider a model
        of the solar system where the planets are assumed to be spheres, which
        is not the case.
  \item {\bf Measurement Errors:} These errors occur when instruments are used
        to measure physical quantities. For example, the temperature of molten
        lava might be measured to within one or two degrees based on the
        magnitude of the exact temperature. The neglected fractional part of
        the measurement would characterize the error.
  \item {\bf Discretization Error:} In order to compute solutions to real world
        mathematical problems using computers, it necessary that the model be
        finite or discretized. For example, weather models based on systems of
        partial differential equations require a discretization of the
        continuous model to fit in the discrete framework of a computer
        simulation. The discretization in this case requires computing averages
        over finite volumes of the physcial space which is a source of error.
  \item {\bf Roundoff Errors:} These errors occur due to finite precision of
        numbers on any computer. Computers are limited by the finite amount of
        memory and storage. Most real numbers require an infinite number of
        digits (at least all irrational numbers). So the numerical values we
        work with will produce errors.
  \item {\bf Arithmetic Operations on Machine Numbers:} We need to be able to
        perform binary operations like addition, subtraction, multiplication,
        and division. If we are using finite precision machine numbers, the
        output will also have errors that can accumulate.
\end{enumerate}
Of these errors, only one can be controlled by computational mathematicians.
Modeling errors are typically beyond the control of computational scientists
since the model is typically handed off to the computational scientist.
Measurements are completely dependent on the measurement instrumentation.
Roundoff error and accumulation of errors due to arithmetic are all dependent
on the computers we use. The only error computational scientists can work on is
on discretization error. We can change the methods used to improve
approximations for mathematical problems.
%%%%%%%%%%%%%%%%%%%%%%%%%%%%%%%%%%%%%%%%%%%%%%%%%%%%%%%%%%%%%%%%%%%%%%%%%%%%%%%%
%%%%%%%%%%%%%%%%%%%%%%%%%%%%%%%%%%%%%%%%%%%%%%%%%%%%%%%%%%%%%%%%%%%%%%%%%%%%%%%%
\end{document}
