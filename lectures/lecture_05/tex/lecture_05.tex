\documentclass[10pt,fleqn]{article}
\usepackage{hyperref}



\setlength{\topmargin}{-.75in}
\addtolength{\textheight}{2.00in}
\setlength{\oddsidemargin}{.00in}
\addtolength{\textwidth}{.75in}

\title{ Math 4610 Fundamentals of Computational Mathematics  - Lecture 5 \\
            \ \\
  The Root Finding Problem for Functions of One Real Variable 
  \footnote{These notes are part of an Open Resource Educational project
            sponsored by Utah State University}}

\author{Joe Koebbe}

\nofiles

\pagestyle{empty}

\setlength{\parindent}{0in}

% new math commands


\setlength{\oddsidemargin}{-0.25in}
\setlength{\evensidemargin}{-0.25in}
\setlength{\textwidth}{6.75in}
\setlength{\headheight}{0.0in}
\setlength{\topmargin}{-0.25in}
\setlength{\textheight}{9.00in}

\makeindex

\usepackage{mathrsfs}

%\usepackage[pdftex]{graphicx}
\usepackage{epstopdf}

\newcounter{beans}

\newcommand{\ds}{\displaystyle}
\newcommand{\limit}[2]{\displaystyle\lim_{#1\to#2}}

\newcommand{\binomial}[2]{\ \left( \begin{array}{c}
                                  #1 \\
                                  #2
                                 \end{array}
                            \right) \
                         }
\newcommand{\ExampleRule}[2]
  {
  \noindent
  \rule{\linewidth}{1pt}
  \begin{example}
    #1
    \label{#2}
  \end{example}
  \rule{\linewidth}{1pt}
  \vskip0.125in
  }

\newcommand{\defbox}[1]
  {
   \ \\
   \noindent
   \setlength\fboxrule{1pt}
   \fbox{
        \begin{minipage}{6.5in}
          #1
        \end{minipage}
        }
   \ \\
  }
\newcommand{\verysmallworkbox}[1]
  {
   \ \\
   \noindent
   \setlength\fboxrule{1pt}
   \fbox{
        \begin{minipage}{6.5in}
           #1
           \ \\
           \vskip0.5in \ \\
           \ \\
        \end{minipage}
        }
   \ \\
  }
\newcommand{\smallworkbox}[1]
  {
   \ \\
   \noindent
   \setlength\fboxrule{1pt}
   \fbox{
        \begin{minipage}{6.5in}
           #1
           \ \\
           \vskip2.5in \ \\
           \ \\
        \end{minipage}
        }
   \ \\
  }
\newcommand{\halfworkbox}[1]
  {
   \ \\
   \noindent
   \setlength\fboxrule{1pt}
   \fbox{
        \begin{minipage}{6.5in}
           #1 \hfill
           \ \\
           \vskip3.25in \ \\
           \ \\
        \end{minipage}
        }
   \ \\
  }
\newcommand{\largeworkbox}[1]
  {
   \ \\
   \noindent
   \setlength\fboxrule{1pt}
   \fbox{
        \begin{minipage}{6.5in}
           #1
           \ \\
           \vskip7.5in \ \\
           \ \\
        \end{minipage}
        }
   \ \\
  }
\newcommand{\flexworkbox}[2]
  {
   \ \\
   \noindent
   \setlength\fboxrule{1pt}
   \fbox{
        \begin{minipage}{6.5in}
           #1
           \ \\

           \vskip#2 \ \\
           \ \\
        \end{minipage}
        }
   \ \\
  }


% symbols for sets of numbers

\newcommand{\natnumb}{$\cal N$}
\newcommand{\whonumb}{$\cal W$}
\newcommand{\intnumb}{$\cal Z$}
\newcommand{\ratnumb}{$\cal Q$}
\newcommand{\irrnumb}{$\cal I$}
\newcommand{\realnumb}{$\cal R$}
\newcommand{\cmplxnumb}{$\cal C$}

% misc. commands

\newcommand{\mma}{{\it Mathematica}}
\newcommand{\sech}{\mbox{ sech}}
 
\newtheorem{theorem}{Theorem}
\newtheorem{example}{Example}
\newtheorem{definition}{Definition}
\newtheorem{problem}{Problem}

\setcounter{secnumdepth}{2}
\setcounter{tocdepth}{4}


\begin{document}
\maketitle
\newpage
\vskip0.1in\hrule\vskip0.1in
\noindent
The Root Finding Problem for Functions of One Real Variable 
\vskip0.1in\hrule\vskip0.1in
In this lecture we will take up one topic in how to work within the "git"
framework and more on roundoff error, problems that arise in using arithmetic
operations on machine numbers, and methods that can be used to modify algorithms
and code to mitigate some of these problems. That is, we will talk about how to
control the accumulation of roundoff errors for those sources of error that can
be addressed by computational scientists.
%%%%%%%%%%%%%%%%%%%%%%%%%%%%%%%%%%%%%%%%%%%%%%%%%%%%%%%%%%%%%%%%%%%%%%%%%%%%%%%%
%%%%%%%%%%%%%%%%%%%%%%%%%%%%%%%%%%%%%%%%%%%%%%%%%%%%%%%%%%%%%%%%%%%%%%%%%%%%%%%%
\vskip0.1in\hrule\vskip0.1in
\noindent
{\bf Content Items:}
\vskip0.1in\hrule\vskip0.1in
\noindent
There are a couple of features in the code that need to be explained.
\begin{list}{$\bullet$}{\usecounter{beans} \parsep=0pt \listparindent=0pt
\topsep=0pt \rightmargin=.35in \leftmargin=.35in \labelsep=5 pt
\itemsep=2pt}
  \item {\bf Using Version Control Systems (VCS) - git:} A brief discussion of
        \lq\lq git\rq\rq\ will be taken up in class to show how to work and
        collaborate with other students and your instructor. The \lq\lq
        git\rq\rq\ platform allows you to work on a laptop at home and then
        \lq\lq push\rq\rq\ your work and any modifications to Github where the
        instructor can get to the work.
     \href{https://jvkoebbe.github.io/math4610/lectures/lecture_05/md/git_primer}{$>$ $>$ go there}
     \href{https://jvkoebbe.github.io/math4610/lectures/lecture_05/pdf/git_primer.pdf}{(pdf)}
  \item {\bf Statement of the Root Finding Problem:} Many applied mathematics
     problems can be stated in the form of a general root finding problem. In
     this lecture the general root finding problem will be stated. Algorithms
     for the root finding problem will be presented in the next few lectures.
     \href{https://jvkoebbe.github.io/math4610/lectures/lecture_05/html/root_finding_problem.html}{$>$ $>$ go there}
     \href{https://jvkoebbe.github.io/math4610/lectures/lecture_05/pdf/root_finding_problem.pdf}{(pdf)}
  \item {\bf Wrap up and Questions:}  If there is time and anyone has questions
        about the lecture, these will be addressed.
\end{list}
%%%%%%%%%%%%%%%%%%%%%%%%%%%%%%%%%%%%%%%%%%%%%%%%%%%%%%%%%%%%%%%%%%%%%%%%%%%%%%%%
%%%%%%%%%%%%%%%%%%%%%%%%%%%%%%%%%%%%%%%%%%%%%%%%%%%%%%%%%%%%%%%%%%%%%%%%%%%%%%%%
\vskip0.1in\hrule\vskip0.1in
\noindent
\href{https://jvkoebbe.github.io/math4610/lectures/lecture_04/md/lecture_04}{[prev]}
\href{https://jvkoebbe.github.io/math4610/lectures/toc_lectures}{[toc]}
\href{https://jvkoebbe.github.io/math4610/lectures/lecture_06/md/lecture_06}{[next]}
%%%%%%%%%%%%%%%%%%%%%%%%%%%%%%%%%%%%%%%%%%%%%%%%%%%%%%%%%%%%%%%%%%%%%%%%%%%%%%%%
%%%%%%%%%%%%%%%%%%%%%%%%%%%%%%%%%%%%%%%%%%%%%%%%%%%%%%%%%%%%%%%%%%%%%%%%%%%%%%%%
\vskip0.1in\hrule\vskip0.1in
\end{document}





In this part of the notes, a brief primer for {\bf git} that will help you get
started using repositories locally. You will also learn to clone and pull
repositories from Github to work on existing repositories.
%%%%%%%%%%%%%%%%%%%%%%%%%%%%%%%%%%%%%%%%%%%%%%%%%%%%%%%%%%%%%%%%%%%%%%%%%%%%%%%%
%%%%%%%%%%%%%%%%%%%%%%%%%%%%%%%%%%%%%%%%%%%%%%%%%%%%%%%%%%%%%%%%%%%%%%%%%%%%%%%%
\vskip0.1in\hrule\vskip0.1in
\noindent
In order to efficiently use {\bf git} you will need to be working in a command
terminal. You can use Cygwin or the command windows in Windows or on your Apple
computer/laptop. This has already been covered in previous lectures. So, open
a terminal and at the prompt, type the following version of the \lq\lq\ 
which\rq\rq\ command.
\begin{verbatim}

     % which git

\end{verbatim}
The reason for doing this is to determine if {\bf git} is installed on your
computer. If so, you can proceed and if not, you will need to install {\bf git}
on your computer or use the computers in the Engineering lab. Note that there
are a number of ways to install and access the software.
%%%%%%%%%%%%%%%%%%%%%%%%%%%%%%%%%%%%%%%%%%%%%%%%%%%%%%%%%%%%%%%%%%%%%%%%%%%%%%%%
%%%%%%%%%%%%%%%%%%%%%%%%%%%%%%%%%%%%%%%%%%%%%%%%%%%%%%%%%%%%%%%%%%%%%%%%%%%%%%%%
\vskip0.1in\hrule\vskip0.1in
\noindent
Assuming {\bf git} is installed, in the command terminal make a temporary
folder using
\begin{verbatim}

     % mkdir gitexample

\end{verbatim}
to create a temporary place to work. Then change directories and look at what is
in the folder.
\begin{verbatim}

     % cd gitexample
     % ls

\end{verbatim}
The folder should (but does not need to be) empty. Next, we will initialize a
repository in the folder. There are lots of options and flags that can be used
with git. We will just use a few in this primer. So, type
\begin{verbatim}

     % git init

\end{verbatim}
The command only takes a second or two and will identify the folder as a
repository folder. The output of the command will look something like the
following.
\begin{verbatim}

     Initialized empty Git repository in /cygdrive/m/gitexample/.git/

\end{verbatim}
Note that the path shown for the folder is dependent on your computer and where
you are doing this work.
%%%%%%%%%%%%%%%%%%%%%%%%%%%%%%%%%%%%%%%%%%%%%%%%%%%%%%%%%%%%%%%%%%%%%%%%%%%%%%%%
%%%%%%%%%%%%%%%%%%%%%%%%%%%%%%%%%%%%%%%%%%%%%%%%%%%%%%%%%%%%%%%%%%%%%%%%%%%%%%%%
\vskip0.1in\hrule\vskip0.1in
\noindent
To see what has been put in the folder you can use the command
\begin{verbatim}

     % ls

\end{verbatim}
Unless you have options set, the folder will still look empty. So, instead, type
\begin{verbatim}

     % ls -a

\end{verbatim}
to display the hidden files. The command will show a subfolder named {\bf .git}
that contains all of the repository bookkeeping for the repository. You never
really need to know the contents of this folder and it is highly recommended
that the contents are not modified. At least meke sure you know what you are
doing if you choose to poke around in there.
%%%%%%%%%%%%%%%%%%%%%%%%%%%%%%%%%%%%%%%%%%%%%%%%%%%%%%%%%%%%%%%%%%%%%%%%%%%%%%%%
%%%%%%%%%%%%%%%%%%%%%%%%%%%%%%%%%%%%%%%%%%%%%%%%%%%%%%%%%%%%%%%%%%%%%%%%%%%%%%%%
\vskip0.1in\hrule\vskip0.1in
\noindent
Now, let's put a file in the folder and see what happens in {\bf git}. Type the
command
\begin{verbatim}

     % touch hello.f

\end{verbatim}
This command creates an empty file that can be modified and worked with. Before
modifying the file, type in the {\bf git} command
\begin{verbatim}

     % git status

\end{verbatim}
to determine how things are accounted for in the folder. The output from the
status command is the following.
\begin{verbatim}

     On branch master

     No commits yet

     Untracked files:
       (use "git add <file>..." to include in what will be committed)

             hello.f

     nothing added to commit but untracked files present (use "git add" to track)

\end{verbatim}
The output shows that a file is waiting to be included in the repository. To
include the file created.
%%%%%%%%%%%%%%%%%%%%%%%%%%%%%%%%%%%%%%%%%%%%%%%%%%%%%%%%%%%%%%%%%%%%%%%%%%%%%%%%
%%%%%%%%%%%%%%%%%%%%%%%%%%%%%%%%%%%%%%%%%%%%%%%%%%%%%%%%%%%%%%%%%%%%%%%%%%%%%%%%
\vskip0.1in\hrule\vskip0.1in
\noindent
To add a file to the repository, the {\bf git} {\bf add} and {\bf commit}
command are used to do the work.  That is,
\begin{verbatim}

     % git add hello.f

     % git commit -a

\end{verbatim}
During the execution of the commit command, an editor session is started. You
must include a comment to the commit to have the command complete the work. All
you need is a short comment like The -a tag is used to include all commits that
are listed.

The commit command has lots of options for being selective in how to add files
and folders. The output looks like the following.
\begin{verbatim}

     % adding hello.f to the repository

\end{verbatim}
The output after the editor comment is entered is something like the following.
\begin{verbatim}

     [master (root-commit) 6a6297f] aaaaaa
      1 file changed, 0 insertions(+), 0 deletions(-)
      create mode 100644 hello.f

\end{verbatim}
This indicates a single file has been included. In the development of code and
documentation the commands above are about all you will need to use {\bf git}.
However, there is a lot more to the {\bf git} environment. Also, note that the
exact form of the output will change slightly due to the fact you are adding
different bits of content on different machines.
%%%%%%%%%%%%%%%%%%%%%%%%%%%%%%%%%%%%%%%%%%%%%%%%%%%%%%%%%%%%%%%%%%%%%%%%%%%%%%%%
%%%%%%%%%%%%%%%%%%%%%%%%%%%%%%%%%%%%%%%%%%%%%%%%%%%%%%%%%%%%%%%%%%%%%%%%%%%%%%%%
\vskip0.1in\hrule\vskip0.1in
\noindent
For example, your entire repository can be pushed up to Github or other VCS
sites. We will come back to this. Let's do one more example before continuing.
Create a folder called src in the repository.
\begin{verbatim}

     % mkdir src

\end{verbatim}
Next, move the file into the folder.
\begin{verbatim}

     % mv hello.f src

\end{verbatim}
Move into the folder using
\begin{verbatim}

     % cd src

\end{verbatim}
and then edit the file to do the hello world example. That is, using
\begin{verbatim}

     % vim hello.f

\end{verbatim}
and edit in the lines as shown below.
\begin{verbatim}

           program main
           print *, "hello world"
           stop
           end
     ~
     ~
     ~
     ~
     ~
     ~
     ~
     ~
     ~
     ~
     ~
     ~
     ~
     ~
     ~
     ~
     ~
     ~
     ~

\end{verbatim}
Save the file and then compile the file. That is,
\begin{verbatim}

     % gfortran hello.f

\end{verbatim}
Note that another file will be created named {\bf a.exe} that can be executed
as in earlier lectures. Now that we have done a little work, we can use the
status command to see how things have changed. Use
\begin{verbatim}

     % git status

\end{verbatim}
which results in
\begin{verbatim}

     On branch master
     Changes not staged for commit:
       (use "git add/rm <file>..." to update what will be committed)
       (use "git checkout -- <file>..." to discard changes in working directory)

             deleted:    ../hello.f

     Untracked files:
       (use "git add <file>..." to include in what will be committed)

             ./

     no changes added to commit (use "git add" and/or "git commit -a")

\end{verbatim}
It should be noted that the git VCS can be invoked in any folder within the
repository folder. This allows you to continue to work without going back to the
root folder for the repository.

The result indicates that we need to add the current folder. So, type
\begin{verbatim}

     % git add ./

\end{verbatim}
Finally, commit the changes using
\begin{verbatim}

     % git commit -a

\end{verbatim}
and adding a comment in the editor as above. Once this is done the work is now
committed to the repository. It always is a good idea to use the status command
to make sure everything has been included or excluded.
%%%%%%%%%%%%%%%%%%%%%%%%%%%%%%%%%%%%%%%%%%%%%%%%%%%%%%%%%%%%%%%%%%%%%%%%%%%%%%%%
%%%%%%%%%%%%%%%%%%%%%%%%%%%%%%%%%%%%%%%%%%%%%%%%%%%%%%%%%%%%%%%%%%%%%%%%%%%%%%%%
\vskip0.1in\hrule\vskip0.1in
\noindent
The second part of this lesson involves cloning a repository from Github. There
are several steps that need to be taken care of first. There are a couple of
configuration parameters that need to be set. First move to a clean directory -
say we name it tempWork. We do not want to perform an initialization for the
repository. Instead, we will need to configure some things in {\bf git}. In
particular, you will need to configure the user name and email. So, type
\begin{verbatim}

     % git config user.name yourChoice

\end{verbatim}
and
\begin{verbatim}

     % git config user.email yourChoice@some.email.isp

\end{verbatim}
You will need to fill in the names. Note that the user name is something you can
choose. As your instructor, I would suggest something simple and all in lower
case letters. It is a bit easier to remember this. For the email, choose an
email address you read most often. That way you can monitor what, when, and
where things are happening when you work with either {\bf git} locally, or
with {\bf Github} out in the real world.
%%%%%%%%%%%%%%%%%%%%%%%%%%%%%%%%%%%%%%%%%%%%%%%%%%%%%%%%%%%%%%%%%%%%%%%%%%%%%%%%
%%%%%%%%%%%%%%%%%%%%%%%%%%%%%%%%%%%%%%%%%%%%%%%%%%%%%%%%%%%%%%%%%%%%%%%%%%%%%%%%
\vskip0.1in\hrule\vskip0.1in
\noindent
Once you have the configuration step done, you are ready to clone a repository
locally. So, start by changing to a working directory. For example, you can
create a directory, say
\begin{verbatim}

     % mkdir tempWork

\end{verbatim}
and then
\begin{verbatim}

     % cd tempWork

\end{verbatim}
Now for the cloning. Type in
\begin{verbatim}

     % git clone https://www.github.com/Github_username/Github_repository_name

\end{verbatim}
When the command is launched, the repository will be created locally and then
will clone the entire contents of the repository you have chosen. For students
in Math 4610, it would be a good idea to clone the \lq\lq\ math4610\rq\rq
repository that you have started for the course. The command for the Math 4610
repository should look like
\begin{verbatim}

     % git clone https://www.github.com/Github_username/math4610

\end{verbatim}
The result will be a repository that you can work on locally.
%%%%%%%%%%%%%%%%%%%%%%%%%%%%%%%%%%%%%%%%%%%%%%%%%%%%%%%%%%%%%%%%%%%%%%%%%%%%%%%%
%%%%%%%%%%%%%%%%%%%%%%%%%%%%%%%%%%%%%%%%%%%%%%%%%%%%%%%%%%%%%%%%%%%%%%%%%%%%%%%%
\vskip0.1in\hrule\vskip0.1in
\noindent
The last bit is to make sure you know how to make changes either locally or on
your github account and make sure the local repository and the Github repository
are exactly the same. If you make changes locally on your copy of the repository
you should first add and remove any files using
\begin{verbatim}

     % git add ...
     % git remove ....

\end{verbatim}
following by a commit
\begin{verbatim}

     % git commit -a

\end{verbatim}
Then the big step is to use the {\bf push} command. The syntax is the following.
\begin{verbatim}

     % git push

\end{verbatim}
You will be prompted for your user name (on Github) and then your password.
The command will then proceed to merge your local work with the repository on
Github. If you modify your repository on Github, you pull content to the local
repository using
\begin{verbatim}

     % git pull

\end{verbatim}
This makes sure change on Github are reflected on the local repository.  Note
that your will again be prompted for your user name and password on Github.
%%%%%%%%%%%%%%%%%%%%%%%%%%%%%%%%%%%%%%%%%%%%%%%%%%%%%%%%%%%%%%%%%%%%%%%%%%%%%%%%
%%%%%%%%%%%%%%%%%%%%%%%%%%%%%%%%%%%%%%%%%%%%%%%%%%%%%%%%%%%%%%%%%%%%%%%%%%%%%%%%
\vskip0.1in\hrule\vskip0.1in
\noindent
Students should note that there are a lot of git command with tons of options
to do all kinds of modifications to your repositories. To find out what is
available type
\begin{verbatim}

     % git

\end{verbatim}
for a list of commands and options.
The output from the command looks like the following.
\begin{verbatim}

     usage: git [--version] [--help] [-C <path>] [-c <name>=<value>]
                [--exec-path[=<path>]] [--html-path] [--man-path] [--info-path]
                [-p | --paginate | --no-pager] [--no-replace-objects] [--bare]
                [--git-dir=<path>] [--work-tree=<path>] [--namespace=<name>]
                <command> [<args>]

     These are common Git commands used in various situations:

     start a working area (see also: git help tutorial)
        clone      Clone a repository into a new directory
        init       Create an empty Git repository or reinitialize an existing one

     work on the current change (see also: git help everyday)
        add        Add file contents to the index
        mv         Move or rename a file, a directory, or a symlink
        reset      Reset current HEAD to the specified state
        rm         Remove files from the working tree and from the index

     examine the history and state (see also: git help revisions)
        bisect     Use binary search to find the commit that introduced a bug
        grep       Print lines matching a pattern
        log        Show commit logs
        show       Show various types of objects
        status     Show the working tree status

     grow, mark and tweak your common history
        branch     List, create, or delete branches
        checkout   Switch branches or restore working tree files
        commit     Record changes to the repository
        diff       Show changes between commits, commit and working tree, etc
        merge      Join two or more development histories together
        rebase     Reapply commits on top of another base tip
        tag        Create, list, delete or verify a tag object signed with GPG

     collaborate (see also: git help workflows)
        fetch      Download objects and refs from another repository
        pull       Fetch from and integrate with another repository or a local branch
        push       Update remote refs along with associated objects

     'git help -a' and 'git help -g' list available subcommands and some
     concept guides. See 'git help <command>' or 'git help <concept>'
     to read about a specific subcommand or concept.

\end{verbatim}
There are also a wealth of on-line resources and books that you can get if you
intend to do a lot more with this computational utility.
