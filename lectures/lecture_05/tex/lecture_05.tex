\documentclass[10pt,fleqn]{article}
%\usepackage{graphicx}


\setlength{\topmargin}{-.75in}
\addtolength{\textheight}{2.00in}
\setlength{\oddsidemargin}{.00in}
\addtolength{\textwidth}{.75in}

\title{Math 4610 Lecture Notes \\
            \ \\
       Using Git to Work Locally
  \footnote{These notes are part of an Open Resource Educational project
            sponsored by Utah State University}}

\author{Joe Koebbe}

\nofiles

\pagestyle{empty}

\setlength{\parindent}{0in}

% new math commands


\setlength{\oddsidemargin}{-0.25in}
\setlength{\evensidemargin}{-0.25in}
\setlength{\textwidth}{6.75in}
\setlength{\headheight}{0.0in}
\setlength{\topmargin}{-0.25in}
\setlength{\textheight}{9.00in}

\makeindex

\usepackage{mathrsfs}

%\usepackage[pdftex]{graphicx}
\usepackage{epstopdf}

\newcounter{beans}

\newcommand{\ds}{\displaystyle}
\newcommand{\limit}[2]{\displaystyle\lim_{#1\to#2}}

\newcommand{\binomial}[2]{\ \left( \begin{array}{c}
                                  #1 \\
                                  #2
                                 \end{array}
                            \right) \
                         }
\newcommand{\ExampleRule}[2]
  {
  \noindent
  \rule{\linewidth}{1pt}
  \begin{example}
    #1
    \label{#2}
  \end{example}
  \rule{\linewidth}{1pt}
  \vskip0.125in
  }

\newcommand{\defbox}[1]
  {
   \ \\
   \noindent
   \setlength\fboxrule{1pt}
   \fbox{
        \begin{minipage}{6.5in}
          #1
        \end{minipage}
        }
   \ \\
  }
\newcommand{\verysmallworkbox}[1]
  {
   \ \\
   \noindent
   \setlength\fboxrule{1pt}
   \fbox{
        \begin{minipage}{6.5in}
           #1
           \ \\
           \vskip0.5in \ \\
           \ \\
        \end{minipage}
        }
   \ \\
  }
\newcommand{\smallworkbox}[1]
  {
   \ \\
   \noindent
   \setlength\fboxrule{1pt}
   \fbox{
        \begin{minipage}{6.5in}
           #1
           \ \\
           \vskip2.5in \ \\
           \ \\
        \end{minipage}
        }
   \ \\
  }
\newcommand{\halfworkbox}[1]
  {
   \ \\
   \noindent
   \setlength\fboxrule{1pt}
   \fbox{
        \begin{minipage}{6.5in}
           #1 \hfill
           \ \\
           \vskip3.25in \ \\
           \ \\
        \end{minipage}
        }
   \ \\
  }
\newcommand{\largeworkbox}[1]
  {
   \ \\
   \noindent
   \setlength\fboxrule{1pt}
   \fbox{
        \begin{minipage}{6.5in}
           #1
           \ \\
           \vskip7.5in \ \\
           \ \\
        \end{minipage}
        }
   \ \\
  }
\newcommand{\flexworkbox}[2]
  {
   \ \\
   \noindent
   \setlength\fboxrule{1pt}
   \fbox{
        \begin{minipage}{6.5in}
           #1
           \ \\

           \vskip#2 \ \\
           \ \\
        \end{minipage}
        }
   \ \\
  }


% symbols for sets of numbers

\newcommand{\natnumb}{$\cal N$}
\newcommand{\whonumb}{$\cal W$}
\newcommand{\intnumb}{$\cal Z$}
\newcommand{\ratnumb}{$\cal Q$}
\newcommand{\irrnumb}{$\cal I$}
\newcommand{\realnumb}{$\cal R$}
\newcommand{\cmplxnumb}{$\cal C$}

% misc. commands

\newcommand{\mma}{{\it Mathematica}}
\newcommand{\sech}{\mbox{ sech}}
 
\newtheorem{theorem}{Theorem}
\newtheorem{example}{Example}
\newtheorem{definition}{Definition}
\newtheorem{problem}{Problem}

\setcounter{secnumdepth}{2}
\setcounter{tocdepth}{4}


\begin{document}
\maketitle
\newpage
%%%%%%%%%%%%%%%%%%%%%%%%%%%%%%%%%%%%%%%%%%%%%%%%%%%%%%%%%%%%%%%%%%%%%%%%%%%%%%%%
%%%%%%%%%%%%%%%%%%%%%%%%%%%%%%%%%%%%%%%%%%%%%%%%%%%%%%%%%%%%%%%%%%%%%%%%%%%%%%%%
\vskip0.1in\hrule\vskip0.1in
\noindent
{\bf Math 4610 Contents: Using Git to Work Locally.} 
\vskip0.1in\hrule\vskip0.1in
\noindent
In this part of the notes, a brief primer for {\bf git} that will help you get
started using repositories locally. You will also learn to clone and pull
repositories from Github to work on existing repositories.
%%%%%%%%%%%%%%%%%%%%%%%%%%%%%%%%%%%%%%%%%%%%%%%%%%%%%%%%%%%%%%%%%%%%%%%%%%%%%%%%
%%%%%%%%%%%%%%%%%%%%%%%%%%%%%%%%%%%%%%%%%%%%%%%%%%%%%%%%%%%%%%%%%%%%%%%%%%%%%%%%
\vskip0.1in\hrule\vskip0.1in
\noindent
In order to efficiently use {\bf git} you will need to be working in a command
terminal. You can use Cygwin or the command windows in Windows or on your Apple
computer/laptop. This has already been covered in previous lectures. So, open
a terminal and at the proompt, type the following command.
\begin{verbatim}

     % which git

\end{verbatim}
The reason for doing this is to determine if {\bf git} is installed on your
computer. If so, we can proceed and if not, you will need to install {\bf git}
on your compouter or use the computers in the Engineering lab.
%%%%%%%%%%%%%%%%%%%%%%%%%%%%%%%%%%%%%%%%%%%%%%%%%%%%%%%%%%%%%%%%%%%%%%%%%%%%%%%%
%%%%%%%%%%%%%%%%%%%%%%%%%%%%%%%%%%%%%%%%%%%%%%%%%%%%%%%%%%%%%%%%%%%%%%%%%%%%%%%%
\vskip0.1in\hrule\vskip0.1in
\noindent
Assuming {\bf git} is installed, in the command terminal make a temporary
folder using
\begin{verbatim}

     % mkdir gitexample

\end{verbatim}
to create a temporary place to work. Then change directories and look at what is
in the folder.
\begin{verbatim}

     % cd gitexample
     % ls

\end{verbatim}
The folder should (but does not need to be) empty. Next, we will initialize a
repository in the folder. There are lots of options and flags that can be used
with git. We will just use a few in this primer. So, type
\begin{verbatim}

     % git init

\end{verbatim}
The command only takes a second or two and will identify the folder as a
repository. The output of the command will look something like the following.
\begin{verbatim}

     Initialized empty Git repository in /cygdrive/m/gitexample/.git/

\end{verbatim}
Note that the path shown for the folder is dependent on your computer and where
you are doing this work.
%%%%%%%%%%%%%%%%%%%%%%%%%%%%%%%%%%%%%%%%%%%%%%%%%%%%%%%%%%%%%%%%%%%%%%%%%%%%%%%%
%%%%%%%%%%%%%%%%%%%%%%%%%%%%%%%%%%%%%%%%%%%%%%%%%%%%%%%%%%%%%%%%%%%%%%%%%%%%%%%%
\vskip0.1in\hrule\vskip0.1in
\noindent
To see what has been put in the folder you can use the command
\begin{verbatim}

     % ls

\end{verbatim}
Unless you have options set, the folder will still look empty. So, instead, type
\begin{verbatim}

     % ls -a

\end{verbatim}
to display the hidden files. The command will show a subfolder named {\bf .git}
that contains all of the repository bookkeeping for the repository. You never
really need to know the contents of this folder and it is highly recommended
that the contents are not modified. At least meke sure you know what you are
doing in there.
%%%%%%%%%%%%%%%%%%%%%%%%%%%%%%%%%%%%%%%%%%%%%%%%%%%%%%%%%%%%%%%%%%%%%%%%%%%%%%%%
%%%%%%%%%%%%%%%%%%%%%%%%%%%%%%%%%%%%%%%%%%%%%%%%%%%%%%%%%%%%%%%%%%%%%%%%%%%%%%%%
\vskip0.1in\hrule\vskip0.1in
\noindent
Now, let's put a file in the folder and see what happens in {\bf git}. Type the
command
\begin{verbatim}

     % touch hello.f

\end{verbatim}
This command creates an empty file that can be modified and worked with. Before
modifying the file, type in the {\bf git} command
\begin{verbatim}

     % git status

\end{verbatim}
to determine how things are accounted for in the folder. The output from the
status command is the following.
\begin{verbatim}

     On branch master

     No commits yet

     Untracked files:
       (use "git add <file>..." to include in what will be committed)

             hello.f

     nothing added to commit but untracked files present (use "git add" to track)

\end{verbatim}
The output shows that a file is waiting to be included in the repository. To
include the file created.
%%%%%%%%%%%%%%%%%%%%%%%%%%%%%%%%%%%%%%%%%%%%%%%%%%%%%%%%%%%%%%%%%%%%%%%%%%%%%%%%
%%%%%%%%%%%%%%%%%%%%%%%%%%%%%%%%%%%%%%%%%%%%%%%%%%%%%%%%%%%%%%%%%%%%%%%%%%%%%%%%
\vskip0.1in\hrule\vskip0.1in
\noindent
To add a file to the repository, the {\bf git} add and commit command are used
to do the work.  That is,
\begin{verbatim}

     % git add hello.f

     % git commit -a

\end{verbatim}
The -a tag is used to include all commits that are listed. The commit command
has lots of options for being selective in how to add files and folders. The
output looks like the following. During the execution of the commit command, 
an editor session is started. You must include a comment to the commit to have
the command complete the work. All you need is a short comment like
\begin{verbatim}

     % adding hello.f to the repository

\end{verbatim}
The output after the editor comment is entered is the following.
\begin{verbatim}

     [master (root-commit) 6a6297f] aaaaaa
      1 file changed, 0 insertions(+), 0 deletions(-)
      create mode 100644 hello.f

\end{verbatim}
This indicates a single file has been included. In the development of code and
documentation the commands above are about all you will need to use {\bf git}.
However, there is a lot more to the {\bf git} environment.
%%%%%%%%%%%%%%%%%%%%%%%%%%%%%%%%%%%%%%%%%%%%%%%%%%%%%%%%%%%%%%%%%%%%%%%%%%%%%%%%
%%%%%%%%%%%%%%%%%%%%%%%%%%%%%%%%%%%%%%%%%%%%%%%%%%%%%%%%%%%%%%%%%%%%%%%%%%%%%%%%
\vskip0.1in\hrule\vskip0.1in
\noindent
For example, your entire repository can be pushed up to Github or other VCS
sites. We will come back to this. Let's do one more example before continuing.
Create a folder called src in the repository.
\begin{verbatim}

     % mkdir src

\end{verbatim}
Next, move the file into the folder.
\begin{verbatim}

     % mv hello.f src

\end{verbatim}
Move into the folder using
\begin{verbatim}

     % cd src

\end{verbatim}
and then edit the file to do the hello world example. That is, using
\begin{verbatim}

     % vim hello.f

\end{verbatim}
and edit in the lines
\begin{verbatim}

           program main
           print *, "hello world"
           stop
           end
     ~
     ~
     ~
     ~
     ~
     ~
     ~
     ~
     ~
     ~
     ~
     ~
     ~
     ~
     ~
     ~
     ~
     ~
     ~

\end{verbatim}
Save the file and then compile the file. That is,
\begin{verbatim}

     % gfortran hello.f

\end{verbatim}
Note that another file will be created named {\bf a.exe} that can be executed
as in earlier lectures. Now that we have done a little work, we can use the
status command to see how things have changed. Use
\begin{verbatim}

     % git status

\end{verbatim}
which results in
\begin{verbatim}

On branch master
Changes not staged for commit:
  (use "git add/rm <file>..." to update what will be committed)
  (use "git checkout -- <file>..." to discard changes in working directory)

        deleted:    ../hello.f

Untracked files:
  (use "git add <file>..." to include in what will be committed)

        ./

no changes added to commit (use "git add" and/or "git commit -a")

\end{verbatim}
The result indicates that we need to add the current folder. So, type
\begin{verbatim}

     % git add ./

\end{verbatim}
Finally, commit the changes using
\begin{verbatim}

     % git commit -a

\end{verbatim}
and adding a comment in the editor as above. Once this is done the work is now
committed to the repository. It always is a good idea to use the status command
to make sure everything has been included or excluded.
\vskip0.1in\hrule\vskip0.1in
%%%%%%%%%%%%%%%%%%%%%%%%%%%%%%%%%%%%%%%%%%%%%%%%%%%%%%%%%%%%%%%%%%%%%%%%%%%%%%%%
%%%%%%%%%%%%%%%%%%%%%%%%%%%%%%%%%%%%%%%%%%%%%%%%%%%%%%%%%%%%%%%%%%%%%%%%%%%%%%%%
\vskip0.1in\hrule\vskip0.1in
\noindent
The second part of this lesson involves cloning a repository from Github. There
are several steps that need to be taken care of first. There are a couple of
configuration parameters that need to be set. First move to a directory

\end{document}







There are two topics that will be covered in this lecture.  These are how to use
{\bf git} to create and work with repositories locally. The second topic
involves development of algorithms to compute the roots of a real-valued
function of a single independent variable. The most powerful property of the
{\bf git} command platform allows users to create, develop, and modify the
contents of repositories and then communicate the work with the outside world
using Github or some other hosting Version Control System (VCS). This will
allow users rapid deployment of all kinds of computational materials.
\vskip0.1in\hrule\vskip0.1in
\noindent
The part of the lecture dedicated to the solution of the root finding problem
marks the first algorithm presented in class. The root finding problem can be
stated as follows. Find a real number, $x$, such that a given function $f(x)$
is zero at that point. The reader should note that this is a very common problem
in almost every scientific discipline. In these lectures, only a small number of
all possible methods will be covered. In addtion, some example problems will be
presented from various applications in sicence and engineering.














\vskip0.1in\hrule\vskip0.1in
\noindent
{\bf Content Items:}
\begin{list}{$\bullet$}{\usecounter{beans} \parsep=0pt \listparindent=0pt
\topsep=0pt \rightmargin=.35in \leftmargin=.35in \labelsep=5 pt
\itemsep=2pt}
  \item {\bf Building a Shared Library for Reusing Code:} (cont) An important
     skill that computational mathematicians should learn is to take relatively
     small codes and turn these into reusable object files. These can be
     collected into a shared library. A shared library can be linked to by other
     programs from this point on. Note that the equivalent of a shared library
     in a Windows setting is called a DLL.

       [> > go there](https://jvkoebbe.github.io/math4610/lectures/lecture_04/md/shared_library_example)
       | [(pdf)](https://jvkoebbe.github.io/math4610/lectures/lecture_04/pdf/shared_library_example.pdf)

  \item {\bf Documentation of Reusable Code:}  Each student will need to write a
     software manual for the algorithms implemented in class. A template for the
     software manual has been created by your instructor using Jekyll on Github.
     Students are required to use the template and can look at the template and
     download a copy of the template for their own use.

       [> > go there](https://jvkoebbe.github.io/math4610/lectures/lecture_04/md/softwaremanual_example)
       | [(md)](https://jvkoebbe.github.io/math4610/lectures/lecture_02/md/softwaremanual_example.md)

  \item {\bf Link to the Software Manual Template:} The link for the software
     maual is:

     [link](https://jvkoebbe.github.io/math4610/softwareManual/softwareManualTemplate.md)

  \item {\bf Wrap up and Questions:}  If there is time and anyone has questions
     about the lecture, these will be addressed.

\end{list}

---

[prev](https://jvkoebbe.github.io/math4610/lectures/lecture_03/md/lecture_03) |
[toc](https://jvkoebbe.github.io/math4610/lectures/toc_lectures) |
[next](https://jvkoebbe.github.io/math4610/lectures/lecture_05/md/lecture_05)

---

\end{document}



\begin{verbatim}

  1. Explicit Euler method and error

\end{verbatim}



Representation of round off error and the analysis of how errors accumulate in
complex computations. One way to proceed is through interval analysis applied to
numbers that are approximated in machine arithmetic. A simple 
example of how to build a shared library will also be discussed. This is an
important skill to have in computational mathematics.
     homework repository that each student must have to complete homework. The
     homework repository should be a private repository with you instructor as
     the only collaborator. Homework will be graded by first cloning the
     repository and then your constructor can create a branch for grading. This
     can be repeated by creating a pull request on the repository.
     [> > go there](https://www.github.io/jvkoebbe/math4610/lectures/lecture_02/homework_structure)
     | [(pdf)](https://jvkoebbe.github.io/math4610/lecture_02/pdf/homework.pdf)

        \href{https://jvkoebbe.github.io/math4610/syllabus/md/syllabus}

  \item
        \href{https://jvkoebbe.github.io/math4610/lectures/lecture_01/pdf/cygwin_primer.pdf}

  \item {\bf Using Version Control Systems (VCS) - git:} A brief discussion of
     \lq\lq git\rq\rq\ will be taken up in class to show how to work and
     collaborate with other students and your instructor. The \lq\lq git\rq\rq\ 
     platform allows you to work on a laptop at home and then \lq\lq push\rq\rq\ 
     your work and any modifications to Github where the instructor can get to
     the work.
       [> > go there](https://jvkoebbe.github.io/math4610/lectures/lecture_02/md/git_primer)
       | [(pdf)](https://jvkoebbe.github.io/math4610/lecture_02/pdf/git_primer.pdf)
  \item {\bf Code for Determining Machine Precision:} A simple algorithm for
     determining the precision for number representation will be covered. Two
     different versions of the code will be presented that determine single and
     double precision for your computer. Note that these two codes will be used
     to show students how to build a shared library.
       [> > go there](https://jvkoebbe.github.io/math4610/lecture_02/html/finite_precision)
       | [(pdf)](https://jvkoebbe.github.io/math4610/lecture_02/pdf/git_primer.pdf)

  \item {\bf Building a Shared Library for Reusing Code:} In important skill
     that computational mathematicians should learn is the ability to take
     relatively small codes and turning these into reusable object files and
     then collecting the object files into a shared library. 
       [> > go there](https://jvkoebbe.github.io/math4610/lecture_02/html/finite_precision)
       | [(pdf)](https://jvkoebbe.github.io/math4610/lecture_02/pdf/git_primer.pdf)
        \href{https://jvkoebbe.github.io/math4610/lectures/lecture_01/pdf/github_primer.pdf}

  \item {\bf Testing the Codes in a Shared Library:} We will go over how to
     write a code that uses the object routines in a shared library. Linking a
     main code to a library will be covered.
       [> > go there](https://jvkoebbe.github.io/math4610/lecture_02/html/finite_precision)
       | [(pdf)](https://jvkoebbe.github.io/math4610/lecture_02/pdf/git_primer.pdf)

  \item {\bf Floating Point Representation of Numbers on Computers:} The IEEE
     standard for number representation will be covered. Number precision for
     32-bit and 64-bit systems will be covered. 

  \item {\bf Wrap up and Questions:} If there is time and anyone has questions
        about the lecture, these will be addressed.

\end{list}
\vskip0.1in\hrule\vskip0.1in
%%%%%%%%%%%%%%%%%%%%%%%%%%%%%%%%%%%%%%%%%%%%%%%%%%%%%%%%%%%%%%%%%%%%%%%%%%%%%%%%
%%%%%%%%%%%%%%%%%%%%%%%%%%%%%%%%%%%%%%%%%%%%%%%%%%%%%%%%%%%%%%%%%%%%%%%%%%%%%%%%
\end{document}
# Math 4610 Fundamentals of Computational Mathematics  - Lecture 3

The accumulation of round off error can be a serious problem in computational
mathematics. The idea in this lecture is to focus on errors that occur in every
arithmetic operation performed by a computer.
Some simple examples of how to
mitigate inexact arithmetic interval analysis will
be covered. Documentation of algorithms and computer codes is essential to any
successful attempt to write reusable code. Finally, we will introduce the
general mathematical problem of finding the roots of a nonlinear function of a
single variable.

<hr>

## Content Items:

  1. **Building a Shared Library for Reusing Code:** An important skill that
     computational mathematicians should learn is to take relatively small codes
     and turn these into reusable object files. These can be collected into a
     shared library. A shared library can be linked to by other programs from
     this point on. Note that the equivalent of a shared library in a Windows
     setting is called a DLL.
       [> > go there](https://jvkoebbe.github.io/math4610/lecture_03/md/shared_library_example)
       | [(pdf)](https://jvkoebbe.github.io/math4610/lecture_03/pdf/git_primer.pdf)

  2. **Roundoff Errors and Arithmetic Operations:** - Every computation done
     combines two numbers that in general are only approximations of the exact
     values. The errors in some cases result in catastrophic cancellation that
     result in output that is garbage. We will look at the sum, difference,
     product, and ratio of numbers. Note that all operations involve some
     combination of these four operations.
     [> > go there](https://jvkoebbe.github.io/math4610/lectures/lecture_02/md/precision_00)
     | [(pdf)](https://jvkoebbe.github.io/math4610/lectures/lecture_02/pdf/precision_00.pdf)

     * [**Intro:**](https://jvkoebbe.github.io/math4610/lectures/lecture_02/md/precision_00)
     * [**Part 1:**](https://jvkoebbe.github.io/math4610/lectures/lecture_02/html/precision_01.html)
     * [**Part 2:**](https://jvkoebbe.github.io/math4610/lectures/lecture_02/html/precision_02.html)
     * [**Part 3:**](https://jvkoebbe.github.io/math4610/lectures/lecture_02/html/precision_03.html)
     * [**Part 4:**](https://jvkoebbe.github.io/math4610/lectures/lecture_02/html/precision_04.html)

  3. **Testing the Codes in a Shared Library:** We will go over how to write a
     code that uses the object routines in a shared library. Linking a main
     code to a library will be covered.
       [> > go there](https://jvkoebbe.github.io/math4610/lecture_03/html/finite_precision)
       | [(pdf)](https://jvkoebbe.github.io/math4610/lecture_03/pdf/git_primer.p

  4. **Analytic Fixes for Cancellation Errors:** In some cases, the arithmetic
     operations needed in an algorithm can be rearranged through the use of
     some mathematical manipulation to obtain an equivalent algorithm that is
     more stable. A couple of examples will be treated.
       [> > go there](https://jvkoebbe.github.io/math4610/lectures/lecture_02/md/git_primer)
       | [(pdf)](https://jvkoebbe.github.io/math4610/lecture_02/pdf/git_primer.pdf)

  5. **An Introduction to Interval Analysis:** Interval analysis involves doing
     arithmetic on intervals instead of numbers. This will be made a bit more
     clear in the lecture. Sums, differences, products, and quotients are all
     treated within the structre of intervals. 
       [> > go there](https://jvkoebbe.github.io/math4610/lecture_02/html/finite_precision)
       | [(pdf)](https://jvkoebbe.github.io/math4610/lecture_02/pdf/git_primer.pdf)

  4. **Documentation of Reusable Code:**  Each student will need to write a
     software manual for the algorithms implemented in class. A template for the
     software manual has been created by your instructor using Jekyll on Github.
     Students are required to use the template and can look at the template and
     download a copy of the template for their own use.
       [> > go there](https://jvkoebbe.github.io/math4610/lecture_02/html/finite_precision)
       | [(pdf)](https://jvkoebbe.github.io/math4610/lecture_02/pdf/git_primer.pdf)

  5. **Statement of the Root Finding Problem:**  Many applied mathematics
     problems can be stated in the form of a general root finding problem. In
     this lecture the general root finding problem will be stated. Algorithms
     for the root finding problem will be presented in the next few lectures.
       [> > go there](https://jvkoebbe.github.io/math4610/lecture_02/html/finite_precision)
       | [(pdf)](https://jvkoebbe.github.io/math4610/lecture_02/pdf/git_primer.pdf)

  6. **Wrap up and Questions:**  If there is time and anyone has questions about
     the lecture, these will be addressed.

---

[prev](https://jvkoebbe.github.io/math4610/lectures/lecture_02/md/lecture_02) |
[toc](https://jvkoebbe.github.io/math4610/lectures/toc_lectures) |
[next](https://jvkoebbe.github.io/math4610/lectures/lecture_04/md/lecture_04)

---





























