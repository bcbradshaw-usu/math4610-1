\documentclass[10pt,fleqn]{article}

\setlength{\topmargin}{-.75in}
\addtolength{\textheight}{2.00in}
\setlength{\oddsidemargin}{.00in}
\addtolength{\textwidth}{.75in}

\nofiles

\pagestyle{empty}

\setlength{\parindent}{0in}

% new math commands


\setlength{\oddsidemargin}{-0.25in}
\setlength{\evensidemargin}{-0.25in}
\setlength{\textwidth}{6.75in}
\setlength{\headheight}{0.0in}
\setlength{\topmargin}{-0.25in}
\setlength{\textheight}{9.00in}

\makeindex

\usepackage{mathrsfs}

%\usepackage[pdftex]{graphicx}
\usepackage{epstopdf}

\newcounter{beans}

\newcommand{\ds}{\displaystyle}
\newcommand{\limit}[2]{\displaystyle\lim_{#1\to#2}}

\newcommand{\binomial}[2]{\ \left( \begin{array}{c}
                                  #1 \\
                                  #2
                                 \end{array}
                            \right) \
                         }
\newcommand{\ExampleRule}[2]
  {
  \noindent
  \rule{\linewidth}{1pt}
  \begin{example}
    #1
    \label{#2}
  \end{example}
  \rule{\linewidth}{1pt}
  \vskip0.125in
  }

\newcommand{\defbox}[1]
  {
   \ \\
   \noindent
   \setlength\fboxrule{1pt}
   \fbox{
        \begin{minipage}{6.5in}
          #1
        \end{minipage}
        }
   \ \\
  }
\newcommand{\verysmallworkbox}[1]
  {
   \ \\
   \noindent
   \setlength\fboxrule{1pt}
   \fbox{
        \begin{minipage}{6.5in}
           #1
           \ \\
           \vskip0.5in \ \\
           \ \\
        \end{minipage}
        }
   \ \\
  }
\newcommand{\smallworkbox}[1]
  {
   \ \\
   \noindent
   \setlength\fboxrule{1pt}
   \fbox{
        \begin{minipage}{6.5in}
           #1
           \ \\
           \vskip2.5in \ \\
           \ \\
        \end{minipage}
        }
   \ \\
  }
\newcommand{\halfworkbox}[1]
  {
   \ \\
   \noindent
   \setlength\fboxrule{1pt}
   \fbox{
        \begin{minipage}{6.5in}
           #1 \hfill
           \ \\
           \vskip3.25in \ \\
           \ \\
        \end{minipage}
        }
   \ \\
  }
\newcommand{\largeworkbox}[1]
  {
   \ \\
   \noindent
   \setlength\fboxrule{1pt}
   \fbox{
        \begin{minipage}{6.5in}
           #1
           \ \\
           \vskip7.5in \ \\
           \ \\
        \end{minipage}
        }
   \ \\
  }
\newcommand{\flexworkbox}[2]
  {
   \ \\
   \noindent
   \setlength\fboxrule{1pt}
   \fbox{
        \begin{minipage}{6.5in}
           #1
           \ \\

           \vskip#2 \ \\
           \ \\
        \end{minipage}
        }
   \ \\
  }


% symbols for sets of numbers

\newcommand{\natnumb}{$\cal N$}
\newcommand{\whonumb}{$\cal W$}
\newcommand{\intnumb}{$\cal Z$}
\newcommand{\ratnumb}{$\cal Q$}
\newcommand{\irrnumb}{$\cal I$}
\newcommand{\realnumb}{$\cal R$}
\newcommand{\cmplxnumb}{$\cal C$}

% misc. commands

\newcommand{\mma}{{\it Mathematica}}
\newcommand{\sech}{\mbox{ sech}}
 
\newtheorem{theorem}{Theorem}
\newtheorem{example}{Example}
\newtheorem{definition}{Definition}
\newtheorem{problem}{Problem}

\setcounter{secnumdepth}{2}
\setcounter{tocdepth}{4}


\begin{document}
%%%%%%%%%%%%%%%%%%%%%%%%%%%%%%%%%%%%%%%%%%%%%%%%%%%%%%%%%%%%%%%%%%%%%%%%%%%%%%%%
%%%%%%%%%%%%%%%%%%%%%%%%%%%%%%%%%%%%%%%%%%%%%%%%%%%%%%%%%%%%%%%%%%%%%%%%%%%%%%%%
\vskip0.1in\hrule\vskip0.1in
\noindent
{\bf Math 4610 Table of Contents for Lectures by Topic:}
\vskip0.1in\hrule\vskip0.1in
\noindent
The topics listed below form the content for lectues in the class. The course
will start with the first topic and will move through each topic in order. The
idea is to go through as many topics as possible. If a topic is not needed, it
can be skipped or maybe come back during a leter lecture. The topics are
self-contained. There is no set number of topics that can/will be covered in a
single lecture. For example, during the first class it is likely that three or
four topics will be covered based on the instructor and how many questions are
being asked.

The contents will likely change from semester to semester. So, the guide for
this course is the table of contents listed here.
%%%%%%%%%%%%%%%%%%%%%%%%%%%%%%%%%%%%%%%%%%%%%%%%%%%%%%%%%%%%%%%%%%%%%%%%%%%%%%%%
%%%%%%%%%%%%%%%%%%%%%%%%%%%%%%%%%%%%%%%%%%%%%%%%%%%%%%%%%%%%%%%%%%%%%%%%%%%%%%%%
\noindent
\begin{enumerate}
  \item Introductory Comments: This course presents fundamental content from
        numerical methods/analysis....
  \item The Syllabus for Math 4610 Fundamentals of Numerical Analysis: This
        course presents fundamental topics and algorithms that are common to
        many areas of computational mathematics....
  \item Github Account: You will create a repository with a specific name. The
        name will be 
        \begin{verbatim}
 
          math4610

        \end{verbatim}
  \item Building a Github Repository for Math 4610: You will learn how to build
        a repository on Github that will be used to turn in your homework and
        projects.
  \item Using a Terminal: It will be imperative that you are able to work in a
        terminal. There are a number of applications that can be used to bring
        up a terminal running a Linux/Unix operating system or an emulater. A
        few of these are PowerShell, Cygwin/MinGW terminals, Bash on Windows.
  \item A Few Shell Commands: A brief primer on how to use commands in a
        terminal are necessary.
  \item A coding example in C and Fortran: This example will show you how to use
        a command line terminal to write, compile, and execute a program.
  \item How to Build Shared Libraries: Reuse of code requires a way to store and
        link to the code.
  \item Testing the Library: 
  \item Using git to work locally: Having a repository on Github is great.
        However, to work locally, git, will let you work locally and easily
        transfer files to/from your repositories.
  \item Taylor Series Review: In an initial course in numerical analysis, Taylor
        series are used in the evaluation of accuracy.
  \item Compiling and Executing Computer Programs:
  \item Creating a Shared Library:
  \item Linking to a Shared Library:
  \item An Example of Arithmetic Accuracy - Approximating the Value of a
        Derivative: We will analyze the approximation of a first derivative with
        a finite difference quotient.
  \item Taylor Series Analysis of the Difference Quotient:
  \item Testing the Difference Quotient Example: We will write a code to test
        the difference quotient approximation.
  \item An Introduction to the Root Finding Problem:
  \item The Intermediate Value Theorem for Continuous Functions:
  \item The Bisection Method: A simple application of the Intermediate Value
        Theorem results in a root finding method.
  \item A Recursive Definition for the Bisection Method:
  \item Rewriting the Bisection Method for Efficiency:
  \item Truncation Error: A definition of truncation error is ....
  \item Roundoff Error:
  \item Absolute and Relative Error:
  \item Examples of Roundoff Error in Real Life: Roundoff error can accumulate
        in computationally intensive algorithms.
  \item Solving a linear system using Gaussian Elimination and Roundoff Error:
  \item Truncation Error vs. Roundoff Error in the Finite Difference
        Approximation:
  \item Bounds on Error: Interval Analysis and Accumulation of Roundoff Error:
  \item Stable Algorithms and Computation: related topic involves selecting
       \lq\lq stable\rq\rq algorithms for solution of real problems. Stable
        algorithms are less sensitive to accumulation of errors.
  \item FLOPS:
  \item Efficiency of an Algotihm: Counting the number of computations in the
        execution of an algorithm.
  \item Tradeoffs between Accuracy and Efficiency:
  \item Number of Operations for Bisection:
  \item Newton's Method for Finding Roots:
  \item The Secant Method:
  \item Review of Linear Systems: Gaussian Elimination:
  \item Review of Linear Systems: Back Substitution:
  \item Direct Methods versus Iterative Methods for Solving Linear Systems:
\end{enumerate}
\vskip0.1in\hrule\vskip0.1in
\end{document}

%%%%%%%%%%%%%%%%%%%%%%%%%%%%%%%%%%%%%%%%%%%%%%%%%%%%%%%%%%%%%%%%%%%%%%%%%%%%%%%%
  \item {\bf Lecture: Solution of Linear Systems of Equations} 
    \begin{list}{$\bullet$}{\usecounter{beans} \parsep=0pt \listparindent=0pt
    \topsep=0pt \rightmargin=.35in \leftmargin=.35in \labelsep=5 pt
    \itemsep=2pt}
      \item Questions from the previous lecture.
      \item Back-substitution and the solution of linear systems of equations.
      \item How to measure errors in the solution of linear equations in terms
            of norms.
      \item Standard norms in computational mathematics.
    \end{list}
%%%%%%%%%%%%%%%%%%%%%%%%%%%%%%%%%%%%%%%%%%%%%%%%%%%%%%%%%%%%%%%%%%%%%%%%%%%%%%%%
%%%%%%%%%%%%%%%%%%%%%%%%%%%%%%%%%%%%%%%%%%%%%%%%%%%%%%%%%%%%%%%%%%%%%%%%%%%%%%%%
  \item {\bf Lecture: Solution of Linear Systems of Equations} 
    \begin{list}{$\bullet$}{\usecounter{beans} \parsep=0pt \listparindent=0pt
    \topsep=0pt \rightmargin=.35in \leftmargin=.35in \labelsep=5 pt
    \itemsep=2pt}
      \item Questions from the previous lecture.
      \item LU factorization of square matrices.
      \item Rewrite of a linear system of equations using LU factorization.
      \item Forward substitution for intermediate solution of linear systems.
      \item LU factorization and solution of linear systems of equations.
    \end{list}
%%%%%%%%%%%%%%%%%%%%%%%%%%%%%%%%%%%%%%%%%%%%%%%%%%%%%%%%%%%%%%%%%%%%%%%%%%%%%%%%
%%%%%%%%%%%%%%%%%%%%%%%%%%%%%%%%%%%%%%%%%%%%%%%%%%%%%%%%%%%%%%%%%%%%%%%%%%%%%%%%
  \item {\bf Lecture: Solution of Linear Systems of Equations} 
    \begin{list}{$\bullet$}{\usecounter{beans} \parsep=0pt \listparindent=0pt
    \topsep=0pt \rightmargin=.35in \leftmargin=.35in \labelsep=5 pt
    \itemsep=2pt}
      \item Questions from the previous lecture.
      \item Comparisons for Gaussian elimination and LU factorization for
            solving linear systems of equations.
      \item Number of Floating Point Operations (flops) by introducing counters
            and the size of the linear system is increases.
    \end{list}
%%%%%%%%%%%%%%%%%%%%%%%%%%%%%%%%%%%%%%%%%%%%%%%%%%%%%%%%%%%%%%%%%%%%%%%%%%%%%%%%
%%%%%%%%%%%%%%%%%%%%%%%%%%%%%%%%%%%%%%%%%%%%%%%%%%%%%%%%%%%%%%%%%%%%%%%%%%%%%%%%
  \item {\bf Lecture: Solution of Linear Systems of Equations} 
    \begin{list}{$\bullet$}{\usecounter{beans} \parsep=0pt \listparindent=0pt
    \topsep=0pt \rightmargin=.35in \leftmargin=.35in \labelsep=5 pt
    \itemsep=2pt}
      \item Questions from the previous lecture.
      \item Reusable codes and putting together shared libraries from codes that
            can easily be used by others.
      \item Building examples of linear system of equations and using benchmarks
            to test codes. ORNL/nanet 
    \end{list}
%%%%%%%%%%%%%%%%%%%%%%%%%%%%%%%%%%%%%%%%%%%%%%%%%%%%%%%%%%%%%%%%%%%%%%%%%%%%%%%%
%%%%%%%%%%%%%%%%%%%%%%%%%%%%%%%%%%%%%%%%%%%%%%%%%%%%%%%%%%%%%%%%%%%%%%%%%%%%%%%%
  \item {\bf Lecture: Matrix Operations} 
    \begin{list}{$\bullet$}{\usecounter{beans} \parsep=0pt \listparindent=0pt
    \topsep=0pt \rightmargin=.35in \leftmargin=.35in \labelsep=5 pt
    \itemsep=2pt}
      \item Questions from the previous lecture.
      \item Computation of the sum of vectors.
      \item Computation of inner products of vectors.
      \item Computation of cross products of vectors.
      \item Computation of matrix-vector products.
      \item Computation of matrix-matrix products.
    \end{list}
%%%%%%%%%%%%%%%%%%%%%%%%%%%%%%%%%%%%%%%%%%%%%%%%%%%%%%%%%%%%%%%%%%%%%%%%%%%%%%%%
%%%%%%%%%%%%%%%%%%%%%%%%%%%%%%%%%%%%%%%%%%%%%%%%%%%%%%%%%%%%%%%%%%%%%%%%%%%%%%%%
  \item {\bf Lecture: Stationary Iterative Method for Solving Linear Systems}
    \begin{list}{$\bullet$}{\usecounter{beans} \parsep=0pt \listparindent=0pt
    \topsep=0pt \rightmargin=.35in \leftmargin=.35in \labelsep=5 pt
    \itemsep=2pt}
      \item Questions from the previous lecture.
      \item Functional Iteration for System Equations.
      \item Jacobi Iteration.
      \item Gauss-Seidel Iteration.
      \item Analysis of these methods.
    \end{list}
%%%%%%%%%%%%%%%%%%%%%%%%%%%%%%%%%%%%%%%%%%%%%%%%%%%%%%%%%%%%%%%%%%%%%%%%%%%%%%%%
%%%%%%%%%%%%%%%%%%%%%%%%%%%%%%%%%%%%%%%%%%%%%%%%%%%%%%%%%%%%%%%%%%%%%%%%%%%%%%%%
  \item {\bf Lecture: Stationary Iterative Method for Solving Linear Systems}
    \begin{list}{$\bullet$}{\usecounter{beans} \parsep=0pt \listparindent=0pt
    \topsep=0pt \rightmargin=.35in \leftmargin=.35in \labelsep=5 pt
    \itemsep=2pt}
      \item Questions from the previous lecture.
      \item Gauss-Seidel Iteration.
      \item Recursion versus explicit evaluation of iteration methods.
      \item Analysis of Gauss-Seidel
    \end{list}
%%%%%%%%%%%%%%%%%%%%%%%%%%%%%%%%%%%%%%%%%%%%%%%%%%%%%%%%%%%%%%%%%%%%%%%%%%%%%%%%
%%%%%%%%%%%%%%%%%%%%%%%%%%%%%%%%%%%%%%%%%%%%%%%%%%%%%%%%%%%%%%%%%%%%%%%%%%%%%%%%
  \item {\bf Lecture: Parallel Algorithms for Solving Linear Systems} 
    \begin{list}{$\bullet$}{\usecounter{beans} \parsep=0pt \listparindent=0pt
    \topsep=0pt \rightmargin=.35in \leftmargin=.35in \labelsep=5 pt
    \itemsep=2pt}
      \item Questions from the previous lecture.
      \item Compiler options for optimization of a code.
      \item Hello world using OpenMP. 
      \item Using OpenMp to parallelize Jacobi Iteration. 
    \end{list}
%%%%%%%%%%%%%%%%%%%%%%%%%%%%%%%%%%%%%%%%%%%%%%%%%%%%%%%%%%%%%%%%%%%%%%%%%%%%%%%%
%%%%%%%%%%%%%%%%%%%%%%%%%%%%%%%%%%%%%%%%%%%%%%%%%%%%%%%%%%%%%%%%%%%%%%%%%%%%%%%%
  \item {\bf Lecture: Finding roots of nonlinear functions of one variable} 
    \begin{list}{$\bullet$}{\usecounter{beans} \parsep=0pt \listparindent=0pt
    \topsep=0pt \rightmargin=.35in \leftmargin=.35in \labelsep=5 pt
    \itemsep=2pt}
      \item Questions from the previous lecture.
      \item The root finding problem.
      \item Functional Iteration for Root Finding.
      \item Convergence of Functional Iteration Via Fixed Point Theorems.
    \end{list}
%%%%%%%%%%%%%%%%%%%%%%%%%%%%%%%%%%%%%%%%%%%%%%%%%%%%%%%%%%%%%%%%%%%%%%%%%%%%%%%%
%%%%%%%%%%%%%%%%%%%%%%%%%%%%%%%%%%%%%%%%%%%%%%%%%%%%%%%%%%%%%%%%%%%%%%%%%%%%%%%%
  \item {\bf Lecture: Finding roots of nonlinear functions of one variable} 
    \begin{list}{$\bullet$}{\usecounter{beans} \parsep=0pt \listparindent=0pt
    \topsep=0pt \rightmargin=.35in \leftmargin=.35in \labelsep=5 pt
    \itemsep=2pt}
      \item Questions from the previous lecture.
      \item The Bisection Method.
      \item The Intermediate Theorem for Continuous Functions on a Closed
            Interval.
      \item Accuracy versus Number of Iterations in the Bisection Method.
    \end{list}
%%%%%%%%%%%%%%%%%%%%%%%%%%%%%%%%%%%%%%%%%%%%%%%%%%%%%%%%%%%%%%%%%%%%%%%%%%%%%%%%
%%%%%%%%%%%%%%%%%%%%%%%%%%%%%%%%%%%%%%%%%%%%%%%%%%%%%%%%%%%%%%%%%%%%%%%%%%%%%%%%
  \item {\bf Lecture: Estimating Eigenvalues of a Square Matrix} 
    \begin{list}{$\bullet$}{\usecounter{beans} \parsep=0pt \listparindent=0pt
    \topsep=0pt \rightmargin=.35in \leftmargin=.35in \labelsep=5 pt
    \itemsep=2pt}
      \item Questions from the previous lecture.
      \item The Rayleigh Quotient for Eigenpairs of a Square Matrix.
      \item The Power Method for estimating the largest eigenvalue of a matrix.
      \item Why the Power Method Works.
    \end{list}
%%%%%%%%%%%%%%%%%%%%%%%%%%%%%%%%%%%%%%%%%%%%%%%%%%%%%%%%%%%%%%%%%%%%%%%%%%%%%%%%
%%%%%%%%%%%%%%%%%%%%%%%%%%%%%%%%%%%%%%%%%%%%%%%%%%%%%%%%%%%%%%%%%%%%%%%%%%%%%%%%
  \item {\bf Lecture: Inverse Iteration for the Smallest Eigenvalue} 
    \begin{list}{$\bullet$}{\usecounter{beans} \parsep=0pt \listparindent=0pt
    \topsep=0pt \rightmargin=.35in \leftmargin=.35in \labelsep=5 pt
    \itemsep=2pt}
      \item Questions from the previous lecture.
      \item Properties of Eigenvalues of Matrices Using Shifting
      \item The Inverse Iteration Method for estimating the smallest of a
            matrix.
      \item Using Shifting, Parallel Implementation to Search for Eigenvalues
    \end{list}
%%%%%%%%%%%%%%%%%%%%%%%%%%%%%%%%%%%%%%%%%%%%%%%%%%%%%%%%%%%%%%%%%%%%%%%%%%%%%%%%
%%%%%%%%%%%%%%%%%%%%%%%%%%%%%%%%%%%%%%%%%%%%%%%%%%%%%%%%%%%%%%%%%%%%%%%%%%%%%%%%
  \item {\bf Lecture: The QR Factorization of a Matrix} 
    \begin{list}{$\bullet$}{\usecounter{beans} \parsep=0pt \listparindent=0pt
    \topsep=0pt \rightmargin=.35in \leftmargin=.35in \labelsep=5 pt
    \itemsep=2pt}
      \item Questions from the previous lecture.
      \item QR factorization of a matrix.
      \item Gram-Schmidt Orthogonalization of Vectors.
      \item Modified Gram-Schmift Orthogonalization for added stability of the
            computational algorithm.
    \end{list}
%%%%%%%%%%%%%%%%%%%%%%%%%%%%%%%%%%%%%%%%%%%%%%%%%%%%%%%%%%%%%%%%%%%%%%%%%%%%%%%%
%%%%%%%%%%%%%%%%%%%%%%%%%%%%%%%%%%%%%%%%%%%%%%%%%%%%%%%%%%%%%%%%%%%%%%%%%%%%%%%%
  \item {\bf Lecture: The QR Factorization for Solving Systems of Linear
         Equations} 
    \begin{list}{$\bullet$}{\usecounter{beans} \parsep=0pt \listparindent=0pt
    \topsep=0pt \rightmargin=.35in \leftmargin=.35in \labelsep=5 pt
    \itemsep=2pt}
      \item Questions from the previous lecture.
      \item The matrix operations needed to solve a linear system.
    \end{list}
%%%%%%%%%%%%%%%%%%%%%%%%%%%%%%%%%%%%%%%%%%%%%%%%%%%%%%%%%%%%%%%%%%%%%%%%%%%%%%%%
%%%%%%%%%%%%%%%%%%%%%%%%%%%%%%%%%%%%%%%%%%%%%%%%%%%%%%%%%%%%%%%%%%%%%%%%%%%%%%%%
  \item {\bf Lecture: The QR Factorization for Solving Least Squares
         Problems}
    \begin{list}{$\bullet$}{\usecounter{beans} \parsep=0pt \listparindent=0pt
    \topsep=0pt \rightmargin=.35in \leftmargin=.35in \labelsep=5 pt
    \itemsep=2pt}
      \item Questions from the previous lecture.
      \item Full Column Rank Problems and the thin QR.
    \end{list}
%%%%%%%%%%%%%%%%%%%%%%%%%%%%%%%%%%%%%%%%%%%%%%%%%%%%%%%%%%%%%%%%%%%%%%%%%%%%%%%%
%%%%%%%%%%%%%%%%%%%%%%%%%%%%%%%%%%%%%%%%%%%%%%%%%%%%%%%%%%%%%%%%%%%%%%%%%%%%%%%%
  \item {\bf Lecture: The QR Factorization for Eigenvalue Problems}
    \begin{list}{$\bullet$}{\usecounter{beans} \parsep=0pt \listparindent=0pt
    \topsep=0pt \rightmargin=.35in \leftmargin=.35in \labelsep=5 pt
    \itemsep=2pt}
      \item Questions from the previous lecture.
      \item The QR algorithm.
      \item Why the QR algorithm works.
    \end{list}
%%%%%%%%%%%%%%%%%%%%%%%%%%%%%%%%%%%%%%%%%%%%%%%%%%%%%%%%%%%%%%%%%%%%%%%%%%%%%%%%
%%%%%%%%%%%%%%%%%%%%%%%%%%%%%%%%%%%%%%%%%%%%%%%%%%%%%%%%%%%%%%%%%%%%%%%%%%%%%%%%
  \item {\bf Lecture: Quantum Computing - IBMQ}
    \begin{list}{$\bullet$}{\usecounter{beans} \parsep=0pt \listparindent=0pt
    \topsep=0pt \rightmargin=.35in \leftmargin=.35in \labelsep=5 pt
    \itemsep=2pt}
      \item Questions from the previous lecture.
      \item What is quantum computing?
      \item Accessing IBMQ.
    \end{list}
%%%%%%%%%%%%%%%%%%%%%%%%%%%%%%%%%%%%%%%%%%%%%%%%%%%%%%%%%%%%%%%%%%%%%%%%%%%%%%%%
%%%%%%%%%%%%%%%%%%%%%%%%%%%%%%%%%%%%%%%%%%%%%%%%%%%%%%%%%%%%%%%%%%%%%%%%%%%%%%%%
\end{enumerate}
\end{document}
